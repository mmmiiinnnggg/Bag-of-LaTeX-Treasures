\phantomsection
\addcontentsline{toc}{section}{Приложения}
\section*{Приложения}
\noindent\textbf{Реализация регуляризованного экстраградиентного метода на Си++ для модельных примеров.}
\begin{lstlisting}[caption={Реализация метода на Си++}]
#include <iostream>
#include <cmath>

using namespace std;

const int N = 5000;
const float v_start = 1.0;
const float eps = 0.01;

float gradPsiError(float v, float w, float delta) { return v + delta*w;}
\\ For the second example: { return 2*w*v + delta*w;}
float gradPError(float w, float delta){ return (w<=-1)? -1.0 + delta: ((w>=1)? 1.0+delta: delta);}

int main()
{
	int k = 1;
	float v = v_start;
	while(k <= N){
		float alpha = 1.0/(k+1);
		float A = k+1;
		float delta = 1.0/(k+1);
		float beta = 1.0/(2+2*A);
		float u = v - beta * (gradPsiError(v, v, delta) + A * gradPError(v, delta));
		float v = v - beta * ( gradPsiError(v, u, delta) + A * gradPError(u, delta) - 2*alpha*(u-v));
		k++;
		cout << "k=" << k << " v=" << v << endl;
	}
	return 0;
}

\end{lstlisting}
