\phantomsection
\addcontentsline{toc}{section}{Список литературы}
\section*{Список литературы}
	\begingroup
		\renewcommand{\section}[2]{}%
		\begin{thebibliography}{}
		% textbooks
		\bibitem{numMethodopt} \emph{Васильев Ф.П.} Численные методы решения экстремальных задач. М.: Наука, 1988.
		\bibitem{5} \emph{Васильев Ф.П.} Методы оптимизации. М.: Факториал, 2002.
		\bibitem{num-methods} \emph{Самарский А.А., Гулин А.В.} Численные методы. М.: Наука, 1989.
		
		% main references
		\bibitem{centralbib} \textit{Антипин A.C., Васильев Ф.П.} Метод стабилизации для решения задач равновесного программирования // Ж. вычисл. матем. и матем. физ. 1999. Т. 39. № 11. С 1779-1786.
		\bibitem{centralbib-2}  \emph{Антипин А.С., Васильев Ф.П., Шпирко С.В.} Регуляризованный экстраградиентный метод решения задач равновесного программирования с неточно заданным множеством. // Ж. вычисл. матем. и матем. физ. 2005, том 45, №4, с. 650–660.
		
		% others main references
		\bibitem{4} \emph{Шпирко C.B.} Метод кососимметричной регуляризации для решения задач равновесного программирования: Дис. ... канд. физ.-матем. наук. М.: ВЦ РАН, 2001.
		\bibitem{6} \emph{Шпирко С.В.} О существовании и единственности решения задачи равновесного программирования, Изв. вузов. Матем., 2002, номер 12, 79–83
		
		% methods 
		\bibitem{7} \textit{Антипин A.C.} Метод внутренней линеаризации для задач равновесного программирования,Ж. вычисл. матем. и матем. физ., 2000, том 40, номер 8, 1142–1162.
		\bibitem{8} \textit{Антипин A.C.} Равновесное программирование: методы градиентного типа // Автоматика и телемехан., 1997. №8. С. 125-137.
		\bibitem{8-2} \textit{Антипин А.С.} Вычисление неподвижных точек экстремальных отображений при помощи методов градиентного типа, Ж. вычисл. матем. и матем. физ., 1997, том 37, номер 1, 42–53.
		\bibitem{9} \textit{Антипин A.C.} Равновесное программирование: проксимальные методы, Ж. вычисл. матем. и матем. физ., 1997, том 37, номер 11, 1327–1329.
		\bibitem{10} \emph{Будак Б.А.} Непрерывные методы решения задач равновесного программирования: Дис. ... канд. физ.-матем. наук. М.: ВЦ РАН, 2003.
		\bibitem{10-2} \emph{Будак Б.А.} Метод стрельбы для решения задач равновесного программирования, Ж. вычисл. матем. и матем. физ., 2013, том 53, номер 12, 2008–2013.

		
		% methods of regulazition
		\bibitem{11} \textit{Васильев Ф.П., Антипин A.C.} Методы регуляризации поиска неподвижной точки экстремальных отображений // Вестн. МГУ. Сер. 15. Вычисл. матем. и матем. физ. 1998. Т. 38. № 1. С. 11-14.
		\bibitem{12} \textit{Антипин А.С., Васильев Ф.П.} Метод квазирешений для решения равновесных задач с неточно заданным
		множеством // Вестн. ун-та Дружбы народов. 2001. № 8(2). С. 10-16.
		\bibitem{13} \textit{Антипин A.C., Васильев Ф.П.} Метод невязки для решения равновесных задач с неточно заданным множеством // Ж. вычисл. матем. и матем. физ. 2001. Т. 41. № 1. С. 3-8.
		\bibitem{14} \textit{Антипин A.C., Васильев Ф.П.} Методы регуляризации для решения задачи равновесного программирования с неточными входными данными, основанные на расширении множества // Ж. вычисл. матем.
		и матем. физ. 2002. Т. 42. № 8. С. 1158-1165.
		
		% concrete methods of regulazition
		\bibitem{15} \textit{Антипин A.C., Васильев Ф.П., Шпирко C.B.} Регуляризованный экстраградиентный метод решения задач
		равновесного программирования // Ж. вычисл. матем. и матем. физ. 2003. Т. 43. № 10. С. 1451-1458.
		\bibitem{16} \textit{Будак Б.А.} Регуляризованный непрерывный метод линеаризации первого порядка прогнозного типа с переменной метрикой для решения задач равновесного программирования с неточно заданным множеством // Ж. вычисл. матем. и матем. физ. 2005. Т. 45. № 4. С. 637-649.
		
		
		\end{thebibliography}
	\endgroup
	\clearpage