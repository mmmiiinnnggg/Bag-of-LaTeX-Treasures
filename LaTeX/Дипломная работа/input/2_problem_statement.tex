%Разработка метода прогнозирования слабой масштабируемости суперкомпьютерных приложений
\phantomsection 
\addcontentsline{toc}{section}{Постановка задачи}
\counterwithout{equation}{section} 
\section*{Постановка задачи}
	Основную задачу равновесного программирования обычно формулируют таким образом: пусть имеется некоторая функция $\Phi(v,w)$, заданная на декартовом произведении $\mathbf{W}\times \mathbf{W}$, где $\mathbf{W}=\{w\in \mathbf{W_0}\subseteq\mathbb{R}^n:g_1(w)\leqslant 0,...,g_m(w)\leqslant 0 \}$ - заданное множество из пространства $\mathbb{R}^n$, $\mathbf{W_0}$ - множество из $\mathbb{E}^n$ (обычно имеет простой вид, возможно $\mathbf{W_{0}}=\mathbb{E}^n$). Требуется найти точку $v_*$ из $\mathbf{W}$, удовлетворяющую неравенству 
	\begin{equation}
	\centering
	\label{intro-1}
	\Phi(v_*,v_*)\leqslant\Phi(v_*,w),\quad \forall w\in \mathbf{W}.
	\end{equation}
	Такую точку называют \emph{равновесной точкой} задачи \eqref{intro-1}. Если функция $\Phi(v,w)$ не зависит от переменной $v$, то задача \eqref{intro-1} превращается в обычную \textit{задачу математического программирования}.
	
	Для этой задачи разработаны различные конструктивные методы, например экстраградиентный метод, проксимальный метод, метод линеаризации, метод стрельбы, которые подробно обсуждаются в следующей главе.  
	
	Следует учитывать, что эта задача, вообще говоря, \textbf{неустойчива} к возмущениям функции $\Phi(v,w)$ и множества $\mathbf{W}$, о чём свидетельствует простой \textbf{пример}:
	
	Допустим, что для возмущения функции и её градиента выполняются условия
	\begin{equation}
	\label{2}
	\lim\limits_{\delta\to 0} |\Phi^{\delta}(v,w)-\Phi(v,w)|=0;\quad  \lim\limits_{\delta\to 0} \|\nabla_w^{\delta}\Phi(v,w)-\nabla_w\Phi(v,w)\|=0
	\end{equation}
	Пусть $\Phi(v,w)=vw$, за приближенную функцию возьмём
	\begin{equation*}
	\Phi^{\delta}(v,w)=vw-\delta w^2,\mathbf{W}=\{w\in \mathbb{E}^1:|w|\leqslant 1 \}.
	\end{equation*}
	 Тогда $\nabla_w \Phi^{\delta}(v,w)=v-2\delta w$. Очевидно, что выполняется условие \eqref{2}, и множество решений $\mathbf{W_*}$ состоит из единственной точки $v_*=0$, однако приближенная задача
	\begin{equation}
	v_*^{\delta}\in \mathbf{W},\quad\Phi^{\delta}(v_*^{\delta},v_*^{\delta})\leqslant \Phi^{\delta}(v_*^{\delta},w),\quad \forall w\in \mathbf{W}
	\end{equation}
	не имеет решения при любых сколь угодно малых $\delta >0$, хотя здесь множество $\mathbf{W}$ - выпукло и компактно.
	
	 Понятно, что простое приближение задачи не даёт положительный результат, и для её решения нужно использовать методы, реализующие \textbf{идею регуляризации}, выдвинутую академиком А.Н.Тихонова. Как и ожидалось, существуют регуляризованные варианты методов решения задачи, опирающиеся на обычные упомянутые ранее методы Они будут подробно рассмотрены в следующей главе.
	 
	 Одно из важных понятий в теории равновесного программирования является \textit{сильной кососимметричностью}. Если выполнено
	 	\begin{equation*}
	 	\label{strong-koso}
	 	\Phi(w,w)-\Phi(w,v)-\Phi(v,w)+\Phi(v,v)\geqslant \mu\|v-w\|^2,\quad \forall w,v\in\mathbf{W_0},
	 	\end{equation*}
	 где $\mu>0$ - какое-то постоянное, то говорят, что функция $\Phi(v,w)$ сильно кососимметрична. 
	 
	 Легко проверяются, что $\langle v,w\rangle$ и $-\|v-w\|^2$ - кососимметричны:
	 \begin{equation*}
	 \begin{aligned}
	 &\langle v,w\rangle:\Phi(w,w)-\Phi(w,v)-\Phi(v,w)+\Phi(v,v)=1\cdot \|v-w\|^2,\forall v,w\in \mathbb{R}^n,\\
	 & -\|v-w\|^2:\Phi(w,w)-\Phi(w,v)-\Phi(v,w)+\Phi(v,v)=2\cdot\|v-w\|^2,\forall v,w\in \mathbb{R}^n.
	 \end{aligned}
	 \end{equation*}
	 В регуляризованных методах широко используется стабилизатор вида $\alpha_k\langle v,w\rangle$, но интересно и рассмотреть стабилизатор $-\alpha_k\| v-w\|^2$, который обладает рядом хороших свойств с точки зрения решения возмущенной задачи (например, свойством сильной кососимметричности, см. \cite{6}. Аналог этого свойства во выпуклом анализе - сильная выпуклость). 
	 
	 Исследование регуляризованных методов, в которых применен стабилизатор вида квадрата нормы, является \textbf{главной задачей} данной работы. Сразу стоит отметить, что данная работа не рассматривает все варианты конструктивных методов, а в основном концентрируется на одном - а именно на \textit{экстраградиентном методе}. 

\clearpage
% \newpage
%На счёт же третьей постановки - если мы ставим задачу разработки программного средства, то это средство нужно будет предъявить, и к нему будут предъявлять все обычные требования как к программному средству по оформлению, документированию и т.д., об этом тогда тоже не забудьте позаботиться.