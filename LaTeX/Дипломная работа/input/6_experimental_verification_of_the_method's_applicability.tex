\section{Численная проверка применимости методов}
Применимость метода оценивалась с помощью запусков программ на модельных примерах. 

Итак, в постановке задачи был рассмотрен простой
\begin{example}
	\label{exp-1}
	$\Phi(v,w)=vw,\mathbf{W}=\{w\in \mathbb{E}^1:|w|\leqslant 1 \},\mathbf{W_*}=\{0 \}.$
\end{example}
Как мы уже поняли, приближенная задача
\begin{equation}
v_*^{\delta}\in \mathbf{W},\quad\Phi^{\delta}(v_*^{\delta},v_*^{\delta})\leqslant \Phi^{\delta}(v_*^{\delta},w),\quad \forall w\in \mathbf{W}
\end{equation}
не имеет решения при любых сколь угодно малых $\delta >0$, то есть исходная задача имеет \textit{неустойчивый характер}.

Протестируем регуляризованный экстраградиентный метод, который предложен и рассмотрен в предыдущей главе. 

Сначала установим справедливость условий метода. Ограничения-неравенства можно записать в следующем виде: $g_1(w)=w-1,g_2(w)=-w-1$. Тогда штрафная функция $P(w)=g_1^+(w)+g_2^+(w)$ равна $-w-1$ при $w\leqslant -1$; $0$ при $-1\leqslant w\leqslant 1$ и $w-1$ при $w\geqslant 1$. Градиент $\nabla_w \Phi(v,w)=v.$ В качестве приближений градиентов возьмём $\nabla_w^k\Phi(v,w)=v+\delta_k w,\nabla_w^kP(w)=\nabla_k P(w)+\delta_k$. Понятно, что выполняются условия приближения \eqref{[3]-6}:
\begin{equation*}
\begin{aligned}
&\|\nabla_w^k\Phi(v,w)-\nabla_w\Phi(v,w)\|=\delta_k\|w\|\leqslant \delta_k(1+\|v\|+\|w\|),\\
&\|\nabla_w^kP(w)-\nabla_wP(w)\|=\delta_k\leqslant \delta_k(1+\|w\|).
\end{aligned}
\end{equation*}
А также ясно, что градиенты функции удовлетворяют условиям Липшица \eqref{[3]-29}\eqref{[3]-29-2}. Функция Тихонова выпукла по $w$, так как:
\begin{equation*}
\big\langle (T_k)_w^{''}(v,w)h,h\big\rangle = \langle 2\alpha_k h,h \rangle = 2\alpha_k\|h\|^2\geqslant 0,\forall h\in \mathbb{E}^1
\end{equation*}
Другие условия на функции $\Phi(v,w),g_i(w),i=1,...,m+s$ и множества $\mathbf{W},\mathbf{W_{0}}$ выполняются, что проверяется легко. Сразу заметим, что здесь множество $\mathbf{W_{0}}$ - вся числовая прямая $\mathbb{E}^1$, поэтому в предложенном методе операции проектирования являются излишними. 

Параметры $\alpha_k,A_k,\delta_k,\beta_k$, использованные в алгоритме, взяты из замечания \ref{remark 3-2} при $\alpha = 1, \delta =1, A = 1, L=1$.

Схема алгоритма выглядит так:
\newpage
\begin{table}[!htbp]
	\begin{tabular}{l}
		\toprule
		 \textbf{Алгоритм:} Регуляризованный экстраградиентный метод для примера \eqref{exp-1}. \\
		\midrule
		\textbf{Даны:} $\nabla_w^k\Phi(v,w;\delta_k) = v+\delta_kw$, и произвольное $v_0\in\mathbb{E}^1$.\\
		Для $k=1,2,...$ делаем цикл, пока $k$ меньше максимального числа итераций: \\
		%\textbf{1.} $\displaystyle \alpha_k = \frac{1}{\sqrt[5]{k+1}},A_k=\sqrt[5]{k+1},\delta_k = \frac{1}{k+1},\beta_k = \frac{1}{2(1+A_k)}$\\
		\textbf{1.} $\displaystyle \alpha_k = \frac{1}{k+1},A_k= k+1,\delta_k = \frac{1}{k+1},\beta_k = \frac{1}{2(1+A_k)}$\\
		\textbf{2.}  $ u_k = v_k-\beta_k\left[ \nabla_w^k\Phi(v_k,v_k;\delta_k)+A_k\nabla_wP(v_k;\delta_k)\right]$\\
		\textbf{3.}  $ v_{k+1} = v_k -\beta_k\left[\nabla_w^k\Phi(v_k,u_k;\delta_k)+A_k\nabla_wP(u_k;\delta_k)-2\alpha_k(u_k-v_k)\right]$\\
		\bottomrule
	\end{tabular}
	\caption{\label{tab:test-1} Схема алгоритма для примеров.}
\end{table}
Его реализация на Си++ доступен в \textbf{Приложениях}. Это алгоритм даёт положительный результат: При $k=7$ значение $v_k$ начинает уменьшаться; при $k=5000$ имеем $v=0.00812547$, что уже меньше $1\%$ отличается от решения $v_*=0$. Это означает, что данный вычислительный процесс сходится к реальному решению исходной задачи с нужной нам точностью.

На другом примере покажем, что метод хорошо работает в случае, когда число решений \textit{больше одного}.
\begin{example}
	\label{exp-2}
	$\Phi(v,w)=vw^2,\mathbf{W}=\{w\in \mathbb{E}^1:|w|\leqslant 1 \},\mathbf{W_*}=\{0,-1 \}.$
\end{example}
Видно, что у этой задачи есть два решения, из них нормальным решение является $v_*=0$. В принципе, эта схема алгоритма отличается от прошлого примера лишь заданием $\nabla_w^k\Phi(v,w)$. При тестировании на программе, получается что при $k=5000$ имеем $v_k=0.000750162$, что меньше чем на $0.1\%$ отличается от нормального решения $v_*=0$. Это в свою очередь показывает, что метод сходится к исходому решению.

Полученные результаты показывают, что методы, предложенные в работе, могут быть использованы для решения равновесных задач, неустойчивых к погрешностям задания исходных данных.
\clearpage