\section{Регуляризованный экстраградиентный метод}
В прошлой главе был рассмотрен метод стабилизации, но не рассматривался конкретный метод поиска точек равновесия функции Тихонова. В этой главе исследуем один из конкретных методов - \textit{экстраградиентный метод.}

Рассмотрим следующий итеративный процесс
\begin{equation}
\label{[3]-7}
\begin{aligned}
&u_k=\mathrm{Pr}_{\mathbf{W_0}}(v_k-\beta_k[\nabla_w^k\Phi(v_k,v_k)+A_k\nabla_w^kP(v_k)]),\\
v_{k+1}=\mathrm{Pr}_{\mathbf{W_0}}&(v_k-\beta_k[\nabla_w^k\Phi(v_k,u_k)+A_k\nabla_w^kP(u_k)-2\alpha_k(u_k-v_k)]),\quad k=0,1,...
\end{aligned}
\end{equation}

В целом он похож на градиентный метод прогнозного типа, который был в обзоре существующих методов, только вместо исходной функции взято её приближение, и добавлены штрафный член и производная стабилизатора. Введем функцию Тихонова 
\begin{equation}
\label{new-Tikhonov}
\begin{aligned}
T_k(v,w)&=\Phi(v,w)+A_kP(w)-\alpha_k\|v-w\|^2,\quad v,w\in\mathbf{W_0},\\
& A_k>0,\alpha_k>0,\quad k=0,1,...
\end{aligned}
\end{equation}

Предполагается, что функции $\Phi(v,w),g_i(w),i=1,..., s$ - дифференцируемы на $\mathbf{W_0}$ и градиент функции \eqref{new-Tikhonov} по $w$ существует и равен
\begin{equation}
\label{[3]-5}
\begin{aligned}
&\nabla_wT_k(v,w)=\nabla_w\Phi(v,w)+A_k\nabla_wP(w)-2\alpha(w-v),\\
&v,w\in\mathbf{W_0},A_k>0,\alpha_k>0,k=0,1,...
\end{aligned}
\end{equation}

Пусть вместо точных значений $\nabla_w\Phi(v,w),\nabla_wP(w)$ известны последовательности приближений $\{\nabla_w^k\Phi(v,w)\},\{\nabla_w^kP(w)\}$ такие, что
\begin{equation}
\label{[3]-6}
\begin{aligned}
&\|\nabla_w^k\Phi(v,w)-\nabla_w\Phi(v,w)\|\leqslant \delta_k(1+\|v\|+\|w\|),\quad \forall v,w\in \mathbf{W_0},\\
&\|\nabla_w^kP(w)-\nabla_wP(w)\|\leqslant \delta_k(1+\|w\|),\quad\forall w\in \mathbf{W_0},\delta_k>0,k=0,1,... 
\end{aligned}
\end{equation}

\subsection{Вспомогательные теоремы}
Была предложена следующая теорема о выпуклости и замкнутости множества решений $\mathbf{W_{*}}$. В ней также установлена единственность нормального решения. 
\begin{theo}
	\label{thero-2-1}
	Пусть выполнены следующие условия\\
	\textbf{1)} $\mathbf{W}_0$ - выпуклое замкнутое множество из $\mathbb{E}^n$,\\
	\textbf{2)} функция $\Phi(v,w)$ непрерывна по совокупности переменных $(v,w)\in\mathbf{W_0}\times\mathbf{W_0}$, выпукла по переменной $w$ на $\mathbf{W_0}$ при каждом фиксированном $v\in\mathbf{W_0}$, удовлетворяет условию кососимметричности \eqref{cososymmetric}. \\
	\textbf{3)} Функции $g_i(w)$ при $i=1,...,m$ непрерывны, выпуклы на $\mathbf{W_0}$; функция $g_i(w)$ при $i=m+1,...,s$ аффинны, т.е. представимы в виде $g_i(w)=\langle a_i,w \rangle-b_i,a_i\in\mathbb{E}^n,b_i\in\mathbb{E}^1$. \\
	\textbf{4)} Пусть множество $\mathbf{W_*}$ решений задачи \eqref{question} непусто. \\
	Тогда $\mathbf{W_*}$ выпукло, замкнуто и задача \eqref{question} имеет единственное решение $v_*$, имеющее минимальную среди всех решений норму и называемое нормальным решением задачи \eqref{question}.
\end{theo}
\noindent\emph{Доказатеьство.} см. \cite{centralbib-2} cтр. 3. $\quad\qedsymbol$

Далее изучим поведение последовательности $\{z_k \}$ точек равновесия  функции \eqref{new-Tikhonov}, определяемых условием 
\begin{equation}
\label{new-Tikhonov-condition}
z_k\in\mathbf{W_0},\quad T_k(z_k,z_k)\leqslant T_k(z_k,w)\quad \forall w\in\mathbf{W_0},\quad k=0,1,...
\end{equation}
Подобные результаты приведены в следующей теореме, её доказательство использует факты предыдущей теоремы.
\begin{theo}
	\label{thero-2-2}
	Пусть выполнены условия теоремы \ref{thero-2-1}, и\\
	\textbf{1)} функции $\Phi(v,w),g_i(w),i=1,...,m$, имеют непрерывные градиенты $\nabla_w\Phi(v,w)$, $\nabla_w g_i(w)$ на $\mathbf{W_0}$; \\
	\textbf{2)} функция Тихонова \eqref{new-Tikhonov} выпукла по $w$ при каждом $k=0,1,...$; \\
	\textbf{3)} существуют постоянные $v>0,c_i\geqslant 0,i=1,...,s$, такие, что 
	\begin{equation}
	\label{norm-condition}
	\Phi(v_*,v_*)\leqslant \Phi(v_*,w)+\sum_{i=1}^{s}c_i(g_i^+(w))^v,\quad \forall w\in\mathbf{W_0},
	\end{equation}
	где $v_*$ - нормальное решение задачи \eqref{question};\\
	\textbf{4)} параметр $p$ из штрафной функции \eqref{penalty} удовлетворяет условиям $p\geqslant v,p>1$; последовательности $\{\alpha_k \},\{A_k \}$ таковы, что
	\begin{equation}
	\label{parameter-condition-2}
	\begin{aligned}
	&\alpha_k>0,\quad A_k>0,\quad k=0,1,...,\quad\\
	&\lim\limits_{k\rightarrow\infty}\alpha_k=0,\quad \lim_{k\rightarrow\infty}A_k=+\infty,\quad\lim_{k\rightarrow\infty}\alpha _kA_k^{v/(p-v)}=+\infty.
	\end{aligned}
	\end{equation}
	(при $p=v$ последнее условие не нужно).
	
	Тогда точки $z_k$, удовлетворяющие условиям \eqref{new-Tikhonov-condition}, существуют, однозначно определяются при каждом $k=0,1...$ и таковы, что
	\begin{equation}
	\label{[3]-12}
	\begin{aligned}
	\|z_k\|\leqslant R_k&\leqslant \sup\limits_{k\geqslant 0}R_k=R,\quad k=0,1,...,\\
	R_k=\left(\frac{B}{\alpha_k A_k^{v/(p-v)}} \right)^{1/2}+\|v_*\|&,\quad B=(p-v)v^{v/(p-v)}p^{-p/(p-v)}|c|^{p/(p-v)},\\
	|c|=\left(\sum\limits_{i=1}^{s}c_i^{p/(p-v)} \right)^{(p-v)/p}& \text{при } p>v,\quad R_k=\|v_*\| \text{при } p=v,
	\end{aligned}
	\end{equation}
	\begin{equation}
	\label{[3]-13}
	\lim_{k\rightarrow\infty}A_kP(z_k)=0,
	\end{equation}
	\begin{equation}
	\label{[3]-14}
	\lim_{k\rightarrow\infty}\|z_k-v_*\|=0,
	\end{equation}
	\begin{equation}
	\label{[3]-15}
	\begin{aligned}
	\|z_k-z_m\|&\leqslant \frac{|A_m-A_k|R_1}{2\alpha_k},\quad \forall k,m=0,1,...\\
	R_1&=\max\limits_{\|w\|\leqslant R}\|\nabla_w P(w)\|.
	\end{aligned}
	\end{equation}
\end{theo}
\begin{remark}
Содержательные классы задач, для которых выполнено условие \eqref{norm-condition}, приведены в \cite{centralbib}.
\end{remark}
\noindent\emph{Доказательство.} Из определения \eqref{new-Tikhonov} функции $T_k(v,w)$ и условия кососимметричности \eqref{cososymmetric} следует, что
\begin{equation}
\label{[3]-16}
T_k(v,v)-T_k(v,w)-T_k(w,v)+T_k(w,w)\geqslant 2\alpha_k\|v-w\|^2,\quad \forall v,w\in\mathbf{W_0}.
\end{equation}
Кроме того, функция $T_k(v,w)$ непрерывна по $(v,w)\in\mathbf{W_0}\times\mathbf{W_0}$, выпукла и непрерывно дифференцируема по $w$ на $\mathbf{W_0}$. Тогда и из \cite{4} следует, что условие \eqref{new-Tikhonov-condition} однозначно определяет точку $z_k$ при каждом $k=0,1,...$.

Докажем оценку \eqref{[3]-12}. Заметим, что из \eqref{cososymmetric} при $v=v_*$ и \eqref{norm-condition} вытекают неравенства
\begin{equation}
\label{[3]-17}
\Phi(w,v_*)-\Phi(w,w)\leqslant \Phi(v_*,v_*)-\Phi(v_*,w)\leqslant \sum_{i=1}^{s}c_i(g_i^+(w))^v,\quad \forall w\in\mathbf{W_0}.
\end{equation} 
Положив в \eqref{[3]-16} $v=z_k,w=v_*$, c учетом \eqref{new-Tikhonov}\eqref{new-Tikhonov-condition}\eqref{[3]-17}, $P(v_*)=0$ и предложения \ref{propsition-1} получим
\begin{equation}
\label{[3]-19}
\begin{aligned}
&2\alpha_k\|z_k-v_*\|^2\leqslant \overbrace{T_k(z_k,z_k)-T_k(z_k,v_*)}^{\leqslant 0\text{ в силу }\eqref{new-Tikhonov-condition}}-T_k(v_*,z_k)+T_k(v_*,v_*)\stackrel{\eqref{new-Tikhonov}}{\leqslant}\\
&\leqslant\Phi(v_*,v_*)-\Phi(v_*,z_k)+\underbrace{A_kP(v_*)}_{=0}-A_kP(z_k)\underbrace{-\alpha_k\|v_*-v_*\|^2}_{=0}+\alpha_k\|v_*-z_k\|^2\stackrel{\eqref{[3]-17}}{\leqslant}\\
&\leqslant \sum_{i=1}^{s}c_i(g_i^+(w))^v-A_kP(z_k)+\alpha_k\|v_*-z_k\|^2\stackrel{\text{Предлож. 1}}{\leqslant}\\
&\leqslant \cancel{A_kP(z_k)}+BA_k^{-v/(p-v)}\cancel{-A_kP(z_k)}+\alpha_k\|v_*-z_k\|^2=BA_k^{-v/(p-v)}+\alpha_k\|v_*-z_k\|^2
\end{aligned}
\end{equation}
Отсюда, получим
\begin{equation}
\begin{aligned}
&\alpha_k\|z_k-v_*\|^2\leqslant BA_k^{-v/(p-v)},\quad \|z_k-v_*\|\leqslant\left(\frac{B}{\alpha_kA_k^{v/(p-v)}}\right)^{1/2},\\
&\|z_k\|=\|z_k-v_*+v_*\|\leqslant\|z_k-v_*\|+\|v_*\|\leqslant \left(\frac{B}{\alpha_kA_k^{v/(p-v)}}\right)^{1/2}+\|v_*\|=R_k
\end{aligned}
\end{equation} 
Это равносильно \eqref{[3]-12} при $p>v$. Аналогично доказывается и при $p=v$.

Докажем \eqref{[3]-13}. Из \eqref{[3]-19} выделяем первое и четвертое выражения, получаем 
\begin{equation}
\label{[3]-20}
A_kP(z_k)\leqslant\sum_{i=1}^{s}c_i(g_i^+(w))^v-\alpha_k\|v_*-z_k\|^2
\end{equation} 
Если $p>v$, то пользуясь неравенством Гёльдера
\begin{equation*}
\sum_{i=1}^{s}a_ib_i\leqslant\left(\sum_{i=1}^{s}a_i^q \right)^{1/q}\left(\sum_{i=1}^{s}b_i^r \right)^{1/r},\quad \forall a_i,b_i\geqslant 0,r,q>1,\frac{1}{q}+\frac{1}{r}=1
\end{equation*} 
при $q=p/v,r=p/(p-v),a_i=g_i^+(z_k)^v,b_i=c_i$ имеем
\begin{equation*}
\begin{aligned}
&\sum_{i=1}^{s}[g_i^+(v_{\delta})]^v c_i=\left(\sum_{i=1}^{s}[g_i^+(z_k)]^p \right)^{v/p}\cdot\left(\sum_{i=1}^{s}c_i^{p/(p-v)} \right)^{(p-v)/p}=|c|(P(z_k))^{v/p}\\
&\Rightarrow 0\leqslant AP(z_k)\leqslant |c|(P(v_{\delta}))^{v/p}-\alpha\|v_*-z_k\|^2\leqslant|c|(P(v_{\delta}))^{v/p},\quad p>v.
\end{aligned}
\end{equation*}
Пусть $x=[A_kP(z_k)]^{v/p}$, тогда можно переписать предыдущее неравенство в виде
\begin{equation*}
0\leqslant x^{p/v}\leqslant|c|A_k^{-v/p}x
\end{equation*} 
Отсюда в силу предложения \ref{lemma11} получаем
\begin{equation}
\label{[3]-21}
0\leqslant A_kP(z_k)=x^{p/v}\leqslant|c|^{p/(p-v)}A_k^{-v/(p-v)}.
\end{equation}
Если $p=v$, то из \eqref{[3]-20} сразу находим $A_kP(z_k)\leqslant |c|_{\infty}P(z_k)-\alpha_k\|v_*-z_k\|^2\leqslant |c|_{\infty}P(z_k)$ для всех $k>k_0$, когда $A_k\geqslant |c|_{\infty}$, и
\begin{equation}
\label{[3]-22}
0\leqslant A_kP(z_k)\leqslant 0\quad \Rightarrow\quad A_kP(z_k)=0
\end{equation}
Из \eqref{[3]-21}\eqref{[3]-22} с учетом \eqref{parameter-condition-2} получим
\begin{equation*}
\lim_{k\rightarrow\infty}A_kP(z_k)=0=\lim_{k\rightarrow\infty}|c|^{p/(p-v)}\overbrace{A_k^{-v/(p-v)}}^{\rightarrow 0, A_k\rightarrow+\infty}\quad p>v,\quad\lim_{k\rightarrow\infty}A_kP(z_k)=0,\quad p=v
\end{equation*}

Докажем равенство \eqref{[3]-14}. Из \eqref{[3]-13} получаем $\lim\limits_{k\rightarrow\infty}P(z_k)=0$. В силу \eqref{penalty} получаем $\lim\limits_{k\rightarrow\infty}g_i^+(z_k)=0,i=1,...,s$, или 
\begin{equation}\label{[3]-23}
\varlimsup_{k\rightarrow \infty}g_i(z_k)\leqslant 0,\quad i=1,...,m,\quad \lim_{k\rightarrow\infty}g_i(z_k)=0,\quad i=m+1,...,s.
\end{equation}
Из оценки \eqref{[3]-12} и теоремы Больцано-Вейерштрасса вытекает, что последовательность $\{z_k\}$ имеет хотя бы одну предельную точку $u_*$(или сходящуюся подпоследовательность $\{z_{k_l}\}\rightarrow u_*$). Так как $\{z_k\}\in\mathbf{W_0}$ и $\mathbf{W_0}$ замкнуто, то $u_*\in\mathbf{W_0}$. Из \eqref{[3]-23} и непрерывности $g_i(w)$ следует, что $g_i(u_*)\leqslant 0,i=1,...,m,g_i(u_*)=0,i=m+1,...,s$. Следовательно, $u_*\in\mathbf{W}$. Теперь можно в \eqref{new-Tikhonov-condition} сделать предельный переход при $k=k_l\rightarrow\infty$:
\begin{equation}
\begin{aligned}
&T_k(z_k,z_k)\leqslant T_k(z_k,w)\quad\Leftrightarrow\quad\Phi(z_k,z_k)\leqslant\Phi(z_k,w)+A_kP(w)-A_kP(z_k)\Rightarrow\\
&\Rightarrow\{k\rightarrow\infty\}\Rightarrow\quad \Phi(u_*,u_*)\leqslant\Phi(u_*,w)+\lim\limits_{k\rightarrow\infty}A_k\underbrace{P(w)}_{=0,\forall w\in \mathbf{W}}-\underbrace{\lim\limits_{k\rightarrow\infty}A_kP(z_k)}_{=0,\eqref{[3]-13}}.
\end{aligned}
\end{equation}
Итак, $\Phi(u_*,u_*)\leqslant\Phi(u_*,w)\,\forall w\in\mathbf{W},$ т.е. $u_*\in\mathbf{W_*}$. Из оценки \eqref{[3]-12} при $k=k_l\rightarrow\infty$ следует $\|u_*\|\leqslant\|v_*\|$. Но $v_*$ - нормальное решение задачи \eqref{question}, т.е. $\|u_*\|\geqslant\|v_*\|$. Следовательно, $\|u_*\|=\|v_*\|$, то есть $u_*$ - нормальное решение этой задачи. Поскольку нормальное решение единственно, то $u_*=v_*$. Это означает, что $\{z_k\}$ имеет единственную предельную точку $v_*$, отсюда следует $\lim\limits_{k\rightarrow\infty}\|z_k-v_*\|=0$.

Наконец, докажем \eqref{[3]-15}. Так как функция $\Phi(v,w)$ выпукла и дифференцируема по переменной $w$ на $\mathbf{W_0}$, то в силу критерия выпуклости [см. \cite{5} стр. 160] имеем
\begin{equation*}
\Phi(v,v)-\Phi(v,w)\leqslant\langle\nabla_w\Phi(v,v),v-w\rangle\quad \Phi(w,w)-\Phi(w,v)\leqslant\langle\nabla_w\Phi(w,w),w-v\rangle
\end{equation*}
Сложим эти два неравенства и в силу \eqref{cososymmetric} получим
\begin{equation}
\label{[3]-24}
\begin{aligned}
0\leqslant& \,\Phi(v,v)-\Phi(v,w)-\Phi(w,v)+\Phi(w,w)\leqslant\\
\leqslant& \,\langle\nabla_w\Phi(v,v)-\nabla_w\Phi(w,w),v-w\rangle,\quad \forall v,w\in\mathbf{W_0}
\end{aligned}
\end{equation} 
Аналогично, из выпуклости и дифференцируемости $P(w)$ на $\mathbf{W_0}$ имеем
\begin{equation}
\label{[3]-25}
0\leqslant \langle\nabla_wP(v)-\nabla_wP(w),v-w\rangle
\end{equation}
Функция $T_k(v,w)$ - выпукла и дифференцируема по $w$ на $\mathbf{W_0}$, поэтому из \eqref{[3]-16}, взяв $v=z_k,w=z_m$, находим
\begin{equation}
\label{[3]-27}
2\alpha_k\|z_k-z_m\|^2\leqslant\langle\nabla_wT_k(z_k,z_k)-\nabla_wT_k(z_m,z_m),z_k-z_m\rangle.
\end{equation}
Кроме того, из \eqref{new-Tikhonov-condition} в силу критерия оптимальности [см. \cite{5} стр. 161] имеем $\forall w\in\mathbf{W_0}$
\begin{equation}
\label{[3]-28}
\begin{aligned}
&0\leqslant \langle\nabla_wT_k(z_k,z_k),w-z_k\rangle\\
&0\leqslant \langle\nabla_wT_m(z_m,z_m),w-z_m\rangle
\end{aligned}
\end{equation}
В первом из неравенств\eqref{[3]-28} положим $w=z_m$, во втором $w=z_k$ и почленно сложим с \eqref{[3]-27}. С учетом \eqref{[3]-5}  и неравенства Коши-Буняковского получим
\begin{equation*}
\begin{aligned}
&2\alpha_k\|z_k-z_m\|^2\leqslant\\
&\leqslant\langle\cancel{\nabla_wT_k(z_k,z_k)}-\nabla_wT_k(z_m,z_m)\cancel{-\nabla_wT_k(z_k,z_k)}+\nabla_wT_m(z_m,z_m),z_k-z_m\rangle\stackrel{\eqref{[3]-5}}{\leqslant}\\
&\leqslant\langle\cancel{\nabla_w\Phi(z_m,z_m)}+A_m\nabla_wP(z_m)\cancel{-\nabla_w\Phi(z_k,z_k)}-A_k\nabla_wP(z_m),z_k-z_m\rangle\stackrel{\text{нер-во К-Б-}}{\leqslant}\\
&\leqslant|A_m-A_k|R_1\cdot \|z_k-z_m\|
\end{aligned}
\end{equation*}
Отсюда следует неравенство \eqref{[3]-15}. Теорема \ref{thero-2-2} доказана. $\qedsymbol$
\subsection{Описание регуляризованного экстраградиентного метода с использованием нового стабилизатора}
\noindent Приступим к исследованию сходимости итерационного метода \eqref{[3]-7}. Справедлива
\begin{theo}
	\label{thero-2-3}
	Пусть выполнены все условия теорем \ref{thero-2-1},\ref{thero-2-2}, и \\
	\textbf{1)} пусть градиенты $\nabla_w\Phi(v,w),\nabla_wP(w)$ удовлетворяют условию Липшища:
	\begin{equation}
	\label{[3]-29}
	\begin{aligned}
	&\max\{ \|\nabla_w\Phi(v,v)-\nabla_w\Phi(w,w)\|,\|\nabla_wP(v)-\nabla_wP(w)\|\}\leqslant L\|v-w\|,\\
	&\forall v,w\in\mathbf{W_0},L=\mathrm{const}>0.
	\end{aligned}
	\end{equation}
	Пусть также выполнено модифицированное условие Липшица вида:
	\begin{equation}
	\label{[3]-29-2}
	\|\nabla_w\Phi(v,w)-\nabla_w\Phi(w,w)\|\leqslant L\|v-w\|,\quad \forall v,w\in\mathbf{W_0},L=\mathrm{const}>0.
	\end{equation}
	\textbf{2)} Вместо точного значения градиентов $\nabla_w\Phi(v,w),\nabla_wP(w)$ известны последовательности их приближений $\{\nabla_w^k\Phi(v,w)\},\{\nabla_w^kP(w) \}$, удовлетворяющие условиям \eqref{[3]-6}.\\
	\textbf{3)} Параметры $\{\alpha_k\},\{\beta_k\},\{\delta_k\},\{A_k\}$ метода \eqref{[3]-7} таковы, что \\
	\begin{equation}
	\label{[3]-30}
	\begin{aligned}
	&\alpha_k >0,A_{k+1}\geqslant A_k>0,\beta_k>0,\delta_k>0,\lim_{k\to \infty}\beta_k=0,\lim_{k\to\infty}\delta_k=0,\\
	&\sup_{k\geqslant 0}\beta_k(1+A_k)<\frac{1}{L},\lim_{k\to\infty}\frac{\delta_k+\delta_kA_k}{\alpha_k}=0,\lim_{k\to \infty}\frac{A_{k+1}-A_k}{\alpha_k^2\beta_k}=0.
	\end{aligned}
	\end{equation}
	Тогда последовательность $\{v_k\}$, порожденная методом \eqref{[3]-7} при любом выборе начального приближения $v_0\in \mathbf{W_0}$, сходится к нормальному решению $v_*$ задачи \eqref{question}, т.е.
	\begin{equation}
	\label{convergence}
	\lim_{k\to \infty}\|v_k-v_*\|=0
	\end{equation}
	причем сходимость в \eqref{convergence} равномерна относительно выбора $\{\nabla_w^k\Phi(v,w)\}$,$\{\nabla_w^kP(w) \}$ из \eqref{[3]-6}.
\end{theo}
\begin{remark}
\label{remark 3-2}
В качестве последовательностей параметров $\{\alpha_k\}$,$\{\beta_k\}$,$\{\delta_k\}$,$\{A_k\}$, выполняющих условиям \eqref{[3]-30}, можно, например, взять
\begin{equation*}
	\alpha_k = (k+1)^{-\alpha},A_k=(k+1)^A,\delta_k=(k+1)^{-\delta},\beta_k=\frac{1}{2L(1+A_k)}
\end{equation*}
где $\alpha,A,\delta$ - положительные числа. % и $\alpha+A<\min\{\frac{1}{2},\delta \},\alpha<A^{v/(p-v)}$ (при $p=v$ последнее неравенство не нужно).
\end{remark}
\noindent\emph{Доказательство}. Справедливо неравенство
\begin{equation}
\label{[3]-33}
\|v_k-v_*\|\leqslant \|v_k-z_k\|+\|z_k-v_*\|,\quad k=0,1,...
\end{equation}
где $v_*$ - нормальное решение задачи \eqref{question}, $z_k$ - решение задачи \eqref{new-Tikhonov-condition}. Из \eqref{[3]-14},\eqref{[3]-33} следует, что для доказательства теоремы \ref{thero-2-3} достаточно установить, что величины $b_k=\|v_k-z_k\|\to 0$ при $k\to \infty$.\\
Покажем, что величины $b_k$ удовлетворяют неравенствам
\begin{equation}
\label{[3]-34}
\begin{aligned}
&b_{k+1}^2\leqslant (1-s_k)b_k^2+d_k,\\
&k\geqslant k_0,0<s_k\leqslant 1,d_k\geqslant 0,\forall k\geqslant k_0,\sum_{k=0}^{\infty}s_k=+\infty,\lim_{k\to \infty}\frac{d_k}{s_k}=0,
\end{aligned}
\end{equation}
где $k_0$ - достаточно большое натуральное число. Заметим, что 
\begin{equation}
\label{[3]-36}
b_{k+1}=\|v_{k+1}-z_{k+1}\|\leqslant \|v_{k+1}-z_k\|+\|z_k-z_{k+1}\|,\quad k=0,1,...
\end{equation}
Для оценки величины $\|z_k-z_{k+1}\|$ имеем неравенство \eqref{[3]-15} при $m=k+1$. Рассмотрим первое слагаемое из \eqref{[3]-36}. Как известно, (см. \cite{5} стр. 183), $\mathrm{Pr}_{\mathbf{W_0}}(x)=a$ тогда и только тогда, когда 
\begin{equation*}
\langle a-x,w-a \rangle\geqslant 0,\quad \forall w\in\mathbf{W_0}.
\end{equation*}
Пользуясь этим свойством проекции, перепишем метод \eqref{[3]-7} в эквивалентной форме:
\begin{equation}
\label{[3]-37}
\big\langle u_k-v_k+\beta_k(\nabla_w^k\Phi(v_k,v_k)+A_k\nabla_w^kP(v_k)),w-u_k\big\rangle\geqslant 0,\quad \forall w\in\mathbf{W_0}.
\end{equation}
\begin{equation}
\label{[3]-38}
\big\langle v_{k+1}-v_k+\beta_k(\nabla_w^k\Phi(v_k,u_k)+A_k\nabla_w^kP(u_k)-2\alpha_k(u_k-v_k)),w-v_{k+1}\big\rangle\geqslant 0, \forall w\in\mathbf{W_0}.
\end{equation}
Подставим $w=v_{k+1}$ в \eqref{[3]-37} и $w=z_k$ в \eqref{[3]-38}, получившиеся неравенства сложим:
\begin{equation}
\label{[3]-39}
\begin{aligned}
&\langle u_k-v_k,v_{k+1}-u_k\rangle +\langle v_{k+1}-v_k,z_k-v_{k+1} \rangle +\\
&+\beta_k\langle \nabla_w^k\Phi(v_k,v_k)\textcolor{blue}{-\nabla_w^k\Phi(u_k,u_k)}+A_k(\nabla_w^kP(v_k)\textcolor{blue}{-\nabla_w^kP(u_k)}),v_{k+1}-u_k\rangle +\\
&+\beta_k\langle \textcolor{blue}{\nabla_w^k\Phi(u_k,u_k)}+A_k\textcolor{blue}{\nabla_w^kP(u_k)},z_k-u_k\rangle+\\
&+\beta_k\langle \nabla_w^k\Phi(v_k,u_k)-\nabla_w^k\Phi(u_k,u_k),z_k-v_{k+1} \rangle-\\
&-2\alpha_k\beta_k\langle u_k-v_k,z_k-v_{k+1}\rangle\geqslant 0.
\end{aligned}
\end{equation}
Справедливо равенство
\begin{equation*}
2\big(\langle u_k-v_k,v_{k+1}-u_k \rangle+\langle v_{k+1}-v_{k},z_k-v_{k+1}\rangle \big)=\|v_k-z_k\|^2-\|v_k-u_k\|^2-\|v_{k+1}-z_k\|^2-\|v_{k+1}-u_k\|^2.
\end{equation*}
Отсюда из \eqref{[3]-39} получаем
\begin{equation}
\label{[3]-40}
\begin{aligned}
&\|v_{k+1}-z_k\|^2\leqslant \|v_k-z_k\|^2-\|v_k-u_k\|^2-\|v_{k+1}-u_k\|^2+\\
&+2\beta_k\big\langle [\nabla_w^k\Phi(v_k,v_k)-\nabla_w\Phi(v_k,v_k)]+A_k[\nabla_w^kP(v_k)-\nabla_wP(v_k)],v_{k+1}-u_k\big\rangle+\\
&+2\beta_k\big\langle [\nabla_w\Phi(v_k,v_k)-\nabla_w\Phi(u_k,u_k)]+A_k[\nabla_wP(v_k)-\nabla_wP(u_k)],v_{k+1}-u_k\big\rangle+\\
&+2\beta_k\big\langle [\nabla_w\Phi(u_k,u_k)-\nabla_w^k\Phi(u_k,u_k)]+A_k[\nabla_wP(u_k)-\nabla_w^kP(u_k)],v_{k+1}-u_k\big\rangle+\\
&+2\beta_k\big\langle [\nabla_w^k\Phi(u_k,u_k)-\nabla_w\Phi(u_k,u_k)]+A_k[\nabla_w^kP(u_k)-\nabla_wP(u_k)],z_k-u_k\big\rangle+\\
&+2\beta_k\big\langle [\nabla_w\Phi(u_k,u_k)-\nabla_w\Phi(z_k,z_k)]+A_k[\nabla_wP(u_k)-\nabla_wP(z_k)],z_k-u_k\big\rangle+\\
&+2\beta_k\big\langle \nabla_w\Phi(z_k,z_k)+A_k\nabla_wP(z_k),z_k-u_k\big\rangle+\\
&+2\beta_k\langle \nabla_w^k\Phi(v_k,u_k)-\nabla_w^k\Phi(u_k,u_k),z_k-v_{k+1} \rangle+4\alpha_k\beta_k\langle u_k-v_k,v_{k+1}-z_k\rangle.
\end{aligned}
\end{equation}
Далее оценим слагаемые из них. Сначала заметим, что справедливы следующие соотношения:
\begin{equation}
\label{2|ab|}
2|ab|\leqslant \varepsilon a^2+b^2/\varepsilon,\forall\varepsilon>0.
\end{equation}
\begin{equation}
\label{An<=Qn}
\frac{(x_1+...+x_n)^2}{n}\leqslant x_1^2+\ldots+x_n^2.
\end{equation}
Для четвертого слагаемого с учетом \eqref{[3]-6} и \eqref{[3]-12} получаем
\begin{equation}
\label{[3]-41}
\begin{aligned}
&2\beta_k\big\langle [\nabla_w^k\Phi(v_k,v_k)-\nabla_w\Phi(v_k,v_k)]+A_k[\nabla_w^kP(v_k)-\nabla_wP(v_k)],v_{k+1}-u_k\big\rangle\leqslant \\
&\leqslant 2\beta_k\big\langle \delta_k(1+2\|v_k\|)+A_k\delta_k\|v_k\|,v_{k+1}-u_k\big\rangle\leqslant \\
&\leqslant 2\beta_k\big\langle (\delta_k+A_k\delta_k)(1+2\|v_k\|),v_{k+1}-u_k\big\rangle\leqslant\\
&\leqslant 2\beta_k(\delta_k+\delta_kA_k)(1+2\|v_k\|)\|v_{k+1}-u_k\|\leqslant\\
&\leqslant\beta_k(\delta_k+\delta_kA_k)2\cdot(1+2\|v_k-z_k\|+\|z_k\|)\|v_{k+1}-u_k\|\leqslant\\
&\leqslant\beta_k(\delta_k+\delta_kA_k)(\frac{1}{2}(1+2R+2\|v_k-z_k\|^2)+2\|v_{k+1}-u_k\|)^2\leqslant\\
&\leqslant\beta_k(\delta_k+\delta_kA_k)((1+2R)^2+4\|v_k-z_k\|^2+2\|v_{k+1}-u_k\|^2).
\end{aligned}
\end{equation}
Аналогично оцениваются и шестое слагаемое
\begin{equation}
\label{[3]-42}
\begin{aligned}
&2\beta_k\big\langle [\nabla_w\Phi(u_k,u_k)-\nabla_w^k\Phi(u_k,u_k)]+A_k[\nabla_wP(u_k)-\nabla_w^kP(u_k)],v_{k+1}-u_k\big\rangle\leqslant\\
&\leqslant 2\beta_k(\delta_k+\delta_kA_k)(1+2\|u_k\|)\|v_{k+1}-u_k\|\leqslant\\
&\leqslant \beta_k (\delta_k+\delta_kA_k)2\cdot(1+2\|u_k-v_k\|+2\|v_k-z_k\|+2\|z_k\|)\|v_{k+1}-u_k\|\leqslant\\
&\leqslant\beta_k (\delta_k+\delta_kA_k)(\frac{1}{3}(1+2R+2\|u_k-v_k\|+2\|v_k-z_k\|)^2+3\|v_{k+1}-u_k\|^2)\leqslant\\
&\leqslant\beta_k (\delta_k+\delta_kA_k)((1+2R)^2+4\|u_k-v_k\|^2+4\|v_k-z_k\|^2+3\|v_{k+1}-u_k\|^2),
\end{aligned}
\end{equation}
и седьмое слагаемое
\begin{equation}
\label{[3]-43}
\begin{aligned}
&2\beta_k\big\langle [\nabla_w\Phi(u_k,u_k)-\nabla_w^k\Phi(u_k,u_k)]+A_k[\nabla_wP(u_k)-\nabla_w^kP(u_k)],z_k-u_k\big\rangle\leqslant\\
&\leqslant 2\beta_k(\delta_k+\delta_kA_k)(1+2\|u_k\|)\|z_k-u_k\|\leqslant\\
&\leqslant \beta_k (\delta_k+\delta_kA_k)2\cdot(1+2\|u_k-v_k\|+2\|v_k-z_k\|+2\|z_k\|)(\|z_k-v_k\|+\|v_k-u_k\|)\leqslant\\
&\leqslant\beta_k (\delta_k+\delta_kA_k)(\frac{1}{3}(1+2R+2\|u_k-v_k\|+2\|v_k-z_k\|)^2+3(\|z_k-v_k\|+\|v_k-u_k\|)^2)\leqslant\\
&\leqslant\beta_k (\delta_k+\delta_kA_k)((1+2R)^2+10\|u_k-v_k\|^2+10\|v_k-z_k\|^2).
\end{aligned}
\end{equation}
Для оценки пятого слагаемого воспользуемся условием Липшица \eqref{[3]-29}:
\begin{equation}
\label{[3]-44}
\begin{aligned}
&2\beta_k\big\langle [\nabla_w\Phi(v_k,v_k)-\nabla_w\Phi(u_k,u_k)]+A_k[\nabla_wP(v_k)-\nabla_wP(u_k)],v_{k+1}-u_k\big\rangle\leqslant\\
&2\beta_k(1+A_k)L\|v_k-u_k\|\|v_{k+1}-u_k\|\leqslant\beta_k(1+A_k)L(\|v_k-u_k\|^2+\|v_{k+1}-u_k\|^2).
\end{aligned}
\end{equation}
Для восьмого слагаемого имеем оценку из \eqref{[3]-24},\eqref{[3]-25} при $v=u_k,w=z_k$:
\begin{equation}
\label{[3]-45}
2\beta_k\big\langle [\nabla_w\Phi(u_k,u_k)-\nabla_w\Phi(z_k,z_k)]+A_k[\nabla_wP(u_k)-\nabla_wP(z_k)],z_k-u_k\big\rangle\leqslant 0.
\end{equation}
Для оценки девятого слагаемого воспользуемся условием оптимальности \eqref{[3]-28} при $w=u_k$:
\begin{equation}
\label{[3]-46}
2\beta_k\big\langle \nabla_w\Phi(z_k,z_k)+A_k\nabla_wP(z_k),z_k-u_k\big\rangle\leqslant 0.
\end{equation}
Последнее слагаемое оценим так: используем \eqref{2|ab|},\eqref{An<=Qn}, получим
\begin{equation*}
\begin{aligned}
&4\alpha_k\beta_k\langle u_k-v_k,v_{k+1}-z_k\rangle = 4\alpha_k\beta_k \langle u_k-z_k+z_k-v_k,v_{k+1}-u_k+u_k-v_k+v_k-z_k\rangle\leqslant\\
&\leqslant 4\alpha_k\beta_k \Big[ \|u_k-z_k\|\left( \|v_{k+1}-u_k\|+\|u_k-v_k\|+\|v_k-z_k\|\right)+\\
&\quad +\langle z_k-v_k,v_{k+1}-u_k\rangle+\langle z_k-v_k, u_k-v_k\rangle-\|v_k-z_k\|^2 \Big] \leqslant\\
&\leqslant 4\alpha_k\beta_k\Big[ 3\varepsilon \|u_k-z_k\|^2+\varepsilon\left(\|v_{k+1}-u_k\|^2+\|u_k-v_k\|^2+\|v_k-z_k\|^2 \right)-\|v_k-z_k\|^2+\\
&\quad+\delta\|z_k-v_k\|^2+\frac{1}{\delta}\|v_{k+1}-u_k\|^2+\Delta\|z_k-u_k\|^2+\frac{1}{\Delta}\|u_k-v_k\|^2\Big]\leqslant\\
&\leqslant 4\alpha_k\beta_k\Big[6\varepsilon \left(\|u_k-v_k\|^2+\|z_k-v_k\|^2\right)+\varepsilon\left(\|v_{k+1}-u_k\|^2+\|u_k-v_k\|^2+\|v_k-z_k\|^2\right)-\\
&\quad-\|v_k-z_k\|^2+\delta\|z_k-v_k\|^2+\frac{1}{\delta}\|v_{k+1}-u_k\|^2+\Delta\|z_k-u_k\|^2+\frac{1}{\Delta}\|u_k-v_k\|^2\Big]\leqslant\\
&\leqslant 4\alpha_k\beta_k\left[ (7\varepsilon +\delta+\Delta-1)\|v_k-z_k\|^2+(7\varepsilon+\frac{1}{\Delta})\|u_k-v_k\|^2+(\varepsilon+\frac{1}{\delta})\|v_{k+1}-u_k\|^2\right],
\end{aligned}
\end{equation*}
где $\varepsilon,\delta,\Delta$ - какие-то положительные числа. Предположим, что
\begin{equation*}
7\varepsilon +\delta+\Delta-1 <0.
\end{equation*}
В качестве значений этих величин можем, например, взять $\varepsilon = \frac{1}{14},\delta = \frac{1}{5},\Delta = \frac{1}{10}$, выполняющих данную систему неравенств. При взятии этих значений имеем такую оценку:
\begin{equation}
\label{[3]-47}
\begin{aligned}
&4\alpha_k\beta_k\langle u_k-v_k,v_{k+1}-z_k\rangle \leqslant\\
\leqslant &-\frac{4}{5}\alpha_k\beta_k\|v_k-z_k\|^2+42\alpha_k\beta_k\|u_k-v_k\|^2+\frac{142}{7}\alpha_k\beta_k\|v_{k+1}-u_k\|^2.
\end{aligned}
\end{equation}
Наконец, десятое слагаемое разбивается на три части, в которых каждая из них оцениваются аналогичным образом:
\begin{equation}
\label{[3]-47-2}
\begin{aligned}
&2\beta_k\langle \nabla_w^k\Phi(v_k,u_k)-\nabla_w^k\Phi(u_k,u_k),z_k-v_{k+1} \rangle=\\
&=2\beta_k\langle \nabla_w^k\Phi(v_k,u_k)-\nabla_w\Phi(v_k,u_k),z_k\textcolor{blue}{-v_k+v_k-u_k+u_k}-v_{k+1} \rangle+\\
&+2\beta_k\langle \nabla_w\Phi(v_k,u_k)-\nabla_w\Phi(u_k,u_k),z_k\textcolor{blue}{-v_k+v_k-u_k+u_k}-v_{k+1}  \rangle+\\
&+2\beta_k\langle \nabla_w\Phi(u_k,u_k)-\nabla_w^k\Phi(u_k,u_k),z_k\textcolor{blue}{-v_k+v_k-u_k+u_k}-v_{k+1}  \rangle\leqslant\\
&\leqslant \beta_k\delta_k\big((1+2R)^2+13\|v_k-z_k\|^2+10\|u_k-v_k\|^2+9\|u_k-v_{k+1}\|^2 \big)+\\
&+\beta_kL(3\|v_k-z_k\|^2+4\|u_k-v_k\|^2+3\|u_k-v_{k+1}\|^2)+\\
&+\beta_k\delta_k\big((1+2R)^2+13\|v_k-z_k\|^2+13\|u_k-v_k\|^2+9\|u_k-v_{k+1}\|^2 \big).
\end{aligned}
\end{equation}
Подставив оценки \eqref{[3]-41}-\eqref{[3]-47-2} в \eqref{[3]-40}, имеем
\begin{equation}
\label{[3]-48}
\begin{aligned}
&\|v_{k+1}-z_k\|^2\leqslant \|v_k-z_k\|^2\left[1-\frac{4}{5}\alpha_k\beta_k+18\beta_k(\delta_k+\delta_kA_k)+\beta_k(26\delta_k+3L)\right]+\\
&+\|v_k-u_k\|^2\left[-1+\beta_k(1+A_k)L+14\beta_k(\delta_k+\delta_kA_k)+\beta_k(42\alpha_k+23\delta_k+4L)\right]+\\
&+\|v_{k+1}-u_k\|^2\left[-1+\beta_k(1+A_k)L+5\beta_k(\delta_k+\delta_kA_k)+\beta_k(\frac{142}{7}\alpha_k+18\delta_k+3L)\right]+\\
&+(3\beta_k(\delta_k+\delta_kA_k)+2\beta_k\delta_k)(1+2R)^2.
\end{aligned}
\end{equation}
По условию теоремы, имеем
\begin{equation*}
\sup_{k\geqslant 0}\beta_k(1+A_k)<1/L,\quad \lim_{k\rightarrow\infty}(\delta_k+\delta_k A_k)=0,\quad \lim_{k\to \infty} \alpha_k=0.
\end{equation*}
Следовательно, коэффициенты при $\|v_k-u_k\|^2,\|v_{k+1}-u_k\|^2$ в \eqref{[3]-48} неположительны $\forall k\geqslant k_0$, где $k_0$ - достаточно большое натуральное число. Поэтому
\begin{equation}
\label{[3]-49}
\begin{aligned}
\|v_{k+1}-z_k\|^2&\leqslant \|v_k-z_k\|^2\left[1-\frac{4}{5}\alpha_k\beta_k+18\beta_k(\delta_k+\delta_kA_k)+\beta_k(26\delta_k+3L)\right]+\\
&+(3\beta_k(\delta_k+\delta_kA_k)+2\beta_k\delta_k)(1+2R)^2,\quad \forall k\geqslant k_0.
\end{aligned}
\end{equation}
Подставим в \eqref{[3]-36} оценки \eqref{[3]-49} и \eqref{[3]-15} при $m=k+1$. В силу монотонности $\{A_k\}$ имеем
\begin{equation*}
\begin{aligned}
0\leqslant b_{k+1}&\leqslant\big\{b_k^2\left[1-\frac{4}{5}\alpha_k\beta_k+18\beta_k(\delta_k+\delta_kA_k)+\beta_k(26\delta_k+3L)\right]+\\
&+(3\beta_k(\delta_k+\delta_kA_k)+2\beta_k\delta_k)(1+2R)^2\big\}^{1/2}+\frac{(A_{k+1}-A_k)R_1}{2\alpha_k},\quad \forall k\geqslant k_0.
\end{aligned}
\end{equation*}
Справедливо следующее:
\begin{equation*}
(a+b)^2\leqslant(1+\varepsilon)(a^2+b^2/\varepsilon),\quad\forall\varepsilon>0.
\end{equation*}
При взятии
\begin{equation*}
\begin{aligned}
a=\Big\{b_k^2\Big[1-\frac{4}{5}\alpha_k\beta_k&+18\beta_k(\delta_k+\delta_kA_k)+\beta_k(26\delta_k+3L)\Big]+(3\beta_k(\delta_k+\delta_kA_k)+2\beta_k\delta_k)(1+2R)^2\Big\}^{1/2},\\
&b=\frac{(A_{k+1}-A_k)R_1}{2\alpha_k},\quad \varepsilon=\frac{1}{2}\alpha_k\beta_k
\end{aligned}
\end{equation*}
имеем
\begin{equation*}
\begin{aligned}
&0\leqslant b_{k+1}^2\leqslant b_k^2\left[1-\frac{4}{5}\alpha_k\beta_k+18\beta_k(\delta_k+\delta_kA_k)+\beta_k(26\delta_k+3L)\right](1+\frac{1}{2}\alpha_k\beta_k)+\\
&+(3\beta_k(\delta_k+\delta_kA_k)+2\beta_k\delta_k)(1+2R)^2\times(1+\frac{1}{2}\alpha_k\beta_k)+\left(\frac{(A_{k+1}-A_k)R_1}{2\alpha_k}\right)^2\frac{2}{\alpha_k\beta_k}(1+\frac{1}{2}\alpha_k\beta_k)=\\
&=b_k^2\left[1-\frac{3}{10}\alpha_k\beta_k-\frac{2}{5}(\alpha_k\beta_k)^2+(18\beta_k(\delta_k+\delta_kA_k)+\beta_k(26\delta_k+3L))(1+\frac{1}{2}\alpha_k\beta_k)\right]+\\
&+(3\beta_k(\delta_k+\delta_kA_k)+2\beta_k\delta_k)(1+2R)^2\times(1+\frac{1}{2}\alpha_k\beta_k)+\left(\frac{(A_{k+1}-A_k)R_1}{2\alpha_k^2\beta_k}\right)^2 2\alpha_k\beta_k(1+\frac{1}{2}\alpha_k\beta_k).
\end{aligned}
\end{equation*}
Как мы видим, последовательность $\{b_k^2\}$ удовлетворяет неравенствам \eqref{[3]-34} при
\begin{equation}
\label{[3]-50}
\begin{aligned}
s_k=\frac{3}{10}\alpha_k\beta_k\Big[1+\frac{4}{3}\alpha_k\beta_k-(60\frac{\delta_k+\delta_kA_k}{\alpha_k}&+\frac{10}{3}\cdot\frac{26\delta_k+3L}{\alpha_k})(1+\frac{1}{2}\alpha_k\beta_k)\Big]\\
d_k=\alpha_k\beta_k\Big[(3\frac{(\delta_k+\delta_kA_k)}{\alpha_k}+2\frac{\delta_k}{\alpha_k})(1+2R)^2&\times(1+\frac{1}{2}\alpha_k\beta_k)+\\
&+\left(\frac{(A_{k+1}-A_k)R_1}{2\alpha_k^2\beta_k}\right)^2 (2+\alpha_k\beta_k)\Big].
\end{aligned}
\end{equation}
С учетом \eqref{[3]-30}, взяв при необходимости номер $k_0$ ещё большим, можем считать, что 
\begin{equation*}
0<s_k\leqslant 1,\quad \forall k\geqslant k_0.
\end{equation*} 
Кроме того, 
\begin{equation*}
0\leqslant A_{k+1}-A_k=\left(\frac{A_{k+1}-A_k}{\alpha_k^2\beta_k}\alpha_k \right)(\alpha_k\beta_k)\leqslant \alpha_k\beta_k,\quad \forall k\geqslant k_0.
\end{equation*}
Суммируя эти неравенства по $k$ от $k_0$ до некоторого $N$, получаем 
\begin{equation*}
\sum_{k=k_0}^{N}\alpha_k\beta_k\geqslant A_{N+1}-A_{k_0}\to \infty, \text{ при } N\to \infty,
\end{equation*}
т.е. ряд $\sum\limits_{k=0}^{\infty}\alpha_k\beta_k$ расходится. Как следует из \eqref{[3]-50}, $\lim\limits_{k\to \infty} \frac{s_k}{\alpha_k\beta_k}=\text{const}$, тогда из признака сравнения вытекает $\sum\limits_{k=0}^{\infty}s_k=+\infty$. Наконец, из условий \eqref{[3]-30} и выражений $\eqref{[3]-50}$ для $s_k,d_k$ вытекает, что 
\begin{equation*}
\lim\limits_{k\to\infty}\frac{d_k}{s_k}=\underbrace{\lim\limits_{k\to\infty}\frac{d_k}{\alpha_k\beta_k}}_{\to 0}\cdot\underbrace{\lim\limits_{k\to\infty}\frac{\alpha_k\beta_k}{s_k}}_{\to \text{const}}=0.
\end{equation*}
Видно, что все условия в \eqref{[3]-34} выполнены.\\
Всякая последовательность $\{b_k^2\}$, удовлетворяющая неравенствам и условиям \eqref{[3]-34}, сходится к нулю. (см. \cite{5} c.90, лемма 6). Тем самым доказано, что $\lim\limits_{k\to \infty}\|v_k-z_k\|=0$. Отсюда и из \eqref{[3]-14},\eqref{[3]-33} следует равенство \eqref{convergence}.\\
Так как коэффициенты $s_k,d_k$ в \eqref{[3]-34}, как можно увидеть из формул \eqref{[3]-50}, не зависят от выбора конкретной реализации приближений $\nabla_w^k\Phi(v,w),\nabla_w^kP(w)$, лишь бы выполнялось условие \eqref{[3]-6}, то сходимость в \eqref{convergence} равномерна относительно выбора $\nabla_w^k\Phi(v,w),\nabla_w^kP(w)$ из \eqref{[3]-6}. Теорема доказана. $\qedsymbol$

Итак, полностью описан регуляризованный экстраградиентный метод и доказана сходимость метода.
\clearpage
%\newpage