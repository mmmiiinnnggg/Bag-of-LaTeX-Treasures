\phantomsection
\addcontentsline{toc}{section}{Введение}
\section*{Введение}
	\textbf{Актуальность темы.} Многие важные проблемы исследования операций, вычислительной математики, математической экономики, связанных с поиском компромисса и согласования частично (или полностью) противоположных интересов сторон конфликта, сводятся к исследованию математических моделей, которое составляет область математики так называемую \emph{равновесным программированием}. В форме равновесной задачи можно записать задачи оптимального управления, многокритериального принятия решений в условиях неопределенности, обратные задачи оптимизации, вариационные неравенства, седловые задачи Лагранжа, задачи поиска равновесия по Нэшу и т.д. \cite{8}. 
	
	\textbf{Литературный обзор.} К настоящему времени достаточно хорошо изучена проблема существования и единственности равновесной точки невозмущенной задачи \cite{6}. Также в работах \cite{4}\cite{6} изложен результат о выпуклости и замкнутости множества решений. 
	Обсуждены, например, в работах \cite{7}-\cite{10-2} различные конкретные методы нахождения точек равновесия. Такие методы обычно имеют итерационный характер. Однако они исследовались при значительных ограничениях на $\Phi(v,w)$ и множество $\mathbf{W}$(см. постановку задачи), что редко бывает на практике не легко применяются. Особое внимание уделяется непрерывным методам, которые порождают целые семейства методов \cite{10}. На практике также обращается внимание на другую особенность поставленной задачи, а именно неустойчивость. С учетом возмущений исходных данных исследовались варианты подобных методов с использования стабилизатора для регуляризации. В различных работах предложены методы регуляризации неустойчивых равновесных задач \cite{11}-\cite{16}, в которых обычно используется стабилизатор вида скалярного произведения $\alpha_k\langle v,w\rangle$. Соответственно, можно попробовать и другие виды стабилизатора, например, $-\alpha_k\|v-w\|^2$. Следует отметить, что данная тема актуальна, и требуется новый подход для решения поставленной неустойчивой задачи.
	
	\textbf{Цель работы.} Разработка метода стабилизации с применением нового стабилизатора вида квадрата нормы для решения задач равновесного программирования с неточными данными, а также разработка регуляризованного экстраградиентного метода с использованием нового стабилизатора вида квадрата нормы. Проведение численных расчётов для проверки и практического использования разобранных методов. 
	
	\textbf{Методы исследования.} В работе используются теория и методы оптимизации, теория равновесного программирования, теория и методы решения некорректных поставленных задач, численные методы решения различных практических задач.
	
	\textbf{Практическая значимость.} Методы, разработанные в данной работе, могут быть применены к решению задач равновесного программирования, функция и множество которых неустойчивы к погрешностям задания исходных данных.
	
	\textbf{Объём и структура работы.} Данная работа состоит из введения, постановки задачи, четырёх глав, заключения, списка литературы и приложения. Полный объём работы составляет $44$ страницы с $1$ листингом алгоритма. Список литературы содержит $19$ наименований. В первой главе изложены основные факты теории равновесного программирования и теории методов решения некорректных поставленных задач. В последующих главах изложен новый материал: в главе 2 описан метод стабилизации для решения задач равновесного программирования с неточно заданным множеством путем применения нового стабилизатора вида квадрата нормы; в главе 3 описан регуляризованный экстраградиентый метод для решения задач равновесного программирования с неточно заданным множеством путем применения нового стабилизатора вида квадрата нормы; в главе 4 изложены результаты вычислительных экспериментов, основанных на разработанных в предыдущих главах методах; в заключении перечислены основные полученные результаты работы; в конце работы даны список литературы и приложения к работе. 
	
	\textbf{Благодарности.} Автор выражает благодарность своему научному руководителю Б.А. Будаку за ценные замечания, высказанные при обсуждении данной работы.
	

\clearpage
%\newpage