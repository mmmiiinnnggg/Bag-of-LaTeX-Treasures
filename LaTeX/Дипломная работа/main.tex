%\documentclass[oneside, final, 12pt]{extreport}
\documentclass[12pt]{article}

\usepackage{amsmath, amsthm, amssymb} %math expressions, theoreme/lemma, math symbols
\usepackage{mathtext} %russian letters in formulas

% fonts and lang
%\usepackage[T1,TS1,T2A]{fontenc}
\usepackage{cmap}
\usepackage[T1, T2A]{fontenc}
\usepackage[utf8]{inputenc}
\usepackage[english, russian]{babel}
% formatting
% \usepackage{geometry} %customize page layout
% \geometry{left = 3cm}
% \geometry{right = 1cm}
% \geometry{top = 1.5cm}
% \geometry{bottom = 2cm}
\usepackage{cancel}
\usepackage{color}
\usepackage{setspace} %set space between lines
\usepackage{indentfirst} %indent first pagearagraph after section header
\usepackage{tocloft} %control table of contents, figures
\usepackage{graphicx}
\usepackage{hhline}
\usepackage{caption}
\usepackage{subcaption}
\usepackage[shortlabels]{enumitem}
\setlist{nolistsep, itemsep=0.1cm, parsep=0pt, leftmargin=1.5cm}
\usepackage{multirow}
\usepackage{booktabs}
\usepackage{algorithm}
\usepackage{algpseudocode}
%\usepackage[pdfborder=000]{hyperref}
\usepackage{hyperref}
\hypersetup{
	colorlinks=true,
	linkcolor=black
}
\usepackage{listings}
\usepackage[usenames,dvipsnames]{xcolor}
\definecolor{mygreen}{rgb}{0,0.6,0}
\definecolor{mygray}{rgb}{0.5,0.5,0.5}
\definecolor{mymauve}{rgb}{0.58,0,0.82}
\lstset{
	backgroundcolor=\color{white}, 
	basicstyle = \footnotesize,       
	breakatwhitespace = false,        
	breaklines = true,                 
	captionpos = b,                    
	commentstyle = \color{mygreen}\bfseries,
	extendedchars = false,             
	frame =shadowbox, 
	framerule=0.5pt,
	keepspaces=true,
	keywordstyle=\color{blue}\bfseries, % keyword style
	language = C++,                     % the language of code
	otherkeywords={string}, 
	numbers=left, 
	numbersep=5pt,
	numberstyle=\tiny\color{mygray},
	rulecolor=\color{black},         
	showspaces=false,  
	showstringspaces=false, 
	showtabs=false,    
	stepnumber=1,         
	stringstyle=\color{mymauve},        % string literal style
	tabsize=2,          
	title=\lstname                      
}

%\numberwithin{equation}{section}
\usepackage{chngcntr}
\counterwithin{equation}{section}  
\allowdisplaybreaks[4]

\textheight=23cm
\textwidth=16cm
\oddsidemargin=5mm % левое поле для нечётных страниц
\evensidemargin=-5mm % севое поле для чётных страниц
%\marginparwidth=36pt
\topmargin=-1cm % расстояние от верхней границы листа до заголовка
%\flushbottom % выстота тела всех страниц одинаковая 
\raggedbottom % позволяет несколько различаться высоте тел различных страниц
\tolerance 3000 % how much badness is allowable without error. 

\linespread{1.3} %1.5 spacing 
\usepackage{setspace} % local set space

\parindent 1.27cm % Абзацный отступ

\sloppy             % текст редко залезает на правое поле
\clubpenalty = 10000  % Запрещаем разрыв страницы после первой строки абзаца
\widowpenalty = 10000 % Запрещаем разрыв страницы после предпоследней строки абзаца
% \global\hyphenpenalty = 1000 % Частота переносов


%Разработка метода прогнозирования слабой масштабируемости суперкомпьютерных приложений

\begin{document}
	\newtheorem{prop}{Предложение}[section]
	\newtheorem{theo}{Теорема}[section]
	\newtheorem{defin}{Определение}[section]
	\newtheorem{remark}{Замечание}[section]
	\newtheorem{example}{Пример}[section]

	
\begin{titlepage}
	\begin{spacing}{1.0}
	\begin{center}
		\textbf{Московский государственный университет имени М.В.\,Ломоносова}\\
		\textbf{\small Университет МГУ-ППИ в Шэньчжэне}\\
		\textbf{\small Факультет вычислительной математики и кибернетики}
		\rule[-10pt]{15cm}{0.05em}\\[1mm]
		\includegraphics[width=100mm]{./images/MSU-BIT}\\[13mm]
		{ Направление подготовки 01.03.02 <<Прикладная математика и информатика>>}\\[13mm]
		{ Выпускная квалификационная работа}\\[3mm]
	    \textbf{\Large\bfseries Регуляризующие алгоритмы в
	    	неустойчивых равновесных задачах}\\[15mm]
    

	    \begin{flushright}
	            \small Работу выполнил\\
	           \small студент \textbf{Сюй Минчуань}\\[10mm]
	            
	            \small Научный руководитель:\\
	            \small к.ф-м.н., доцент, \textbf{Б.А.Будак}%[10mm]
	            
	            %\small Научный консультант:\\
	            %\small звание, степень \textbf{ФИО}
	    \end{flushright}

	    \vspace{\fill}
	    \textbf{Шэньчжэнь}\\\textbf{2021}
	\end{center}
	\end{spacing}
\end{titlepage}

\clearpage
%\newpage

	\tableofcontents
	\thispagestyle{empty}
	\clearpage
	\setcounter{page}{3}

	\phantomsection
\addcontentsline{toc}{section}{Введение}
\section*{Введение}
	\textbf{Актуальность темы.} Многие важные проблемы исследования операций, вычислительной математики, математической экономики, связанных с поиском компромисса и согласования частично (или полностью) противоположных интересов сторон конфликта, сводятся к исследованию математических моделей, которое составляет область математики так называемую \emph{равновесным программированием}. В форме равновесной задачи можно записать задачи оптимального управления, многокритериального принятия решений в условиях неопределенности, обратные задачи оптимизации, вариационные неравенства, седловые задачи Лагранжа, задачи поиска равновесия по Нэшу и т.д. \cite{8}. 
	
	\textbf{Литературный обзор.} К настоящему времени достаточно хорошо изучена проблема существования и единственности равновесной точки невозмущенной задачи \cite{6}. Также в работах \cite{4}\cite{6} изложен результат о выпуклости и замкнутости множества решений. 
	Обсуждены, например, в работах \cite{7}-\cite{10-2} различные конкретные методы нахождения точек равновесия. Такие методы обычно имеют итерационный характер. Однако они исследовались при значительных ограничениях на $\Phi(v,w)$ и множество $\mathbf{W}$(см. постановку задачи), что редко бывает на практике не легко применяются. Особое внимание уделяется непрерывным методам, которые порождают целые семейства методов \cite{10}. На практике также обращается внимание на другую особенность поставленной задачи, а именно неустойчивость. С учетом возмущений исходных данных исследовались варианты подобных методов с использования стабилизатора для регуляризации. В различных работах предложены методы регуляризации неустойчивых равновесных задач \cite{11}-\cite{16}, в которых обычно используется стабилизатор вида скалярного произведения $\alpha_k\langle v,w\rangle$. Соответственно, можно попробовать и другие виды стабилизатора, например, $-\alpha_k\|v-w\|^2$. Следует отметить, что данная тема актуальна, и требуется новый подход для решения поставленной неустойчивой задачи.
	
	\textbf{Цель работы.} Разработка метода стабилизации с применением нового стабилизатора вида квадрата нормы для решения задач равновесного программирования с неточными данными, а также разработка регуляризованного экстраградиентного метода с использованием нового стабилизатора вида квадрата нормы. Проведение численных расчётов для проверки и практического использования разобранных методов. 
	
	\textbf{Методы исследования.} В работе используются теория и методы оптимизации, теория равновесного программирования, теория и методы решения некорректных поставленных задач, численные методы решения различных практических задач.
	
	\textbf{Практическая значимость.} Методы, разработанные в данной работе, могут быть применены к решению задач равновесного программирования, функция и множество которых неустойчивы к погрешностям задания исходных данных.
	
	\textbf{Объём и структура работы.} Данная работа состоит из введения, постановки задачи, четырёх глав, заключения, списка литературы и приложения. Полный объём работы составляет $44$ страницы с $1$ листингом алгоритма. Список литературы содержит $19$ наименований. В первой главе изложены основные факты теории равновесного программирования и теории методов решения некорректных поставленных задач. В последующих главах изложен новый материал: в главе 2 описан метод стабилизации для решения задач равновесного программирования с неточно заданным множеством путем применения нового стабилизатора вида квадрата нормы; в главе 3 описан регуляризованный экстраградиентый метод для решения задач равновесного программирования с неточно заданным множеством путем применения нового стабилизатора вида квадрата нормы; в главе 4 изложены результаты вычислительных экспериментов, основанных на разработанных в предыдущих главах методах; в заключении перечислены основные полученные результаты работы; в конце работы даны список литературы и приложения к работе. 
	
	\textbf{Благодарности.} Автор выражает благодарность своему научному руководителю Б.А. Будаку за ценные замечания, высказанные при обсуждении данной работы.
	

\clearpage
%\newpage
	%Разработка метода прогнозирования слабой масштабируемости суперкомпьютерных приложений
\phantomsection 
\addcontentsline{toc}{section}{Постановка задачи}
\counterwithout{equation}{section} 
\section*{Постановка задачи}
	Основную задачу равновесного программирования обычно формулируют таким образом: пусть имеется некоторая функция $\Phi(v,w)$, заданная на декартовом произведении $\mathbf{W}\times \mathbf{W}$, где $\mathbf{W}=\{w\in \mathbf{W_0}\subseteq\mathbb{R}^n:g_1(w)\leqslant 0,...,g_m(w)\leqslant 0 \}$ - заданное множество из пространства $\mathbb{R}^n$, $\mathbf{W_0}$ - множество из $\mathbb{E}^n$ (обычно имеет простой вид, возможно $\mathbf{W_{0}}=\mathbb{E}^n$). Требуется найти точку $v_*$ из $\mathbf{W}$, удовлетворяющую неравенству 
	\begin{equation}
	\centering
	\label{intro-1}
	\Phi(v_*,v_*)\leqslant\Phi(v_*,w),\quad \forall w\in \mathbf{W}.
	\end{equation}
	Такую точку называют \emph{равновесной точкой} задачи \eqref{intro-1}. Если функция $\Phi(v,w)$ не зависит от переменной $v$, то задача \eqref{intro-1} превращается в обычную \textit{задачу математического программирования}.
	
	Для этой задачи разработаны различные конструктивные методы, например экстраградиентный метод, проксимальный метод, метод линеаризации, метод стрельбы, которые подробно обсуждаются в следующей главе.  
	
	Следует учитывать, что эта задача, вообще говоря, \textbf{неустойчива} к возмущениям функции $\Phi(v,w)$ и множества $\mathbf{W}$, о чём свидетельствует простой \textbf{пример}:
	
	Допустим, что для возмущения функции и её градиента выполняются условия
	\begin{equation}
	\label{2}
	\lim\limits_{\delta\to 0} |\Phi^{\delta}(v,w)-\Phi(v,w)|=0;\quad  \lim\limits_{\delta\to 0} \|\nabla_w^{\delta}\Phi(v,w)-\nabla_w\Phi(v,w)\|=0
	\end{equation}
	Пусть $\Phi(v,w)=vw$, за приближенную функцию возьмём
	\begin{equation*}
	\Phi^{\delta}(v,w)=vw-\delta w^2,\mathbf{W}=\{w\in \mathbb{E}^1:|w|\leqslant 1 \}.
	\end{equation*}
	 Тогда $\nabla_w \Phi^{\delta}(v,w)=v-2\delta w$. Очевидно, что выполняется условие \eqref{2}, и множество решений $\mathbf{W_*}$ состоит из единственной точки $v_*=0$, однако приближенная задача
	\begin{equation}
	v_*^{\delta}\in \mathbf{W},\quad\Phi^{\delta}(v_*^{\delta},v_*^{\delta})\leqslant \Phi^{\delta}(v_*^{\delta},w),\quad \forall w\in \mathbf{W}
	\end{equation}
	не имеет решения при любых сколь угодно малых $\delta >0$, хотя здесь множество $\mathbf{W}$ - выпукло и компактно.
	
	 Понятно, что простое приближение задачи не даёт положительный результат, и для её решения нужно использовать методы, реализующие \textbf{идею регуляризации}, выдвинутую академиком А.Н.Тихонова. Как и ожидалось, существуют регуляризованные варианты методов решения задачи, опирающиеся на обычные упомянутые ранее методы Они будут подробно рассмотрены в следующей главе.
	 
	 Одно из важных понятий в теории равновесного программирования является \textit{сильной кососимметричностью}. Если выполнено
	 	\begin{equation*}
	 	\label{strong-koso}
	 	\Phi(w,w)-\Phi(w,v)-\Phi(v,w)+\Phi(v,v)\geqslant \mu\|v-w\|^2,\quad \forall w,v\in\mathbf{W_0},
	 	\end{equation*}
	 где $\mu>0$ - какое-то постоянное, то говорят, что функция $\Phi(v,w)$ сильно кососимметрична. 
	 
	 Легко проверяются, что $\langle v,w\rangle$ и $-\|v-w\|^2$ - кососимметричны:
	 \begin{equation*}
	 \begin{aligned}
	 &\langle v,w\rangle:\Phi(w,w)-\Phi(w,v)-\Phi(v,w)+\Phi(v,v)=1\cdot \|v-w\|^2,\forall v,w\in \mathbb{R}^n,\\
	 & -\|v-w\|^2:\Phi(w,w)-\Phi(w,v)-\Phi(v,w)+\Phi(v,v)=2\cdot\|v-w\|^2,\forall v,w\in \mathbb{R}^n.
	 \end{aligned}
	 \end{equation*}
	 В регуляризованных методах широко используется стабилизатор вида $\alpha_k\langle v,w\rangle$, но интересно и рассмотреть стабилизатор $-\alpha_k\| v-w\|^2$, который обладает рядом хороших свойств с точки зрения решения возмущенной задачи (например, свойством сильной кососимметричности, см. \cite{6}. Аналог этого свойства во выпуклом анализе - сильная выпуклость). 
	 
	 Исследование регуляризованных методов, в которых применен стабилизатор вида квадрата нормы, является \textbf{главной задачей} данной работы. Сразу стоит отметить, что данная работа не рассматривает все варианты конструктивных методов, а в основном концентрируется на одном - а именно на \textit{экстраградиентном методе}. 

\clearpage
% \newpage
%На счёт же третьей постановки - если мы ставим задачу разработки программного средства, то это средство нужно будет предъявить, и к нему будут предъявлять все обычные требования как к программному средству по оформлению, документированию и т.д., об этом тогда тоже не забудьте позаботиться.
	\counterwithin{equation}{section} 
\section{Обзор существующих подходов к решению \newline равновесных задач}
К данному моменту проведено большое количество исследований методов решения равновесных задач. Здесь мы проведем обзор основных методов к решению поставленной задачи, которая имеет характер неустойчивости при возмущенных данных. С учетом этого, мы будем концентрироваться как на методах, основанных на классической постановке, так и на регуляризованных методах, необходимых в случае возмущенных данных. Сразу заметим, что в приведенных ниже методах регуляризации даются только способы регуляризации, а не рассматриваются конкретные методы для вспомогательных равновесных задач. О них можно, например, узнать в работах \cite{centralbib-2}\cite{15}\cite{16}, или в главе 3.
\subsection{Основные конструктивные методы}
\textbf{Методы градиентного типа}: Концепция \textit{градиентного спуска}, широко используемая в теории оптимизации, переносится в область исследования равновесного программирования, но с учетом характера равновесных задач добавляется ещё один \textit{прогнозный шаг}. 

Рассмотрим вычислительный процесс, на каждой итерации которого в качестве вспомогательной задачи осуществляются две \textit{операции проектирования} на множество $\mathbf{W}$. Пусть начальное приближение $v_0\in \mathbf{W}$ - задано. Последовательный итерационный процесс описывается следующим образом:
\begin{equation}
\label{gradient method}
\left\{\begin{array}{l}
u_{k}=\operatorname{Pr}_{\mathbf{W}} \big[ v_k-\alpha_k\nabla_w\Phi(v_k,v_k)\big],\\
v_{k+1}=\operatorname{Pr}_{\mathbf{W}} \big[ v_k-\alpha_k\nabla_w\Phi(u_k,u_k)\big],
\end{array}\right.
\end{equation} 
$\mathrm{Pr}_{\mathbf{W}}$ - операция проектирования на множество $\mathbf{W}$. Вычисление $u_k$ - прогнозный шаг. Если этот шаг осуществляется только по одной из переменных, например, по первой, то получаем процесс в форме
\begin{equation}
\left\{\begin{array}{l}
u_{k}=\operatorname{Pr}_{\mathbf{W}} \big[ v_k-\alpha_k\nabla_w\Phi(v_k,v_k)\big],\\
v_{k+1}=\operatorname{Pr}_{\mathbf{W}} \big[ v_k-\alpha_k\nabla_w\Phi(u_k,\textcolor{blue}{v_k})\big],
\end{array}\right.
\end{equation}
Приведем теорему о сходимости метода к одному из равновесных решений. Доказательство её можно найти в \cite{8}\cite{8-2}.
\begin{theo}
	Если множество решений задачи \eqref{intro-1} не пусто и удовлетворяет условию кососимметричности\cite{8} на множестве $\mathbf{W}$, целевая функция $\Phi(v,w)$ непрерывна по $v$, и выпукла по $w$ при любом $v\in\mathbf{W}$, $g_i(w),i=1,...,m$ - выпуклы, $\mathbf{W}\in\mathbb{R}^n$ - выпуклое замкнутое множество, кроме того, функции $\nabla\Phi(v,w),g(w)$ и $\nabla g(w)$ удовлетворяют условиям Липшица с константами $|\nabla\Phi|,|g|$ и $|\nabla g|$, то последовательность $v_n$, порожденная методом \eqref{gradient method} с параметром
	\begin{equation*}
	0<\alpha_k<\frac{1}{\sqrt{2((|\nabla\Phi|+|\nabla g|L)^2+|g|^2)}},
	\end{equation*}
	 сходится монотонно по норме к одному из равновесных решений, т.е. $v_n\to v_*\in \mathbf{W}$ при $n\to \infty$.
\end{theo}
Обычно в литературе встречается термин \textit{экстраградиентный метод}, описывающий вычислительный процесс в случае $\mathbf{W}\ne \mathbb{R}^n$. 
 
В работе \cite{8-2} также вводились понятия \textit{вырожденного, острого и квадратичного} равновесия, которые обобщают понятие обычной равновесной точки, и для них формулировались и доказались теоремы о сходимости методов.

\textbf{Метод линеаризации}: Этот метод основан на идее \textit{аппроксимации допустимого множества изнутри} с помощью семейства пересекающихся шаров. Другими словами, мы делаем операцию проектирования на меняющееся, но простое линеаризованное множество.

Рассмотрим следующий итерационный процесс, полагая, что начальное приближение $v_0$ - \textit{строго допустимая} (или точка Слейтера), т.е. $g_i(v_0)<0,i=1,2,..,m$. Вычислим последовательность $\{v_n\}$ с помощью метода
\begin{equation}
\label{linear}
\begin{aligned}
&\left\{\begin{array}{l}
u_{k}=\operatorname{Pr}_{\Theta_k} \big[ v_k-\alpha_k\nabla_w\Phi(v_k,v_k)\big],\\
v_{k+1}=\operatorname{Pr}_{\Theta_k} \big[ v_k-\alpha_k\nabla_w\Phi(u_k,u_k)\big],
\end{array}\right.\\
\Theta_k = \{w\in\mathbf{W_0}:&\frac{1}{2}|w-v|^2e+\alpha_k\big[ \nabla g(v_k)(w-v_k)+g(v_k)\big]\leqslant 0 \}
\end{aligned}
\end{equation}
где $e=(1,...,1)\in\mathbb{R}^m$, $\alpha_k>0$ и $\nabla g(v)$ - матрица, состоящая из вектор-градиентов $g_i(v),i=1,2,...,m$.
Сначала напишем неравенство Липшица:
\begin{equation}
\label{lip-1}
g(w)-g(v)-\nabla g(v)(w-v)\leqslant \frac{L}{2}|w-v|^2,\quad \forall w\in\mathbf{W_0},\forall v\in\mathbf{W}.
\end{equation}
где $L$ - векторная константа с компонентами $L_i,i=1,2,...,m$, то есть каждая $i$-я компонента является константой Липшица для $i$-го функционального ограничения. Далее используем условие Липшица для оператора $\nabla_w\Phi$
\begin{equation}
\label{lip-2}
|\nabla_w\Phi(v+h,v+h)-\nabla_w\Phi(v,v)|\leqslant L_0|h|,\forall v,v+h\in\mathbf{W_0}.
\end{equation}
Сформулируем теорему о сходимости метода внутренней линеаризации. Доказательство можно найти в \cite{7}.
\begin{theo}
	Пусть задача \eqref{intro-1} удовлетворяет условию регулярности Слейтера, пусть также функция этой задачи представлена в канонической форме $\Phi(v,w)=P(w)+K(v,w)$, где $K(v,w)$ - кососимметричная функция. Пусть также $\Phi(v,w)$ и $g(w)$ - функции, выпуклые и дифференцируемые по $w\in\mathbf{W_0}$ для любого $v\in\mathbf{W_0}$, градиенты их удовлетворяют условиям Липшица \eqref{lip-1}\eqref{lip-2} с константами $L_0$ и $L$ (последняя векторная константа), последовательность $\bar{p}_n\leqslant C,p_{n+1}\leqslant C$ ограничены\cite{7}, $\mathbf{W_0}\in \mathbb{R}^n$ - выпуклое, замкнутое множество; тогда последовательность $v_n$, порожденная методом \eqref{linear} с параметром
	\begin{equation*}
	0<\alpha_k<\min\{1/L_1,...,1/L_m,[2(\alpha L_0^2+\langle L,C\rangle )]^{-1} \}
	\end{equation*}
	монотонно по норме пространства сходится к решению задачи \eqref{intro-1}. При этом последовательность $v_n$ является строго допустимой\cite{7}.
\end{theo}
Если выполнено также \textit{условие квадратичного равновесия}\cite{8}, то гарантируется сходимость метода со скоростью геометрической прогрессии, то есть
\begin{equation*}
|v_{k+1}-v_*|^2\leqslant q(\alpha)^{k+1}|v_0-v_*|^2, k\to \infty, q\in(0,1), \alpha\text{ - параметр}.
\end{equation*}

\textbf{Проксимальный метод}: Этот метод обычно применяется в случае, когда равновесные задачи становятся \textit{негладкими}, то есть функция $\Phi(v,w)$ не всегда дифференцируема по $w$.

Рассмотрим проксимальную итеративную схему, предпологая, что $v_0$ - начальное приближение из $\mathbf{W}$:
\begin{equation}
\label{approx}
\left\{\begin{array}{l}
u_k=\operatorname{Arg}\min\limits_{w\in\mathbf{W}}\big\{ \frac{1}{2}\|w-v_k\|^2+\alpha_k\Phi(v_k,w)\big\}\\
v_{k+1}=\operatorname{Arg}\min\limits_{w\in\mathbf{W}}\big\{ \frac{1}{2}\|w-v_k\|^2+\alpha_k\Phi(u_k,w)\big\}\\
\end{array}
\right.
\end{equation}

Сформулируем теорему о сходимости \textit{явного} проксимального прогнозного метода, с доказательством можно ознакомиться в \cite{9}.
\begin{theo}
Пусть множество решений задачи \eqref{intro-1} не пусто и функция $\Phi(v,w)$ удовлетворяет условию кососимметричности, непрерывна по $v$ и выпукла по $w$ при каждом $v\in\mathbf{W},\mathbf{W}\in\mathbb{R}^n$ - выпуклое замкнутое множество. Кроме того, функции $\Phi(v,w)$ и $g(w)$ выпуклы по $w$ и удовлетворяют условиям Липшица. Тогда последовательность $v_n$, порожденная методом \eqref{approx} с параметром 
\begin{equation*}
0<\alpha_k < \big(\sqrt{2(|\Phi|^2+|g|^2)} \big)^{-1}
\end{equation*}
сходится монотонно по норме к одному из равновесных решений, т.е. $v_n\to v_*$ при $n\to \infty$.
\end{theo}
Сразу заметим, что если $\Phi(v,w)$ - дифференцируема по $w$, то метод становится экстраградиентным. 

\textbf{Непрерывные варианты методов}: Рассмотрим непрерывные аналоги приведенных выше методов, представляющие собой генераторы целого семейства методов при использовании той или иной разностной схемы решения дифференциальных уравнений. Понятно, что их количество большое, поэтому здесь мы приведем лишь одну схему из возможных вариантов непрерывного экстраградиентного метода. 

Формально, предлагается непрерывный экстраградиентный метод второго порядка с переменной метрикой:
\begin{equation}
\label{conti-extra}
\left\{\begin{array}{l}
v(0)=v_0,\dot{v}(0)=v_1\quad t\geqslant 0.\\
u(t)=\operatorname{Pr}^{G(v(t))}_{\mathbf{W}}\big[ v(t)-\gamma(t)G^{-1}(v(t))\nabla_w\Phi(v(t),v(t))\big],\\
\mu(t)G^{-1}(v(t))\ddot{v}(t)+\beta(t)\dot{v}(t)+v(t)=\\
=\operatorname{Pr}^{G(v(t))}_{\mathbf{W}}\big[v(t)-\gamma(t)G^{-1}(v(t))\nabla_w\Phi(u(t),u(t)) \big],
\end{array}
\right.
\end{equation}
где $v_0$ - любая фиксированнная точка из $\mathbb{E}^n$, $\nabla_w\Phi(v,w)$ - градиент функции $\Phi(v,w)$ по переменной $w$, $\gamma(t),\mu(t),\beta(t)$ - параметры метода, $G(v)$ для каждого $v$ - заданная симметричная положительно определенная матрица, $\operatorname{Pr}^{G(v(t))}_{\mathbf{W}}[z]$ - так называемая \textit{$G$-проекция} точки $z$ на множество $\mathbf{W}$, то есть точка $\operatorname{Pr}^{G(v(t))}_{\mathbf{W}}[z]$ - решение задачи минимизации
\begin{equation*}
f(y)=\frac{1}{2}\langle G(v(t))(y-z),y-z \rangle \to\inf_{y\in\mathbf{W}}
\end{equation*}
Формулируем теорему о сходимости для метода \eqref{conti-extra}.
\begin{theo} Пусть выполнены следующие условия:\\
\textbf{1)} $\mathbf{W} \subseteq \mathbb{E}^{n}$ - выпуклое замкнутое множество, множество решений задачи \eqref{intro-1} $\mathbf{W_{*}}$ неnycтo; \\
\textbf{2)} Функция $\Phi(v,w)$ выпукла, непрерывно дифференцируема по $w$ при любом $v $ из $ \mathbb{E}^{n} $; удовлетворяет условию кососимметричности на $\mathbf{W}$:
\begin{equation*}
\Phi(v, v)-\Phi(v, w)-\Phi(w, v)+\Phi(w, w) \geqslant 0 \quad \forall v, w \in \mathbf{W}
\end{equation*}
сужение ее частного градиента по второй перемениой на диагональ множества $ \mathbb{E}^{n} \times \mathbb{E}^{n}$ удовлетворяет условия Липшица
\begin{equation*}
\left\|\nabla_{w} \Phi(u, u)-\nabla_{w} \Phi(v, v)\right\| \leqslant L_{w}\|u-v\|, \quad \forall u, v \in \mathbb{E}^{n}
\end{equation*}
\textbf{3)} $G(v)$ - симметричная положительно определенная матрица при любом $v$ из $\mathbb{E}^n$; существуют сильно выпуклая дважды дифференцируемая функция $\Psi(v)$ и положительные константы $m, M$, где $m \leqslant M$, такие что
\begin{equation*}
G(v) \equiv \Psi^{\prime \prime}(v) ; \quad m\|w\|^{2} \leqslant\langle G(v) w, w\rangle \leqslant M\|w\|^{2} \quad \forall v, w \in \mathbb{E}^{n}
\end{equation*}
\textbf{4)} Параметры $\mu(t), \beta(t), \gamma(t) $ удовлетворяют следующим условиям:
\begin{equation*}
\begin{aligned}
&\gamma(t)>0 ; \lim _{t \rightarrow \infty} \gamma(t)=\gamma_{0} ; 0<\gamma_{0}<\min \left\{\frac{m}{6 L_{w}}, \frac{m^{2}}{4 M L_{w}}\right\};\\
&\mu(t) \in C^{2}[0 ;+\infty) ; \beta(t) \in C^{1}[0 ;+\infty) ; \mu(t) \geqslant \mu_{0}>0 ; m \beta^{2}(t)>4 \mu(t);\\
&\mu^{\prime \prime}(t) \geqslant 0, \mu^{\prime}(t) \leqslant 0, \beta^{\prime}(t) \leqslant 0, \lim _{t \rightarrow \infty} \beta(t)=\beta_{\infty}>0, \lim _{t \rightarrow \infty}\mu^{\prime}(t)=\mu_{\infty}^{\prime} 
\end{aligned}
\end{equation*}
Тогда существует такая точка $v^{\prime}$ из множества решений $\mathbf{W_{*}}$ задачи \eqref{intro-1}, что 
\begin{equation*}
\lim\limits_{t\to\infty}\|v(t)-v^{\prime}\|=0;\lim\limits_{t\to\infty}\|\dot{v}(t)\|=0;\lim\limits_{t\to\infty}\|\ddot{v}(t)\|=0;
\end{equation*}

\end{theo}

О доказательстве этой теоремы, о непрерывных методах других итеративных методов, а также о их регуляризованных вариантах методов можно узнать в диссертации \cite{10}. 

\textbf{Метод стрельбы}: Один из новых разработанных методов к решению равновесных задач является так называемый \textit{метод стрельбы}. Его вычислительный процесс выглядит следующим образом:
\begin{equation}
\label{shooting}
\left\{\begin{array}{l}
v_0\in \mathbf{W}\text{ известно},\\
w_k = \operatorname{Pr}_{\mathbf{W}}(v_k-\alpha_k\nabla_w\Phi(v_k,v_k)),\\
C_k = \{z\in \mathbf{W}:\|w_k-z\|\leqslant\|v_k-z\|\},\\
Q_k=\{z\in\mathbf{W}:\langle z-v_k,v_k,v_0\rangle\geqslant 0 \},\\
v_{k+1} = \operatorname{Pr}_{C_k\cap Q_k}(v_0),\quad k=0,1,2,...
\end{array}
\right.
\end{equation}

Название ``метод стрельбы'' взято как естественная интерпретация приведенных формул – в самом деле, проектируется одна и та же точка на \textit{постоянно меняющееся} множество, что вполне логично трактовать как стрельбу по движущейся мишени.

Достаточные условия, гарантирующие сходимость метода \eqref{shooting} к множеству решений задачи \eqref{intro-1}, описывает
следующая
\begin{theo}
Пусть $\mathbf{W}$ - непустое замкнутое выпуклое множество гильбертова пространства $\mathbb{H}$, функционал $\Phi(v, w)$ выпуклый и непрерывно дифференцируемый по Фреше по переменной $w$ на $\mathbf{W}$ при любом $v\in\mathbf{W}$, сужение его частного градиента $\nabla_w\Phi(v, v)$ удовлетворяет условию обратимой строгой монотонности с коэффициентом $\alpha$:
\begin{equation*}
\langle \nabla_w\Phi(v,v)-\nabla_w\Phi(w,w),v-w\rangle\geqslant \alpha\|\nabla_w\Phi(v,v)-\nabla_w\Phi(w,w)\|^2,\forall v,w\in\mathbf{W}.
\end{equation*}
Пусть множество $\mathbf{W_{*}}$ решений задачи \eqref{intro-1} непусто, а шаг метода \eqref{shooting} удовлетворяет условию
$\alpha_k\in[\varepsilon, 2\alpha)$, где $0 < \varepsilon< 2\alpha$. Тогда все слабые предельные точки генерируемой им последовательности $v_k$ принадлежат $\mathbf{W_{*}}$. 

Если, кроме того, множество $\operatorname{Pr}_{\mathbf{W_{*}}}(v_0)$ непусто, то последовательность $v_k$ сходится к нему по норме $\mathbb{H}$.
\end{theo}
Подробные доказательство и обсуждение можно найти в \cite{10-2}.

Итак, в этом разделе рассмотрены различные методы решения исходной задачи \eqref{intro-1}, которые применяются в разных случаях. В следующем разделе обсудим методы, применяемые в случае присутствия возмущений данных функции $\Phi(v,w)$ и множества $\mathbf{W}$. 
\subsection{Методы регуляризации}
Прежде всего, выделяем два основных направления учета ограничений множества $\mathbf{W}$: методы, основанные на \textit{расширении множества}, и методы, основанные на \textit{сочетании с штрафной функцией}. 

\emph{Сочетание с штрафной функцией}:
Вместо сложного множества $\mathbf{W}$ работаем на более простом множестве $\mathbf{W_0}$ (возможно $\mathbf{W_0}=\mathbb{R}^n$), добавив штрафное слагаемое $AP_{\delta}(w)$ в тихоновскую функцию. Эту идею можем увидеть в \cite{5}, стр. 323.

\emph{Расширение множества $\mathbf{W}$}:
Эта идея снятия ограничения заключается в том, что мы ``расширяем'' множество $\mathbf{W}$ на $\mathbf{W}(\delta)$:
\begin{equation*}
\mathbf{W}(\delta)=\{w\in \mathbf{W_0}:g_{i\delta}^+(w)\leqslant\Theta[1+\Omega(w)],i=\overline{1,s}\}.
\end{equation*}
При $\forall \Theta\geqslant\delta>0$ имеем $\mathbf{W}\subset \mathbf{W}(\delta).$

Предположим, что множество $\mathbf{W}$ теперь задается ограничениями типа \textit{неравенств и равенств}:
\begin{equation*}
\mathbf{W}=\{w\in\mathbf{W_0} :g_i(w)\leqslant 0,i=1,...,m;g_i(w)=0,i=m+1,...,s\}.
\end{equation*}
где $\Omega(w)$ - какая-либо функция со следующими свойствами: $\Omega(w)\geqslant 0,\forall w\in\mathbf{W_0}$ и множество $\{w\in\mathbf{W_0}:\Omega(w)\leqslant c \}$ ограничено при всех $c$, при которых это множество не пусто. Функцию $\Omega(w)$
с указанными свойствами будем называть \textit{стабилизатором} задачи \eqref{intro-1}. Предположим, что приближения $\Phi_{\delta}(v, w), g_{i\delta}(w), w \in\mathbf{W_0}$, функций $\Phi( v, w), g_i(w)$ таковы, что
\begin{equation}
\label{noiiise}
\begin{aligned}
&|\Phi_{\delta}(v,w)-\Phi(v,w)|\leqslant \delta(1+\Omega(w)),\forall v\in \mathbf{W_0},w\in\mathbf{W_0},\\
&|g_{i\delta}(w)-g_{i}(w)|\leqslant \delta(1+\Omega(w)),\forall v\in \mathbf{W_0},i=1,...,s,
\end{aligned}
\end{equation}
где $\delta > 0$ - мера погрешности.

Далее будем предполагать, что для некоторой точки $v^* \in\mathbf{W^{*}}$ существуют постоянные $c_1\geqslant 0,...,c_s\geqslant 0$ такие, что
\begin{equation}
\label{key-condition}
\Phi(v_*,v_*)\leqslant \Phi(v_*,w)+\sum_{i=1}^{s}c_ig_i^+(w),\forall w\in\mathbf{W_0}.
\end{equation}
где $g_i^+(w)=\max\{g_i(w),0\},i=1,....,m$, и $g_i^+(w)=|g_i(w)|,i=m+1,....,s$. В \cite{centralbib} приведены классы задач, удовлетворяющие этому условию.

В ниже приведенных методах используется \textit{идея расширения множества}. Похожие схемы можно увидеть и в случае, когда используется сочетание с штрафной функцией\cite{centralbib}\cite{12}\cite{13}. Поэтому здесь мы ограничимся рассмотрением расширения множества.

\textbf{Метод стабилизации:} Начнем рассмотрение со схемы метода стабилизации. Введем функцию Тихонова: 
\begin{equation}
t_{\delta} (v,w)=\Phi_{\delta}(v,w)+\alpha \Omega(w),\quad w\in \mathbf{W_0},\alpha>0.
\end{equation}
Будем искать точку, удовлетворяющую условиям
\begin{equation}
\label{stab-method}
v_{\delta}\in \mathbf{W_{\delta}},\quad t_{\delta}(v_\delta,v_\delta)\leqslant \inf\limits_{w\in W(\delta)}t_{\delta}(v_{\delta},w)+\varepsilon,\quad \varepsilon>0.
\end{equation}
Доказана теорема\cite{11}, устанавливающая сходимость метода при определенном выборе параметров.
\begin{theo}
\label{theo-stab}
Пусть выполнены следующие условия:\\
\textbf{1)} $\mathbf{W_0}$ - замкнутое ограниченное множество, функции $g_i(w),i=1,...,m,|g_i(w)|,i=m+1,\ldots,s,\Phi(w,w)$ полунепрерывны снизу на $\mathbf{W_0}$; функция $\Phi(v,w)$ полунепрерывна сверху по $v$ на $\mathbf{W_0}$ при любом фиксированном $w\in\mathbf{W_0}$; множество $\mathbf{W_*}$ решений задачи \eqref{intro-1} непусто; для некоторой точки $v_*\in\mathbf{W_*}$ выполнено неравенство \eqref{key-condition}; функция $\Phi(v,w)$ кососимметрична на $\mathbf{W_0}$,\\
\textbf{2)} $\Omega(w)$ - стаблизатор задачи \eqref{intro-1},\\
\textbf{3)} приближения $\Phi_{\delta}(v,w),g_{i\delta}(w)$ функций $\Phi(v,w),P(w)$ удовлетворяют условиям \eqref{noiiise};\\
\textbf{4)} параметры $\alpha=\alpha(\delta),\Theta=\Theta(\delta),\varepsilon=\varepsilon(\delta),\delta>0$ метода \eqref{stab-method} таковы, что
\begin{equation}
\begin{aligned}
&\alpha(\delta)>0,\Theta(\delta)\geqslant \delta>0,\varepsilon(\delta)>0,\lim_{\delta\rightarrow 0} [\alpha(\delta+\Theta(\delta)+\varepsilon(\delta))]=0,\\
&\sup_{\delta>0}\frac{\delta+2|c|_1\Theta(\delta)}{\alpha(\delta)}<1,[\delta+\alpha(\delta)][1+\Omega(v_*)]\leqslant \varepsilon(\delta), \sup_{\delta>0}\frac{\varepsilon(\delta)}{\alpha(\delta)}<+\infty.
\end{aligned}
\end{equation}
Тогда множество $\mathbf{W_{*\delta}}$ точек, удовлетворяющих \eqref{stab-method}, непусто, при всех $\delta>0$, и 
\begin{equation}
\label{conv}
\lim_{\delta\rightarrow 0}\rho(v_{\delta},\mathbf{W_*})=0,\quad\lim_{\delta\rightarrow 0} \rho(\Phi(v_{\delta},v_{\delta}),\Phi_*)=0,
\end{equation}
где $\Phi_*$ - множество значений функций $\Phi(v,v)$, когда $v$ пробегает множество $\mathbf{W_*}$, причем сходимость в \eqref{conv} равномерная относительно выбора $\Phi_{\delta}(v,w),P_{\delta}(w)$ из \eqref{noiiise} и точки $v_{\delta}$ из $\mathbf{W_{*\delta}}$.
\end{theo}

\textbf{Метод невязки:} В этом методе предполагается введение множества
\begin{equation*}
V(\delta)=\{v\in W(\delta):\Phi_{\delta}(v,v)\leqslant \inf\limits_{w\in W(\delta)}[\Phi_{\delta}(v,w)+\Theta M_1\Omega(w)]+\sigma,\sigma>0\}.
\end{equation*}
где $M_1$ - достаточно большая константа. Будем искать точку $v=v_{\delta}$ из условий 
\begin{equation}
\label{neva-method}
v_{\delta}\in V(\delta),\Omega(v_{\delta})\leqslant\inf\limits_{v\in V(\delta)}\Omega(w)+\mu,\mu>0.
\end{equation} Формально описан метод невязки. Приведем теорему для установления сходимости метода.
\begin{theo}
Пусть выполнены условия \textbf{1) - 3)} теоремы \ref{theo-stab} и, кроме того, параметры $\Theta=\Theta(\delta),\sigma=\sigma(\delta)$ метода \eqref{neva-method} таковы, что
\begin{equation*}
\Theta(\delta)\geqslant \delta >0,\sigma(\delta)>0,\mu(\delta)<0,\lim\limits_{\delta\to 0}[\Theta(\delta)+\sigma(\delta)]=0,2M_1\Theta(\delta)\leqslant\sigma(\delta),\sup\limits_{\delta>0}\mu(\delta)<\infty.
\end{equation*}

Тогда множество $V(\delta) \ne\varnothing,\forall \delta > 0$, множество $\mathbf{W_{*\delta}}$ точек $v_{\delta}$, удовлетворяющих условиям \eqref{neva-method}, непусто при всех $\delta>0 $ и для метода \eqref{neva-method} также справедливы неравенства \eqref{conv}. Если дополнительно известно, что функция $\Omega(w)$ полунепрерывна снизу на $\mathbf{W_0}$ и $\lim\limits_{\delta\to\infty}\mu(\delta) = 0$, то
\begin{equation}
\label{conv-2}
\lim_{\delta\rightarrow 0}\rho(v_{\delta},\mathbf{W_{**}})=0,\quad\lim_{\delta\rightarrow 0} \rho(\Phi(v_{\delta},v_{\delta}),\Phi_{**})=0,
\end{equation}
где $\mathbf{W_{**}}=\{v\in \mathbf{W_*} : \Omega(v)\leqslant \Omega(v_*)\},\Phi_{**}=\{\Phi:\Phi=\Phi(v,v),v\in\mathbf{W_{**}} \}$. Сходимость в \eqref{conv}\eqref{conv-2} равномерная относительно выбора $\Phi_{\delta}(v,w),g_{i\delta}(w)$ из \eqref{noiiise} и точки $v_{\delta}$ из $\mathbf{W_{*\delta}}$.
\end{theo}

\textbf{Метод квазирешений:} Предполагается, что $\Omega(v_*)\leqslant r$, где $r$ - число, $v_*$ - какая-то точка из $\mathbf{W_*}$, для которой выполняется условие \eqref{key-condition}. Введем множество
\begin{equation*}
\mathbf{W}(\delta,r)=\{ v\in\mathbf{W}(\delta):\Omega(v)\leqslant r \}.
\end{equation*}
Понятно, что $v_*\in \mathbf{W}(\delta,r)$, так что $\mathbf{W}(\delta,r)\ne \varnothing,\forall \delta >0$. Ищется точка
\begin{equation}
\label{kvasi-method}
 v_{\delta}: v_{\delta}\in W(\delta,r),\Phi_{\delta}(v_{\delta},v_{\delta})\leqslant\inf\limits_{w\in W(\delta)}[\Phi_{\delta}(v_{\delta},w)+\Theta M_2\Omega(w)]+\xi,\xi>0.
\end{equation}
Здесь $M_2$ - const, $W(\delta,r)=\{v\in W(\delta):\Omega(v)\leqslant r \}$. Приведем теорему о сходимости.
\begin{theo}
Пусть выполнены условия \textbf{1) - 3)} теоремы \ref{theo-stab}, функция $\Omega(w)$ полунепрерывна снизу на $\mathbf{W_0}$, выполнено $\Omega(v_*)\leqslant r$ и параметры $\Theta=\Theta(\delta),\xi=\xi(\delta)$ метода \eqref{kvasi-method} таковы, что
\begin{equation*}
\Theta(\delta)\geqslant \delta >0,\xi(\delta)>0,\lim\limits_{\delta\to 0}[\Theta(\delta)+\xi(\delta)]=0,2M_2\Theta(1+r)\leqslant\xi(\delta),
\end{equation*}
Тогда множество $\mathbf{W_{*\delta}}$ точек $v_{\delta}$, удовлетворяющих условиям \eqref{kvasi-method}, непусто при всех $\delta>0 $ и 
\begin{equation}
\label{conv-3}
\lim_{\delta\rightarrow 0}\rho(v_{\delta},\mathbf{W_{*r}})=0,\quad\lim_{\delta\rightarrow 0} \rho(\Phi(v_{\delta},v_{\delta}),\Phi_{*r})=0,
\end{equation}
где $\mathbf{W_{*r}}=\{v\in \mathbf{W_*} : \Omega(v)\leqslant r\},\Phi_{*r}=\{\Phi:\Phi=\Phi(v,v),v\in\mathbf{W_{*r}} \}$. Сходимость в \eqref{conv-3} равномерная относительно выбора $\Phi_{\delta}(v,w),g_{i\delta}(w)$ из \eqref{noiiise} и точки $v_{\delta}$ из $\mathbf{W_{*\delta}}$.
\end{theo}

Итак, формально описаны методы регуляризации. В работах(например, \cite{centralbib}\cite{centralbib-2}\cite{15}) широко используется стабилизатор вида $\alpha_k\langle v,w\rangle$. В следующих главах будет обсуждаться вопрос о том, что будет получаться, если используется новый стабилизатор задачи, то есть стабилизатор вида $-\alpha_k\|v-w\|^2$, который учитывает характер равновесной задачи и сам обладает хорошим свойством, пригодным для решения прикладных равновесных задач.
\clearpage

	\section{Метод стабилизации для неустойчивых равновесных задач}
Первый вопрос, который обсуждается в данной работе, такой: что будет получаться в методе стабилизации, если используется стабилизатор вида $-\alpha_k\|v-w\|^2$. Будем рассматривать метод стабилизации, использующий идеи метода штрафов.

Возникает естественный вопрос: в чем причина использования такого нового стабилизатора? Оказывается, что он обладает хорошим свойством, называемым \textit{сильной кососимметричностью}. Это свойство аналогично свойству сильной выпуклости в теории оптимизации - сильно выпуклость. Как известно, в идее регуляризации Тихонова к минимизируемому функционалу добавлен член $\alpha_k\|u\|^2$, который сильно выпуклый. Соответственно, мы добавляем сильно кососимметричный член $-\alpha_k\|v-w\|^2$ к функции $\Phi(v,w)$.
\subsection{Некоторые предварительные рассуждения}
Вернемся к постановке задачи. Рассматривается задача равновесного программирования: найти точку $v_*$ из условий
\begin{equation}
\label{question}
\begin{aligned}
v_*\in\mathbf{W}=\{w\in\mathbf{W_0} :g_i(w)&\leqslant 0,i=1,\ldots,m;g_i(w)=0,i=m+1,\ldots,s\},\\
\Phi(v_*,v_*)&\leqslant\Phi(v_*,w),\quad\forall w\in\mathbf{W},
\end{aligned}
\end{equation}
где $\mathbf{W_0}$ - заданное множество из евклидова пространства $\mathbb{E}^n$. Функции $\Phi(v,w),g_i(w),i=1,\ldots,s$ определены на множестве $\mathbf{W_0}$. Точки $v_*$, удовлетворяющие условиям \eqref{question}, называются \emph{точками равновесия}. Множество точек равновесия будем обозначать через $\mathbf{W_*}$. Предполагается, что $\mathbf{W_*}\ne\varnothing$.
\begin{defin}
	Будем называть функцию
	\begin{equation}
	\label{penalty}
	P(w)=\sum_{i=1}^{s}[g^+_i(w)]^p,\quad w\in\mathbf{W_0},p>0
	\end{equation}
	\textbf{простейшей штрафной функцией}, где
	\begin{equation*}
	g^+_i=\max\{0;g_i\},i=1,...,m;g^+_i=|g_i|,i=m+1,...,s.
	\end{equation*}
\end{defin}
\noindent Пусть приближения $\Phi_{\delta}(v,w),P_{\delta}(w)$ для функций $\Phi(v,w),P(w)$ таковы, что
\begin{equation}
\label{noise}
\begin{aligned}
&\big|\Phi_{\delta}(v,w)-\Phi(v,w)\big|\leqslant \delta\big(1+\|v\|^2+\|w\|^2\big),\quad v,w\in\mathbf{W_0},\delta >0,\\
&\big|P_{\delta}(w)-P(w)\big|\leqslant \delta\big(1+\|w\|^2\big),\quad w\in\mathbf{W_0},\delta >0.
\end{aligned}
\end{equation}
По аналогии с методом стабилизации с учетом ограничений множества и возмущений введем функцию Тихонова, используя стабилизатор вида $\Omega(v,w)=-\|v-w\|^2$:
\begin{equation}
\label{Tikhonov}
t_{\delta}(v,w)=\Phi_{\delta}(v,w)+AP_{\delta}(w)-\alpha\|v-w\|^2,\quad v,w\in\mathbf{W_0},\alpha>0,A>0.
\end{equation}
Отметим, что этот стабилизатор, вообще говоря, не обладает свойствам, упомянутым в методах регуляризации предыдущей главы. Кроме того, это стабилизатор, зависящий от двух переменных. Тем не менее, он обладает свойством сильной кососимметричности. Будем искать точку $v_{\delta}$, удовлетворяющую условиям
\begin{equation}
\label{Tikhonov-condition}
v_{\delta}\in\mathbf{W_0},\quad t_{\delta}(v_{\delta},v_{\delta})\leqslant t_{\delta}(v_{\delta},w)+\varepsilon,\quad \forall w\in\mathbf{W_0},\varepsilon>0,
\end{equation}
т.е. искать равновесную точку функции Тихонова. Допустим, что для некоторой точки $v_*\in \mathbf{W_*}$ существует постоянные $v>0,c_i\geqslant 0$ такие, что
\begin{equation}
\label{7}
\Phi(v_*,v_*)\leqslant \Phi(v_*,w)+\sum_{i=1}^{s} c_i[g_i^+(w)]^v,\quad \forall w\in \mathbf{W_0}
\end{equation}
Это условие мы уже видели в обсуждении методов регуляризации. Формулируем первое вспомогательное 
\begin{prop}
	\label{InequOfPenFun}
	Если для точки $v_*\in\mathbf{W_*}$ выполнено условие \eqref{7}, $p\geqslant v$, то справедливы неравенства 
	\begin{equation}
	\label{propsition-1}
	\begin{aligned}
	&\sum_{i=1}^{s}c_i[g_i^+(w)]^v\leqslant AP(w)+BA^{-v/(p-v)},\quad \forall w\in\mathbf{W_0},A>0,p>v,\\
	&\sum_{i=1}^{s}c_i[g_i^+(w)]^v\leqslant AP(w),\quad \forall w\in\mathbf{W_0},A>|c|_{\infty}=\max\limits_{1\leqslant i\leqslant s}c_i,p=v,
	\end{aligned}
	\end{equation}
	где $B=(p-v)v^{v/(p-v)}p^{-v/(p-v)}|c|^{v/(p-v)},|c|=(\sum\limits_{i=1}^{s}|c_i|^{p/(p-v)})^{(p-v)/p}$ при $p>v$.
\end{prop}
\noindent\emph{Доказательство.} см. \cite{centralbib}, cтр. 5. $\quad\qedsymbol$

Далее приведем и \textit{докажем} предложение о достаточном условии для непустоты множества $\mathbf{W_{*\delta}}$, т.е. множество равновесных точек функции Тихонова.
\begin{prop}
	\label{propsition-2}
	Если 
	\begin{equation}
	\label{13}
	\begin{aligned}
	&BA^{-v/(p-v)}+\delta\|v_*\|^2(3+A)+2(\delta+A\delta)\leqslant 1/2\varepsilon(\delta),\\ &\alpha\|v_*-w\|^2+\|w\|^2(\delta+A\delta)\leqslant 1/2 \varepsilon(\delta),\quad\forall w\in\mathbf{W_0},\delta>0,
	\end{aligned}
	\end{equation}
	где точка $v_*\in\mathbf{W_*}$ выполнено условие \eqref{7}, $p\geqslant v$, то в \eqref{Tikhonov-condition} можно принять $v_{\delta}=v_*$, то есть при этих предположениях множество $\mathbf{W_{*\delta}}$ точек $v_{\delta}$, удовлетворяющих условию \eqref{Tikhonov-condition} при выборе $\Phi_{\delta}(v,w),P_{\delta}(w)$ из \eqref{noise}, непусто.
\end{prop}
\begin{remark}
	Второе условия предложения будет выполнено, например, если на множество $\mathbf{W_0}$ накладывается условие ограниченности.
\end{remark}

\noindent\emph{Доказательство.} В самом деле, в силу \eqref{noise}\eqref{Tikhonov}\eqref{7}\eqref{propsition-1}\eqref{13} и $P(v_*)=0$ имеем
\begin{align}
&t_{\delta}(v_*,v_*)\stackrel{\eqref{Tikhonov}}{=}\Phi_{\delta}(v_*,v_*)+AP_{\delta}(v_*)\stackrel{\eqref{noise}}{\leqslant}\notag\\
&\leqslant\Phi(v_*,v_*)+\delta(1+2\|v_*\|^2)+\overbrace{AP(v_*)}^{=0}+A[\delta(1+\|v_*\|^2)]\stackrel{\eqref{7}}{\leqslant}\notag\\
&\leqslant \Phi(v_*,w)+\sum_{i=1}^{s}c_i[g_i^+(w)]^v+\delta\|v_*\|^2(2+A)+\delta+A\delta\stackrel{\eqref{propsition-1}}{\leqslant}\notag\\
&\leqslant\Phi(v_*,w)+AP(w)+BA^{-v/(p-v)}+\delta\|v_*\|^2(2+A)+\delta+A\delta\stackrel{\eqref{noise}}{\leqslant}\\
&\leqslant \Phi_{\delta}(v_*,w)+AP_{\delta}(w)\textcolor{blue}{-\alpha\|v_*-w\|^2}\textcolor{blue}{+\alpha\|v_*-w\|^2}+\notag\\
&\quad +BA^{-v/(p-v)}+\delta\|v_*\|^2(3+A)+(2+\|w\|^2)(\delta+A\delta)\stackrel{\eqref{Tikhonov}\eqref{13}}{\leqslant}\notag\\
&\leqslant t_{\delta}(v_*,w)+\underbrace{BA^{-v/(p-v)}+\delta\|v_*\|^2(3+A)+2(\delta+A\delta)}_{\leqslant 1/2\varepsilon(\delta)}+\notag\\
&\quad +\underbrace{\alpha\|v_*-w\|^2+\|w\|^2(\delta+A\delta)}_{\leqslant 1/2\varepsilon(\delta)}\leqslant t_{\delta}(v_*,w)+\varepsilon(\delta).\,\quad \qedsymbol \notag
\end{align}
Заметим, что дальнейшее изложение не зависит от способа поиска
точек $v_{\delta}\in\mathbf{W_{*\delta}}$, для нас будет важен лишь сам факт существования таких точек.
\begin{defin}
	Если выполнено
	\begin{equation}
	\label{cososymmetric}
	\Phi(w,w)-\Phi(w,v)-\Phi(v,w)+\Phi(v,v)\geqslant 0,\quad \forall w,v\in\mathbf{W_0},
	\end{equation}
	то говорят, что функция $\Phi(v,w)$ \textbf{кососимметрична}. 
\end{defin}
\noindent Свойства функций, принадлежащих к классу кососимметричных, можно посмотреть, например, в \cite{8}. Заметим, что из \eqref{cososymmetric} при $v=v_*$ и условия \eqref{7} следует, что 
\begin{equation}
\label{15}
\Phi(w,w)-\Phi(w,v_*)\geqslant \Phi(v_*,w)-\Phi(v_*,v_*)\geqslant -\sum_{i=1}^{s}c_i[g_i^+(w)]^v,\quad \forall w\in\mathbf{W_0}.
\end{equation}
\noindent Ещё одно предложение, которое используется в ходе доказательства метода:
\begin{prop}
	\label{lemma11}
	Пусть $z,b,d\geqslant 0,p>1$ таковы, что
	\begin{equation}
	0\leqslant z^p\leqslant bz+d.
	\end{equation}
	Тогда
	\begin{equation}
	0\leqslant z\leqslant (b^q+qd)^{1/p},
	\end{equation}
	где $q$ определяется равенством $\frac{1}{p}+\frac{1}{q}=1$.
\end{prop}
\noindent\emph{Доказатеьство.} см. \cite{numMethodopt}, cтр. 99, лемма 11. $\quad\qedsymbol$
\subsection{Описание метода стабилизации с использованием нового стабилизатора}
Переходим к самому описанию метода стабилизации. Основная теорема, описывающая сходимость метода стабилизации, формулируется и доказывается следующим образом: 
\begin{theo}
	\label{theor-1}
	Пусть выполнены следующие условия\\
	\textbf{1)} $\mathbf{W_0}$ - замкнутое ограниченное множество, функции $g_i(w),i=1,...,m,|g_i(w)|,i=m+1,\ldots,s,\Phi(w,w)$ полунепрерывны снизу на $\mathbf{W_0}$; функция $\Phi(v,w)$ полунепрерывна сверху по $v$ на $\mathbf{W_0}$ при любом фиксированном $w\in\mathbf{W_0}$; множество $\mathbf{W_*}$ решений задачи \eqref{question} непусто; для некоторой точки $v_*\in\mathbf{W_*}$ выполнено неравенство \eqref{7}; функция $\Phi(v,w)$ кососимметрична на $\mathbf{W_0}$, т.е. выполнено неравенство \eqref{cososymmetric};\\
	\textbf{2)} $\Omega(v,w)=-\alpha_k\|v-w\|^2$ - стабилизатор задачи \eqref{question}, $P(w)$ - штрафная функция, определенная формулой \eqref{penalty} при $p\geqslant v$;\\
	\textbf{3)} приближения $\Phi_{\delta}(v,w),P_{\delta}(w)$ функций $\Phi(v,w),P(w)$ удовлетворяют условиям \eqref{noise};\\
	\textbf{4)} параметры $\alpha=\alpha(\delta)>0,A=A(\delta)>0,\varepsilon=\varepsilon(\delta)>0,\delta>0$, удовлетворяют условиям \eqref{13} и, кроме того,
	\begin{equation}
	\label{16}
	\begin{aligned}
	&\lim_{\delta\rightarrow 0}\alpha(\delta)=0,\quad \lim_{\delta\rightarrow 0}A(\delta)=0,\quad \lim_{\delta\rightarrow 0}\varepsilon(\delta)=0,\quad\lim_{\delta\rightarrow 0} \delta A(\delta)=0,\\
	&\sup_{\delta>0}\frac{3\delta+\delta A(\delta)}{\alpha(\delta)}<+\infty,\quad \sup_{\delta>0}\frac{\varepsilon(\delta)}{\alpha(\delta)}<+\infty.
	\end{aligned}
	\end{equation}
	Тогда множество $\mathbf{W_{*\delta}}\ne\varnothing$ при всех $\delta>0$, и 
	\begin{equation}
	\label{17}
	\lim_{\delta\rightarrow 0}\rho(v_{\delta},\mathbf{W_*})=0,\quad\lim_{\delta\rightarrow 0} \rho(\Phi(v_{\delta},v_{\delta}),\Phi_*)=0,
	\end{equation}
	где $\Phi_*$ - множество значений функций $\Phi(v,v)$, когда $v$ пробегает множество $\mathbf{W_*}$, причем сходимость в \eqref{17} равномерная относительно выбора $\Phi_{\delta}(v,w),P_{\delta}(w)$ из \eqref{noise} и точки $v_{\delta}$ из $\mathbf{W_{*\delta}}$.
\end{theo}
\noindent\emph{Доказательство.} Из предложения \ref{propsition-2} и условия \eqref{7}\eqref{13} следует, что $\mathbf{W_{*\delta}}\ne\varnothing$, $\forall\delta >0.$ Для любой точки $v_{\delta}\in\mathbf{W_{*\delta}}$ в силу \eqref{noise}-\eqref{Tikhonov-condition} имеем
\begin{equation}
\label{18}
\begin{aligned}
&\Phi(v_{\delta},v_{\delta})\leqslant \Phi(v_{\delta},v_{\delta})+\overbrace{AP(v_{\delta})}^{\geqslant 0}\stackrel{\eqref{noise}}{\leqslant}\\
&\leqslant\Phi_{\delta}(v_{\delta},v_{\delta})+AP_{\delta}(v_{\delta})+\delta(1+2\|v_{\delta}\|^2)+A\delta(1+\|v_\delta\|^2)\leqslant\\
&\leqslant\Phi_{\delta}(v_{\delta},v_{\delta})+AP_{\delta}(v_{\delta})\textcolor{blue}{-\alpha\|v_{\delta}-w\|^2}+\delta\|v_{\delta}\|^2(2+A)+\\
&\quad +\delta+A\delta\textcolor{blue}{+\alpha\|v_{\delta}-w\|^2}\stackrel{\eqref{Tikhonov-condition}}{\leqslant}\\
&\leqslant t_{\delta}(v_{\delta},v_{\delta})+\delta\|v_{\delta}\|^2(2+A)+\delta+A\delta+\alpha\|v_{\delta}-w\|^2\stackrel{\eqref{Tikhonov}\eqref{Tikhonov-condition}}{\leqslant}\\
&\leqslant \Phi_{\delta}(v_{\delta},w)+AP_{\delta}(w)\cancel{-\alpha\|v_{\delta}-w\|^2}+\varepsilon+\delta\|v_{\delta}\|^2(2+A)+\\
&\quad+\delta+A\delta\cancel{+\alpha\|v_{\delta}-w\|^2}\stackrel{\eqref{noise}}{\leqslant}\\
&\leqslant \Phi(v_{\delta},w)+AP(w)+\delta\|v_{\delta}\|^2(3+A)+(2+\|w\|^2)(\delta+A\delta)+\varepsilon,\forall w\in \mathbf{W_0}.
\end{aligned}
\end{equation}
Пусть $w=v_*$, где $v_*$ - точка, удовлетворяющая условию \eqref{7}. Продолжим неравенство: из \eqref{propsition-1}\eqref{15} и $P(v_*)=0$ получаем
\begin{equation}
\label{18-2}
\begin{aligned}
&\Phi(v_{\delta},v_*)+\overbrace{AP(v_*)}^{ P(v_*)=0}+\delta\|v_{\delta}\|^2(3+A)+(2+\|v_*\|^2)(\delta+A\delta)+\varepsilon\stackrel{\eqref{15}}{\leqslant}\\
&\leqslant \Phi(v_{\delta},v_{\delta})+\sum_{i=1}^{s}c_i[g_i^+(v_{\delta})]^v+\delta\|v_{\delta}\|^2(3+A)+(2+\|v_*\|^2)(\delta+A\delta)+\varepsilon\stackrel{\eqref{propsition-1}}{\leqslant}\\ 
&\leqslant \Phi(v_{\delta},v_{\delta})+AP(v_{\delta})+BA^{-v/(p-v)}+\delta\|v_{\delta}\|^2(3+A)+(2+\|v_*\|^2)(\delta+A\delta)+\varepsilon.
\end{aligned}
\end{equation}
Далее, выделяя из цепочки \eqref{18}\eqref{18-2} неравенств второе и предпоследнее, получаем
\begin{equation}
\label{22}
\begin{aligned}
&AP(v_{\delta})\leqslant\sum_{i=1}^{s}c_i[g_i^+(v_{\delta})]^v+\delta\|v_{\delta}\|^2(3+A)+(2+\|v_*\|^2)(\delta+A\delta)+\varepsilon=\\
&=\sum_{i=1}^{s}c_i[g_i^+(v_{\delta})]^v+\alpha\left[\|v_{\delta}\|^2\overbrace{\left(\frac{3\delta+\delta A}{\alpha} \right)}^{\leqslant \text{const}}+(2+\|v_*\|^2)\cdot \overbrace{\frac{\delta+\delta A}{\alpha}}^{\leqslant\text{const}}+\overbrace{\frac{\varepsilon}{\alpha}}^{\leqslant\text{const}} \right]\stackrel{\eqref{16}}{\leqslant}\\
&\leqslant \sum_{i=1}^{s}c_i[g_i^+(v_{\delta})]^v+\alpha C_1,
\end{aligned}
\end{equation}
где $C_1=\|v_{\delta}\|^2\sup\limits_{\delta>0}[3\delta+\delta A(\delta)/\alpha(\delta)]+(2+\|v_*\|^2)\sup\limits_{\delta>0}[\delta+\delta A(\delta)/\alpha(\delta)]+\sup\limits_{\delta>0}[\varepsilon/\alpha]\leqslant+\infty$ в силу \eqref{16}.\\
Если $p=v$, то 
\begin{equation}
\label{23}
\begin{aligned}
&AP(v_{\delta})\leqslant\sum_{i=1}^{s}c_i[g_i^+(v_{\delta})]^v+\alpha C_1\leqslant|c|_{\infty}\sum_{i=1}^{s}[g_i^+(v_{\delta})]^v+\alpha C_1=|c|_{\infty}P(v_\delta)+\alpha C_1\\
&\Rightarrow 0\leqslant AP(v_{\delta})\leqslant \frac{A}{A-|c|_{\infty}}\alpha C_1,\quad p=v.
\end{aligned}
\end{equation}
Если $p>v$, то используя неравенство Гёльдера
\begin{equation}
\sum_{i=1}^{s}a_ib_i\leqslant\left(\sum_{i=1}^{s}a_i^q \right)^{1/q}\left(\sum_{i=1}^{s}b_i^r \right)^{1/r},\quad \forall a_i,b_i\geqslant 0,r,q>1,\frac{1}{q}+\frac{1}{r}=1
\end{equation} 
при $q=p/v,r=p/(p-v),a_i=g_i^+(v_{\delta})^v,b_i=c_i$ с учетом \eqref{22} имеем
\begin{equation}
\begin{aligned}
&\sum_{i=1}^{s}[g_i^+(v_{\delta})]^v c_i=\left(\sum_{i=1}^{s}[g_i^+(v_{\delta})]^p \right)^{v/p}\cdot\left(\sum_{i=1}^{s}c_i^{p/(p-v)} \right)^{(p-v)/p}=|c|P(v_{\delta})^{v/p}\\
&\Rightarrow 0\leqslant AP(v_{\delta})\leqslant |c|P(v_{\delta})^{v/p}+\alpha C_1,\quad p>v.
\end{aligned}
\end{equation}
Теперь обозначим $z=[AP(v_{\delta})]^{v/p}$, имеем $0\leqslant z^{p/v}\leqslant |c|A^{-v/p}z+\alpha C_1$. Тогда в силу предложения \ref{lemma11} имеем
\begin{equation}
\label{24}
0\leqslant AP(v_{\delta})\leqslant z^{p/v}\leqslant |c|^{p/(p-v)}A^{-v/(p-v)}+\frac{p}{p-v}\alpha C_1.
\end{equation}
Тогда в силу \eqref{16}\eqref{23}\eqref{24} получаем
\begin{equation}
\label{25}
\begin{aligned}
&\lim_{\delta\rightarrow 0}A(\delta)P(v_{\delta})=0=\lim_{\delta\rightarrow 0}\overbrace{\frac{A}{A-|c|_{\infty}}}^{\rightarrow 1\text{ при }A\rightarrow\infty}\overbrace{\alpha C_1}^{\rightarrow 0},p=v;\\
&\lim_{\delta\rightarrow 0}A(\delta)P(v_{\delta})=0=\lim_{\delta\rightarrow 0}|c|^{p/(p-v)}\overbrace{A^{-v/(p-v)}}^{\rightarrow 0\text{ при }A\rightarrow\infty}+\overbrace{\frac{p}{p-v}\alpha C_1}^{\rightarrow 0},p>v
\end{aligned}
\end{equation}
Поскольку $\mathbf{W_0}$ - ограниченное множество, имеем $\sup\limits_{\delta>0}|v_{\delta}|<C_2$. Из теоремы Больцано-Вейерштрасса получаем, что семейство точек $\{v_{\delta} \}$ при $\delta\rightarrow 0$ имеет хотя бы одну предельную точку $v_0$. Пусть $\lim\limits_{k\rightarrow\infty}v_{\delta_k}=v_0$, где $\{\delta_k \}\rightarrow 0$ при $k\rightarrow \infty$. В силу замкнутости $\mathbf{W_0}$ имеем $v_0\in\mathbf{W_0}$, Из \eqref{16}\eqref{25} и полунепрерывности снизу функции $P(w)$ следует: $0\leqslant P(v_0)\leqslant \varliminf\limits_{k\rightarrow 0}P(v_{\delta_k})=\textcolor{blue}{\{\lim\limits_{\delta\rightarrow 0} A(\delta)P(v_{\delta})=0,\lim\limits_{\delta\rightarrow 0}A(\delta)\rightarrow\infty\}}=0$, поэтому $ P(v_0)=0,$ значит $v_0\in\mathbf{W}.$\\
Далее, выберем в \eqref{18} первое и последное звенья, следуя $P(w)=0,w\in\mathbf{W}$ и сделаем предельный переход при $\delta=\delta_k\rightarrow 0$:
\begin{equation}
\label{varlim}
\begin{aligned}
&\Phi(v_{\delta},v_{\delta})\leqslant \Phi(v_{\delta},w)+\overbrace{AP(w)}^{=0}+2\overbrace{\delta\|v_{\delta}\|^2(2+A)}^{\eqref{16},\rightarrow 0}+2\overbrace{(\delta+A\delta)}^{\eqref{16},\rightarrow 0}+\overbrace{\varepsilon}^{\rightarrow 0}\Rightarrow\\
&\Rightarrow\varlimsup\limits_{k\rightarrow\infty}\Phi(v_{\delta},v_{\delta})\leqslant \varlimsup\limits_{k\rightarrow\infty}\Phi(v_{\delta},w)
\end{aligned}
\end{equation} 
Поэтому в силу свойств функции $\Phi(v,w)$ имеем
\begin{equation}
\begin{aligned}
&\Phi(v_0,v_0)\overbrace{\leqslant}^{\text{п/н снизу.}}\varliminf\limits_{k\rightarrow\infty}\Phi(v_{\delta_k},v_{\delta_k})\leqslant\varlimsup\limits_{k\rightarrow\infty}\Phi(v_{\delta_k},v_{\delta_k})\stackrel{\eqref{varlim}}{\leqslant}\\
&\leqslant\varlimsup\limits_{k\rightarrow\infty}\Phi(v_{\delta_k},w)\overbrace{\leqslant}^{\text{п/н сверху.}}\Phi(v_0,w),\quad\forall w\in\mathbf{W}.
\end{aligned}
\end{equation}
Таким образом, каждая предельная точка $v_0$ семейства $\{v_{\delta} \}$ при $\delta\rightarrow 0$ удовлетворяет условиям $v_0\in\mathbf{W_*},\Phi(v_0,v_0)\in\Phi_*$. Равенства \eqref{17} доказаны.\\
Остается убедиться в том, что сходимости в \eqref{17} равномерны относительно $\Phi_{\delta}(v,w),P_{\delta}(w)$ из \eqref{noise}, $v_{\delta}$ из $\mathbf{W_{*\delta}}$. Для этого надо показать, что 
\begin{equation}
\label{26}
\lim_{\delta\rightarrow 0}\sup_{v\in\mathbf{W_{*\delta}}}\rho(v,\mathbf{W_*})=0,\quad\lim_{\delta\rightarrow 0}\sup_{v\in\mathbf{W_{*\delta}}}\rho(\Phi(v,v),\Phi_*)=0.
\end{equation}
Пусть $\{\delta_k\}$ - выбрана подпоследовательность, на которой достигается верхний предел, то есть
\begin{equation}
\label{def}
\varlimsup_{\delta\rightarrow 0}\sup_{v\in\mathbf{W_{*\delta}}}\rho(v,\mathbf{W_*})=\lim_{k\rightarrow\infty} \sup_{v\in\mathbf{W_{*\delta}}}\rho(\Phi(v,v),\Phi_*).
\end{equation}
По определению верхней грани, для любого номера $k$ найдётся точка $v_{\delta_k}\in\mathbf{W_{*\delta_k}}$, соответствующая какой-то реализации $\Phi_{\delta}(v,w),P_{\delta}(w)$ из \eqref{noise} при $\delta=\delta_k$ такая, что
\begin{equation}
\label{27}
\sup_{v\in\mathbf{W_{*\delta}}}\rho(v,\mathbf{W_*})\leqslant\rho(v_{\delta_k},\mathbf{W_*})+\frac{1}{k},\quad k=1,2,\ldots
\end{equation}  
Из \eqref{17} при $\delta=\delta_k$ имеем $\lim\limits_{\delta\rightarrow 0}\rho(v_{\delta_k},\mathbf{W_*})=0$. После этого в \eqref{27} сделаем предельный переход при $k\rightarrow \infty$ с учётом свойства предела и \eqref{def}, следует
\begin{equation}
0\leqslant \varliminf_{\delta\rightarrow 0}\sup_{v\in\mathbf{W_{*\delta}}}\rho(v,\mathbf{W_*})\leqslant\varlimsup_{\delta\rightarrow 0}\sup_{v\in\mathbf{W_{*\delta}}}\rho(v,\mathbf{W_*})\leqslant\lim_{\delta\rightarrow 0}\sup_{v\in\mathbf{W_{*\delta}}}\rho(v,\mathbf{W_*})=0
\end{equation}
что равносильно первому равенству \eqref{26}, поскольку верхний и нижний предел равны нулю. Второе равенство \eqref{26} устанавливается аналогично. Теорема \ref{theor-1} доказана. $\qedsymbol$

Итак, метод стабилизации с использованием стабилизатора $-\alpha_k\|v-w\|^2$ формально описан и доказана его сходимость. 
\clearpage
%\newpage
	\section{Регуляризованный экстраградиентный метод}
В прошлой главе был рассмотрен метод стабилизации, но не рассматривался конкретный метод поиска точек равновесия функции Тихонова. В этой главе исследуем один из конкретных методов - \textit{экстраградиентный метод.}

Рассмотрим следующий итеративный процесс
\begin{equation}
\label{[3]-7}
\begin{aligned}
&u_k=\mathrm{Pr}_{\mathbf{W_0}}(v_k-\beta_k[\nabla_w^k\Phi(v_k,v_k)+A_k\nabla_w^kP(v_k)]),\\
v_{k+1}=\mathrm{Pr}_{\mathbf{W_0}}&(v_k-\beta_k[\nabla_w^k\Phi(v_k,u_k)+A_k\nabla_w^kP(u_k)-2\alpha_k(u_k-v_k)]),\quad k=0,1,...
\end{aligned}
\end{equation}

В целом он похож на градиентный метод прогнозного типа, который был в обзоре существующих методов, только вместо исходной функции взято её приближение, и добавлены штрафный член и производная стабилизатора. Введем функцию Тихонова 
\begin{equation}
\label{new-Tikhonov}
\begin{aligned}
T_k(v,w)&=\Phi(v,w)+A_kP(w)-\alpha_k\|v-w\|^2,\quad v,w\in\mathbf{W_0},\\
& A_k>0,\alpha_k>0,\quad k=0,1,...
\end{aligned}
\end{equation}

Предполагается, что функции $\Phi(v,w),g_i(w),i=1,..., s$ - дифференцируемы на $\mathbf{W_0}$ и градиент функции \eqref{new-Tikhonov} по $w$ существует и равен
\begin{equation}
\label{[3]-5}
\begin{aligned}
&\nabla_wT_k(v,w)=\nabla_w\Phi(v,w)+A_k\nabla_wP(w)-2\alpha(w-v),\\
&v,w\in\mathbf{W_0},A_k>0,\alpha_k>0,k=0,1,...
\end{aligned}
\end{equation}

Пусть вместо точных значений $\nabla_w\Phi(v,w),\nabla_wP(w)$ известны последовательности приближений $\{\nabla_w^k\Phi(v,w)\},\{\nabla_w^kP(w)\}$ такие, что
\begin{equation}
\label{[3]-6}
\begin{aligned}
&\|\nabla_w^k\Phi(v,w)-\nabla_w\Phi(v,w)\|\leqslant \delta_k(1+\|v\|+\|w\|),\quad \forall v,w\in \mathbf{W_0},\\
&\|\nabla_w^kP(w)-\nabla_wP(w)\|\leqslant \delta_k(1+\|w\|),\quad\forall w\in \mathbf{W_0},\delta_k>0,k=0,1,... 
\end{aligned}
\end{equation}

\subsection{Вспомогательные теоремы}
Была предложена следующая теорема о выпуклости и замкнутости множества решений $\mathbf{W_{*}}$. В ней также установлена единственность нормального решения. 
\begin{theo}
	\label{thero-2-1}
	Пусть выполнены следующие условия\\
	\textbf{1)} $\mathbf{W}_0$ - выпуклое замкнутое множество из $\mathbb{E}^n$,\\
	\textbf{2)} функция $\Phi(v,w)$ непрерывна по совокупности переменных $(v,w)\in\mathbf{W_0}\times\mathbf{W_0}$, выпукла по переменной $w$ на $\mathbf{W_0}$ при каждом фиксированном $v\in\mathbf{W_0}$, удовлетворяет условию кососимметричности \eqref{cososymmetric}. \\
	\textbf{3)} Функции $g_i(w)$ при $i=1,...,m$ непрерывны, выпуклы на $\mathbf{W_0}$; функция $g_i(w)$ при $i=m+1,...,s$ аффинны, т.е. представимы в виде $g_i(w)=\langle a_i,w \rangle-b_i,a_i\in\mathbb{E}^n,b_i\in\mathbb{E}^1$. \\
	\textbf{4)} Пусть множество $\mathbf{W_*}$ решений задачи \eqref{question} непусто. \\
	Тогда $\mathbf{W_*}$ выпукло, замкнуто и задача \eqref{question} имеет единственное решение $v_*$, имеющее минимальную среди всех решений норму и называемое нормальным решением задачи \eqref{question}.
\end{theo}
\noindent\emph{Доказатеьство.} см. \cite{centralbib-2} cтр. 3. $\quad\qedsymbol$

Далее изучим поведение последовательности $\{z_k \}$ точек равновесия  функции \eqref{new-Tikhonov}, определяемых условием 
\begin{equation}
\label{new-Tikhonov-condition}
z_k\in\mathbf{W_0},\quad T_k(z_k,z_k)\leqslant T_k(z_k,w)\quad \forall w\in\mathbf{W_0},\quad k=0,1,...
\end{equation}
Подобные результаты приведены в следующей теореме, её доказательство использует факты предыдущей теоремы.
\begin{theo}
	\label{thero-2-2}
	Пусть выполнены условия теоремы \ref{thero-2-1}, и\\
	\textbf{1)} функции $\Phi(v,w),g_i(w),i=1,...,m$, имеют непрерывные градиенты $\nabla_w\Phi(v,w)$, $\nabla_w g_i(w)$ на $\mathbf{W_0}$; \\
	\textbf{2)} функция Тихонова \eqref{new-Tikhonov} выпукла по $w$ при каждом $k=0,1,...$; \\
	\textbf{3)} существуют постоянные $v>0,c_i\geqslant 0,i=1,...,s$, такие, что 
	\begin{equation}
	\label{norm-condition}
	\Phi(v_*,v_*)\leqslant \Phi(v_*,w)+\sum_{i=1}^{s}c_i(g_i^+(w))^v,\quad \forall w\in\mathbf{W_0},
	\end{equation}
	где $v_*$ - нормальное решение задачи \eqref{question};\\
	\textbf{4)} параметр $p$ из штрафной функции \eqref{penalty} удовлетворяет условиям $p\geqslant v,p>1$; последовательности $\{\alpha_k \},\{A_k \}$ таковы, что
	\begin{equation}
	\label{parameter-condition-2}
	\begin{aligned}
	&\alpha_k>0,\quad A_k>0,\quad k=0,1,...,\quad\\
	&\lim\limits_{k\rightarrow\infty}\alpha_k=0,\quad \lim_{k\rightarrow\infty}A_k=+\infty,\quad\lim_{k\rightarrow\infty}\alpha _kA_k^{v/(p-v)}=+\infty.
	\end{aligned}
	\end{equation}
	(при $p=v$ последнее условие не нужно).
	
	Тогда точки $z_k$, удовлетворяющие условиям \eqref{new-Tikhonov-condition}, существуют, однозначно определяются при каждом $k=0,1...$ и таковы, что
	\begin{equation}
	\label{[3]-12}
	\begin{aligned}
	\|z_k\|\leqslant R_k&\leqslant \sup\limits_{k\geqslant 0}R_k=R,\quad k=0,1,...,\\
	R_k=\left(\frac{B}{\alpha_k A_k^{v/(p-v)}} \right)^{1/2}+\|v_*\|&,\quad B=(p-v)v^{v/(p-v)}p^{-p/(p-v)}|c|^{p/(p-v)},\\
	|c|=\left(\sum\limits_{i=1}^{s}c_i^{p/(p-v)} \right)^{(p-v)/p}& \text{при } p>v,\quad R_k=\|v_*\| \text{при } p=v,
	\end{aligned}
	\end{equation}
	\begin{equation}
	\label{[3]-13}
	\lim_{k\rightarrow\infty}A_kP(z_k)=0,
	\end{equation}
	\begin{equation}
	\label{[3]-14}
	\lim_{k\rightarrow\infty}\|z_k-v_*\|=0,
	\end{equation}
	\begin{equation}
	\label{[3]-15}
	\begin{aligned}
	\|z_k-z_m\|&\leqslant \frac{|A_m-A_k|R_1}{2\alpha_k},\quad \forall k,m=0,1,...\\
	R_1&=\max\limits_{\|w\|\leqslant R}\|\nabla_w P(w)\|.
	\end{aligned}
	\end{equation}
\end{theo}
\begin{remark}
Содержательные классы задач, для которых выполнено условие \eqref{norm-condition}, приведены в \cite{centralbib}.
\end{remark}
\noindent\emph{Доказательство.} Из определения \eqref{new-Tikhonov} функции $T_k(v,w)$ и условия кососимметричности \eqref{cososymmetric} следует, что
\begin{equation}
\label{[3]-16}
T_k(v,v)-T_k(v,w)-T_k(w,v)+T_k(w,w)\geqslant 2\alpha_k\|v-w\|^2,\quad \forall v,w\in\mathbf{W_0}.
\end{equation}
Кроме того, функция $T_k(v,w)$ непрерывна по $(v,w)\in\mathbf{W_0}\times\mathbf{W_0}$, выпукла и непрерывно дифференцируема по $w$ на $\mathbf{W_0}$. Тогда и из \cite{4} следует, что условие \eqref{new-Tikhonov-condition} однозначно определяет точку $z_k$ при каждом $k=0,1,...$.

Докажем оценку \eqref{[3]-12}. Заметим, что из \eqref{cososymmetric} при $v=v_*$ и \eqref{norm-condition} вытекают неравенства
\begin{equation}
\label{[3]-17}
\Phi(w,v_*)-\Phi(w,w)\leqslant \Phi(v_*,v_*)-\Phi(v_*,w)\leqslant \sum_{i=1}^{s}c_i(g_i^+(w))^v,\quad \forall w\in\mathbf{W_0}.
\end{equation} 
Положив в \eqref{[3]-16} $v=z_k,w=v_*$, c учетом \eqref{new-Tikhonov}\eqref{new-Tikhonov-condition}\eqref{[3]-17}, $P(v_*)=0$ и предложения \ref{propsition-1} получим
\begin{equation}
\label{[3]-19}
\begin{aligned}
&2\alpha_k\|z_k-v_*\|^2\leqslant \overbrace{T_k(z_k,z_k)-T_k(z_k,v_*)}^{\leqslant 0\text{ в силу }\eqref{new-Tikhonov-condition}}-T_k(v_*,z_k)+T_k(v_*,v_*)\stackrel{\eqref{new-Tikhonov}}{\leqslant}\\
&\leqslant\Phi(v_*,v_*)-\Phi(v_*,z_k)+\underbrace{A_kP(v_*)}_{=0}-A_kP(z_k)\underbrace{-\alpha_k\|v_*-v_*\|^2}_{=0}+\alpha_k\|v_*-z_k\|^2\stackrel{\eqref{[3]-17}}{\leqslant}\\
&\leqslant \sum_{i=1}^{s}c_i(g_i^+(w))^v-A_kP(z_k)+\alpha_k\|v_*-z_k\|^2\stackrel{\text{Предлож. 1}}{\leqslant}\\
&\leqslant \cancel{A_kP(z_k)}+BA_k^{-v/(p-v)}\cancel{-A_kP(z_k)}+\alpha_k\|v_*-z_k\|^2=BA_k^{-v/(p-v)}+\alpha_k\|v_*-z_k\|^2
\end{aligned}
\end{equation}
Отсюда, получим
\begin{equation}
\begin{aligned}
&\alpha_k\|z_k-v_*\|^2\leqslant BA_k^{-v/(p-v)},\quad \|z_k-v_*\|\leqslant\left(\frac{B}{\alpha_kA_k^{v/(p-v)}}\right)^{1/2},\\
&\|z_k\|=\|z_k-v_*+v_*\|\leqslant\|z_k-v_*\|+\|v_*\|\leqslant \left(\frac{B}{\alpha_kA_k^{v/(p-v)}}\right)^{1/2}+\|v_*\|=R_k
\end{aligned}
\end{equation} 
Это равносильно \eqref{[3]-12} при $p>v$. Аналогично доказывается и при $p=v$.

Докажем \eqref{[3]-13}. Из \eqref{[3]-19} выделяем первое и четвертое выражения, получаем 
\begin{equation}
\label{[3]-20}
A_kP(z_k)\leqslant\sum_{i=1}^{s}c_i(g_i^+(w))^v-\alpha_k\|v_*-z_k\|^2
\end{equation} 
Если $p>v$, то пользуясь неравенством Гёльдера
\begin{equation*}
\sum_{i=1}^{s}a_ib_i\leqslant\left(\sum_{i=1}^{s}a_i^q \right)^{1/q}\left(\sum_{i=1}^{s}b_i^r \right)^{1/r},\quad \forall a_i,b_i\geqslant 0,r,q>1,\frac{1}{q}+\frac{1}{r}=1
\end{equation*} 
при $q=p/v,r=p/(p-v),a_i=g_i^+(z_k)^v,b_i=c_i$ имеем
\begin{equation*}
\begin{aligned}
&\sum_{i=1}^{s}[g_i^+(v_{\delta})]^v c_i=\left(\sum_{i=1}^{s}[g_i^+(z_k)]^p \right)^{v/p}\cdot\left(\sum_{i=1}^{s}c_i^{p/(p-v)} \right)^{(p-v)/p}=|c|(P(z_k))^{v/p}\\
&\Rightarrow 0\leqslant AP(z_k)\leqslant |c|(P(v_{\delta}))^{v/p}-\alpha\|v_*-z_k\|^2\leqslant|c|(P(v_{\delta}))^{v/p},\quad p>v.
\end{aligned}
\end{equation*}
Пусть $x=[A_kP(z_k)]^{v/p}$, тогда можно переписать предыдущее неравенство в виде
\begin{equation*}
0\leqslant x^{p/v}\leqslant|c|A_k^{-v/p}x
\end{equation*} 
Отсюда в силу предложения \ref{lemma11} получаем
\begin{equation}
\label{[3]-21}
0\leqslant A_kP(z_k)=x^{p/v}\leqslant|c|^{p/(p-v)}A_k^{-v/(p-v)}.
\end{equation}
Если $p=v$, то из \eqref{[3]-20} сразу находим $A_kP(z_k)\leqslant |c|_{\infty}P(z_k)-\alpha_k\|v_*-z_k\|^2\leqslant |c|_{\infty}P(z_k)$ для всех $k>k_0$, когда $A_k\geqslant |c|_{\infty}$, и
\begin{equation}
\label{[3]-22}
0\leqslant A_kP(z_k)\leqslant 0\quad \Rightarrow\quad A_kP(z_k)=0
\end{equation}
Из \eqref{[3]-21}\eqref{[3]-22} с учетом \eqref{parameter-condition-2} получим
\begin{equation*}
\lim_{k\rightarrow\infty}A_kP(z_k)=0=\lim_{k\rightarrow\infty}|c|^{p/(p-v)}\overbrace{A_k^{-v/(p-v)}}^{\rightarrow 0, A_k\rightarrow+\infty}\quad p>v,\quad\lim_{k\rightarrow\infty}A_kP(z_k)=0,\quad p=v
\end{equation*}

Докажем равенство \eqref{[3]-14}. Из \eqref{[3]-13} получаем $\lim\limits_{k\rightarrow\infty}P(z_k)=0$. В силу \eqref{penalty} получаем $\lim\limits_{k\rightarrow\infty}g_i^+(z_k)=0,i=1,...,s$, или 
\begin{equation}\label{[3]-23}
\varlimsup_{k\rightarrow \infty}g_i(z_k)\leqslant 0,\quad i=1,...,m,\quad \lim_{k\rightarrow\infty}g_i(z_k)=0,\quad i=m+1,...,s.
\end{equation}
Из оценки \eqref{[3]-12} и теоремы Больцано-Вейерштрасса вытекает, что последовательность $\{z_k\}$ имеет хотя бы одну предельную точку $u_*$(или сходящуюся подпоследовательность $\{z_{k_l}\}\rightarrow u_*$). Так как $\{z_k\}\in\mathbf{W_0}$ и $\mathbf{W_0}$ замкнуто, то $u_*\in\mathbf{W_0}$. Из \eqref{[3]-23} и непрерывности $g_i(w)$ следует, что $g_i(u_*)\leqslant 0,i=1,...,m,g_i(u_*)=0,i=m+1,...,s$. Следовательно, $u_*\in\mathbf{W}$. Теперь можно в \eqref{new-Tikhonov-condition} сделать предельный переход при $k=k_l\rightarrow\infty$:
\begin{equation}
\begin{aligned}
&T_k(z_k,z_k)\leqslant T_k(z_k,w)\quad\Leftrightarrow\quad\Phi(z_k,z_k)\leqslant\Phi(z_k,w)+A_kP(w)-A_kP(z_k)\Rightarrow\\
&\Rightarrow\{k\rightarrow\infty\}\Rightarrow\quad \Phi(u_*,u_*)\leqslant\Phi(u_*,w)+\lim\limits_{k\rightarrow\infty}A_k\underbrace{P(w)}_{=0,\forall w\in \mathbf{W}}-\underbrace{\lim\limits_{k\rightarrow\infty}A_kP(z_k)}_{=0,\eqref{[3]-13}}.
\end{aligned}
\end{equation}
Итак, $\Phi(u_*,u_*)\leqslant\Phi(u_*,w)\,\forall w\in\mathbf{W},$ т.е. $u_*\in\mathbf{W_*}$. Из оценки \eqref{[3]-12} при $k=k_l\rightarrow\infty$ следует $\|u_*\|\leqslant\|v_*\|$. Но $v_*$ - нормальное решение задачи \eqref{question}, т.е. $\|u_*\|\geqslant\|v_*\|$. Следовательно, $\|u_*\|=\|v_*\|$, то есть $u_*$ - нормальное решение этой задачи. Поскольку нормальное решение единственно, то $u_*=v_*$. Это означает, что $\{z_k\}$ имеет единственную предельную точку $v_*$, отсюда следует $\lim\limits_{k\rightarrow\infty}\|z_k-v_*\|=0$.

Наконец, докажем \eqref{[3]-15}. Так как функция $\Phi(v,w)$ выпукла и дифференцируема по переменной $w$ на $\mathbf{W_0}$, то в силу критерия выпуклости [см. \cite{5} стр. 160] имеем
\begin{equation*}
\Phi(v,v)-\Phi(v,w)\leqslant\langle\nabla_w\Phi(v,v),v-w\rangle\quad \Phi(w,w)-\Phi(w,v)\leqslant\langle\nabla_w\Phi(w,w),w-v\rangle
\end{equation*}
Сложим эти два неравенства и в силу \eqref{cososymmetric} получим
\begin{equation}
\label{[3]-24}
\begin{aligned}
0\leqslant& \,\Phi(v,v)-\Phi(v,w)-\Phi(w,v)+\Phi(w,w)\leqslant\\
\leqslant& \,\langle\nabla_w\Phi(v,v)-\nabla_w\Phi(w,w),v-w\rangle,\quad \forall v,w\in\mathbf{W_0}
\end{aligned}
\end{equation} 
Аналогично, из выпуклости и дифференцируемости $P(w)$ на $\mathbf{W_0}$ имеем
\begin{equation}
\label{[3]-25}
0\leqslant \langle\nabla_wP(v)-\nabla_wP(w),v-w\rangle
\end{equation}
Функция $T_k(v,w)$ - выпукла и дифференцируема по $w$ на $\mathbf{W_0}$, поэтому из \eqref{[3]-16}, взяв $v=z_k,w=z_m$, находим
\begin{equation}
\label{[3]-27}
2\alpha_k\|z_k-z_m\|^2\leqslant\langle\nabla_wT_k(z_k,z_k)-\nabla_wT_k(z_m,z_m),z_k-z_m\rangle.
\end{equation}
Кроме того, из \eqref{new-Tikhonov-condition} в силу критерия оптимальности [см. \cite{5} стр. 161] имеем $\forall w\in\mathbf{W_0}$
\begin{equation}
\label{[3]-28}
\begin{aligned}
&0\leqslant \langle\nabla_wT_k(z_k,z_k),w-z_k\rangle\\
&0\leqslant \langle\nabla_wT_m(z_m,z_m),w-z_m\rangle
\end{aligned}
\end{equation}
В первом из неравенств\eqref{[3]-28} положим $w=z_m$, во втором $w=z_k$ и почленно сложим с \eqref{[3]-27}. С учетом \eqref{[3]-5}  и неравенства Коши-Буняковского получим
\begin{equation*}
\begin{aligned}
&2\alpha_k\|z_k-z_m\|^2\leqslant\\
&\leqslant\langle\cancel{\nabla_wT_k(z_k,z_k)}-\nabla_wT_k(z_m,z_m)\cancel{-\nabla_wT_k(z_k,z_k)}+\nabla_wT_m(z_m,z_m),z_k-z_m\rangle\stackrel{\eqref{[3]-5}}{\leqslant}\\
&\leqslant\langle\cancel{\nabla_w\Phi(z_m,z_m)}+A_m\nabla_wP(z_m)\cancel{-\nabla_w\Phi(z_k,z_k)}-A_k\nabla_wP(z_m),z_k-z_m\rangle\stackrel{\text{нер-во К-Б-}}{\leqslant}\\
&\leqslant|A_m-A_k|R_1\cdot \|z_k-z_m\|
\end{aligned}
\end{equation*}
Отсюда следует неравенство \eqref{[3]-15}. Теорема \ref{thero-2-2} доказана. $\qedsymbol$
\subsection{Описание регуляризованного экстраградиентного метода с использованием нового стабилизатора}
\noindent Приступим к исследованию сходимости итерационного метода \eqref{[3]-7}. Справедлива
\begin{theo}
	\label{thero-2-3}
	Пусть выполнены все условия теорем \ref{thero-2-1},\ref{thero-2-2}, и \\
	\textbf{1)} пусть градиенты $\nabla_w\Phi(v,w),\nabla_wP(w)$ удовлетворяют условию Липшища:
	\begin{equation}
	\label{[3]-29}
	\begin{aligned}
	&\max\{ \|\nabla_w\Phi(v,v)-\nabla_w\Phi(w,w)\|,\|\nabla_wP(v)-\nabla_wP(w)\|\}\leqslant L\|v-w\|,\\
	&\forall v,w\in\mathbf{W_0},L=\mathrm{const}>0.
	\end{aligned}
	\end{equation}
	Пусть также выполнено модифицированное условие Липшица вида:
	\begin{equation}
	\label{[3]-29-2}
	\|\nabla_w\Phi(v,w)-\nabla_w\Phi(w,w)\|\leqslant L\|v-w\|,\quad \forall v,w\in\mathbf{W_0},L=\mathrm{const}>0.
	\end{equation}
	\textbf{2)} Вместо точного значения градиентов $\nabla_w\Phi(v,w),\nabla_wP(w)$ известны последовательности их приближений $\{\nabla_w^k\Phi(v,w)\},\{\nabla_w^kP(w) \}$, удовлетворяющие условиям \eqref{[3]-6}.\\
	\textbf{3)} Параметры $\{\alpha_k\},\{\beta_k\},\{\delta_k\},\{A_k\}$ метода \eqref{[3]-7} таковы, что \\
	\begin{equation}
	\label{[3]-30}
	\begin{aligned}
	&\alpha_k >0,A_{k+1}\geqslant A_k>0,\beta_k>0,\delta_k>0,\lim_{k\to \infty}\beta_k=0,\lim_{k\to\infty}\delta_k=0,\\
	&\sup_{k\geqslant 0}\beta_k(1+A_k)<\frac{1}{L},\lim_{k\to\infty}\frac{\delta_k+\delta_kA_k}{\alpha_k}=0,\lim_{k\to \infty}\frac{A_{k+1}-A_k}{\alpha_k^2\beta_k}=0.
	\end{aligned}
	\end{equation}
	Тогда последовательность $\{v_k\}$, порожденная методом \eqref{[3]-7} при любом выборе начального приближения $v_0\in \mathbf{W_0}$, сходится к нормальному решению $v_*$ задачи \eqref{question}, т.е.
	\begin{equation}
	\label{convergence}
	\lim_{k\to \infty}\|v_k-v_*\|=0
	\end{equation}
	причем сходимость в \eqref{convergence} равномерна относительно выбора $\{\nabla_w^k\Phi(v,w)\}$,$\{\nabla_w^kP(w) \}$ из \eqref{[3]-6}.
\end{theo}
\begin{remark}
\label{remark 3-2}
В качестве последовательностей параметров $\{\alpha_k\}$,$\{\beta_k\}$,$\{\delta_k\}$,$\{A_k\}$, выполняющих условиям \eqref{[3]-30}, можно, например, взять
\begin{equation*}
	\alpha_k = (k+1)^{-\alpha},A_k=(k+1)^A,\delta_k=(k+1)^{-\delta},\beta_k=\frac{1}{2L(1+A_k)}
\end{equation*}
где $\alpha,A,\delta$ - положительные числа. % и $\alpha+A<\min\{\frac{1}{2},\delta \},\alpha<A^{v/(p-v)}$ (при $p=v$ последнее неравенство не нужно).
\end{remark}
\noindent\emph{Доказательство}. Справедливо неравенство
\begin{equation}
\label{[3]-33}
\|v_k-v_*\|\leqslant \|v_k-z_k\|+\|z_k-v_*\|,\quad k=0,1,...
\end{equation}
где $v_*$ - нормальное решение задачи \eqref{question}, $z_k$ - решение задачи \eqref{new-Tikhonov-condition}. Из \eqref{[3]-14},\eqref{[3]-33} следует, что для доказательства теоремы \ref{thero-2-3} достаточно установить, что величины $b_k=\|v_k-z_k\|\to 0$ при $k\to \infty$.\\
Покажем, что величины $b_k$ удовлетворяют неравенствам
\begin{equation}
\label{[3]-34}
\begin{aligned}
&b_{k+1}^2\leqslant (1-s_k)b_k^2+d_k,\\
&k\geqslant k_0,0<s_k\leqslant 1,d_k\geqslant 0,\forall k\geqslant k_0,\sum_{k=0}^{\infty}s_k=+\infty,\lim_{k\to \infty}\frac{d_k}{s_k}=0,
\end{aligned}
\end{equation}
где $k_0$ - достаточно большое натуральное число. Заметим, что 
\begin{equation}
\label{[3]-36}
b_{k+1}=\|v_{k+1}-z_{k+1}\|\leqslant \|v_{k+1}-z_k\|+\|z_k-z_{k+1}\|,\quad k=0,1,...
\end{equation}
Для оценки величины $\|z_k-z_{k+1}\|$ имеем неравенство \eqref{[3]-15} при $m=k+1$. Рассмотрим первое слагаемое из \eqref{[3]-36}. Как известно, (см. \cite{5} стр. 183), $\mathrm{Pr}_{\mathbf{W_0}}(x)=a$ тогда и только тогда, когда 
\begin{equation*}
\langle a-x,w-a \rangle\geqslant 0,\quad \forall w\in\mathbf{W_0}.
\end{equation*}
Пользуясь этим свойством проекции, перепишем метод \eqref{[3]-7} в эквивалентной форме:
\begin{equation}
\label{[3]-37}
\big\langle u_k-v_k+\beta_k(\nabla_w^k\Phi(v_k,v_k)+A_k\nabla_w^kP(v_k)),w-u_k\big\rangle\geqslant 0,\quad \forall w\in\mathbf{W_0}.
\end{equation}
\begin{equation}
\label{[3]-38}
\big\langle v_{k+1}-v_k+\beta_k(\nabla_w^k\Phi(v_k,u_k)+A_k\nabla_w^kP(u_k)-2\alpha_k(u_k-v_k)),w-v_{k+1}\big\rangle\geqslant 0, \forall w\in\mathbf{W_0}.
\end{equation}
Подставим $w=v_{k+1}$ в \eqref{[3]-37} и $w=z_k$ в \eqref{[3]-38}, получившиеся неравенства сложим:
\begin{equation}
\label{[3]-39}
\begin{aligned}
&\langle u_k-v_k,v_{k+1}-u_k\rangle +\langle v_{k+1}-v_k,z_k-v_{k+1} \rangle +\\
&+\beta_k\langle \nabla_w^k\Phi(v_k,v_k)\textcolor{blue}{-\nabla_w^k\Phi(u_k,u_k)}+A_k(\nabla_w^kP(v_k)\textcolor{blue}{-\nabla_w^kP(u_k)}),v_{k+1}-u_k\rangle +\\
&+\beta_k\langle \textcolor{blue}{\nabla_w^k\Phi(u_k,u_k)}+A_k\textcolor{blue}{\nabla_w^kP(u_k)},z_k-u_k\rangle+\\
&+\beta_k\langle \nabla_w^k\Phi(v_k,u_k)-\nabla_w^k\Phi(u_k,u_k),z_k-v_{k+1} \rangle-\\
&-2\alpha_k\beta_k\langle u_k-v_k,z_k-v_{k+1}\rangle\geqslant 0.
\end{aligned}
\end{equation}
Справедливо равенство
\begin{equation*}
2\big(\langle u_k-v_k,v_{k+1}-u_k \rangle+\langle v_{k+1}-v_{k},z_k-v_{k+1}\rangle \big)=\|v_k-z_k\|^2-\|v_k-u_k\|^2-\|v_{k+1}-z_k\|^2-\|v_{k+1}-u_k\|^2.
\end{equation*}
Отсюда из \eqref{[3]-39} получаем
\begin{equation}
\label{[3]-40}
\begin{aligned}
&\|v_{k+1}-z_k\|^2\leqslant \|v_k-z_k\|^2-\|v_k-u_k\|^2-\|v_{k+1}-u_k\|^2+\\
&+2\beta_k\big\langle [\nabla_w^k\Phi(v_k,v_k)-\nabla_w\Phi(v_k,v_k)]+A_k[\nabla_w^kP(v_k)-\nabla_wP(v_k)],v_{k+1}-u_k\big\rangle+\\
&+2\beta_k\big\langle [\nabla_w\Phi(v_k,v_k)-\nabla_w\Phi(u_k,u_k)]+A_k[\nabla_wP(v_k)-\nabla_wP(u_k)],v_{k+1}-u_k\big\rangle+\\
&+2\beta_k\big\langle [\nabla_w\Phi(u_k,u_k)-\nabla_w^k\Phi(u_k,u_k)]+A_k[\nabla_wP(u_k)-\nabla_w^kP(u_k)],v_{k+1}-u_k\big\rangle+\\
&+2\beta_k\big\langle [\nabla_w^k\Phi(u_k,u_k)-\nabla_w\Phi(u_k,u_k)]+A_k[\nabla_w^kP(u_k)-\nabla_wP(u_k)],z_k-u_k\big\rangle+\\
&+2\beta_k\big\langle [\nabla_w\Phi(u_k,u_k)-\nabla_w\Phi(z_k,z_k)]+A_k[\nabla_wP(u_k)-\nabla_wP(z_k)],z_k-u_k\big\rangle+\\
&+2\beta_k\big\langle \nabla_w\Phi(z_k,z_k)+A_k\nabla_wP(z_k),z_k-u_k\big\rangle+\\
&+2\beta_k\langle \nabla_w^k\Phi(v_k,u_k)-\nabla_w^k\Phi(u_k,u_k),z_k-v_{k+1} \rangle+4\alpha_k\beta_k\langle u_k-v_k,v_{k+1}-z_k\rangle.
\end{aligned}
\end{equation}
Далее оценим слагаемые из них. Сначала заметим, что справедливы следующие соотношения:
\begin{equation}
\label{2|ab|}
2|ab|\leqslant \varepsilon a^2+b^2/\varepsilon,\forall\varepsilon>0.
\end{equation}
\begin{equation}
\label{An<=Qn}
\frac{(x_1+...+x_n)^2}{n}\leqslant x_1^2+\ldots+x_n^2.
\end{equation}
Для четвертого слагаемого с учетом \eqref{[3]-6} и \eqref{[3]-12} получаем
\begin{equation}
\label{[3]-41}
\begin{aligned}
&2\beta_k\big\langle [\nabla_w^k\Phi(v_k,v_k)-\nabla_w\Phi(v_k,v_k)]+A_k[\nabla_w^kP(v_k)-\nabla_wP(v_k)],v_{k+1}-u_k\big\rangle\leqslant \\
&\leqslant 2\beta_k\big\langle \delta_k(1+2\|v_k\|)+A_k\delta_k\|v_k\|,v_{k+1}-u_k\big\rangle\leqslant \\
&\leqslant 2\beta_k\big\langle (\delta_k+A_k\delta_k)(1+2\|v_k\|),v_{k+1}-u_k\big\rangle\leqslant\\
&\leqslant 2\beta_k(\delta_k+\delta_kA_k)(1+2\|v_k\|)\|v_{k+1}-u_k\|\leqslant\\
&\leqslant\beta_k(\delta_k+\delta_kA_k)2\cdot(1+2\|v_k-z_k\|+\|z_k\|)\|v_{k+1}-u_k\|\leqslant\\
&\leqslant\beta_k(\delta_k+\delta_kA_k)(\frac{1}{2}(1+2R+2\|v_k-z_k\|^2)+2\|v_{k+1}-u_k\|)^2\leqslant\\
&\leqslant\beta_k(\delta_k+\delta_kA_k)((1+2R)^2+4\|v_k-z_k\|^2+2\|v_{k+1}-u_k\|^2).
\end{aligned}
\end{equation}
Аналогично оцениваются и шестое слагаемое
\begin{equation}
\label{[3]-42}
\begin{aligned}
&2\beta_k\big\langle [\nabla_w\Phi(u_k,u_k)-\nabla_w^k\Phi(u_k,u_k)]+A_k[\nabla_wP(u_k)-\nabla_w^kP(u_k)],v_{k+1}-u_k\big\rangle\leqslant\\
&\leqslant 2\beta_k(\delta_k+\delta_kA_k)(1+2\|u_k\|)\|v_{k+1}-u_k\|\leqslant\\
&\leqslant \beta_k (\delta_k+\delta_kA_k)2\cdot(1+2\|u_k-v_k\|+2\|v_k-z_k\|+2\|z_k\|)\|v_{k+1}-u_k\|\leqslant\\
&\leqslant\beta_k (\delta_k+\delta_kA_k)(\frac{1}{3}(1+2R+2\|u_k-v_k\|+2\|v_k-z_k\|)^2+3\|v_{k+1}-u_k\|^2)\leqslant\\
&\leqslant\beta_k (\delta_k+\delta_kA_k)((1+2R)^2+4\|u_k-v_k\|^2+4\|v_k-z_k\|^2+3\|v_{k+1}-u_k\|^2),
\end{aligned}
\end{equation}
и седьмое слагаемое
\begin{equation}
\label{[3]-43}
\begin{aligned}
&2\beta_k\big\langle [\nabla_w\Phi(u_k,u_k)-\nabla_w^k\Phi(u_k,u_k)]+A_k[\nabla_wP(u_k)-\nabla_w^kP(u_k)],z_k-u_k\big\rangle\leqslant\\
&\leqslant 2\beta_k(\delta_k+\delta_kA_k)(1+2\|u_k\|)\|z_k-u_k\|\leqslant\\
&\leqslant \beta_k (\delta_k+\delta_kA_k)2\cdot(1+2\|u_k-v_k\|+2\|v_k-z_k\|+2\|z_k\|)(\|z_k-v_k\|+\|v_k-u_k\|)\leqslant\\
&\leqslant\beta_k (\delta_k+\delta_kA_k)(\frac{1}{3}(1+2R+2\|u_k-v_k\|+2\|v_k-z_k\|)^2+3(\|z_k-v_k\|+\|v_k-u_k\|)^2)\leqslant\\
&\leqslant\beta_k (\delta_k+\delta_kA_k)((1+2R)^2+10\|u_k-v_k\|^2+10\|v_k-z_k\|^2).
\end{aligned}
\end{equation}
Для оценки пятого слагаемого воспользуемся условием Липшица \eqref{[3]-29}:
\begin{equation}
\label{[3]-44}
\begin{aligned}
&2\beta_k\big\langle [\nabla_w\Phi(v_k,v_k)-\nabla_w\Phi(u_k,u_k)]+A_k[\nabla_wP(v_k)-\nabla_wP(u_k)],v_{k+1}-u_k\big\rangle\leqslant\\
&2\beta_k(1+A_k)L\|v_k-u_k\|\|v_{k+1}-u_k\|\leqslant\beta_k(1+A_k)L(\|v_k-u_k\|^2+\|v_{k+1}-u_k\|^2).
\end{aligned}
\end{equation}
Для восьмого слагаемого имеем оценку из \eqref{[3]-24},\eqref{[3]-25} при $v=u_k,w=z_k$:
\begin{equation}
\label{[3]-45}
2\beta_k\big\langle [\nabla_w\Phi(u_k,u_k)-\nabla_w\Phi(z_k,z_k)]+A_k[\nabla_wP(u_k)-\nabla_wP(z_k)],z_k-u_k\big\rangle\leqslant 0.
\end{equation}
Для оценки девятого слагаемого воспользуемся условием оптимальности \eqref{[3]-28} при $w=u_k$:
\begin{equation}
\label{[3]-46}
2\beta_k\big\langle \nabla_w\Phi(z_k,z_k)+A_k\nabla_wP(z_k),z_k-u_k\big\rangle\leqslant 0.
\end{equation}
Последнее слагаемое оценим так: используем \eqref{2|ab|},\eqref{An<=Qn}, получим
\begin{equation*}
\begin{aligned}
&4\alpha_k\beta_k\langle u_k-v_k,v_{k+1}-z_k\rangle = 4\alpha_k\beta_k \langle u_k-z_k+z_k-v_k,v_{k+1}-u_k+u_k-v_k+v_k-z_k\rangle\leqslant\\
&\leqslant 4\alpha_k\beta_k \Big[ \|u_k-z_k\|\left( \|v_{k+1}-u_k\|+\|u_k-v_k\|+\|v_k-z_k\|\right)+\\
&\quad +\langle z_k-v_k,v_{k+1}-u_k\rangle+\langle z_k-v_k, u_k-v_k\rangle-\|v_k-z_k\|^2 \Big] \leqslant\\
&\leqslant 4\alpha_k\beta_k\Big[ 3\varepsilon \|u_k-z_k\|^2+\varepsilon\left(\|v_{k+1}-u_k\|^2+\|u_k-v_k\|^2+\|v_k-z_k\|^2 \right)-\|v_k-z_k\|^2+\\
&\quad+\delta\|z_k-v_k\|^2+\frac{1}{\delta}\|v_{k+1}-u_k\|^2+\Delta\|z_k-u_k\|^2+\frac{1}{\Delta}\|u_k-v_k\|^2\Big]\leqslant\\
&\leqslant 4\alpha_k\beta_k\Big[6\varepsilon \left(\|u_k-v_k\|^2+\|z_k-v_k\|^2\right)+\varepsilon\left(\|v_{k+1}-u_k\|^2+\|u_k-v_k\|^2+\|v_k-z_k\|^2\right)-\\
&\quad-\|v_k-z_k\|^2+\delta\|z_k-v_k\|^2+\frac{1}{\delta}\|v_{k+1}-u_k\|^2+\Delta\|z_k-u_k\|^2+\frac{1}{\Delta}\|u_k-v_k\|^2\Big]\leqslant\\
&\leqslant 4\alpha_k\beta_k\left[ (7\varepsilon +\delta+\Delta-1)\|v_k-z_k\|^2+(7\varepsilon+\frac{1}{\Delta})\|u_k-v_k\|^2+(\varepsilon+\frac{1}{\delta})\|v_{k+1}-u_k\|^2\right],
\end{aligned}
\end{equation*}
где $\varepsilon,\delta,\Delta$ - какие-то положительные числа. Предположим, что
\begin{equation*}
7\varepsilon +\delta+\Delta-1 <0.
\end{equation*}
В качестве значений этих величин можем, например, взять $\varepsilon = \frac{1}{14},\delta = \frac{1}{5},\Delta = \frac{1}{10}$, выполняющих данную систему неравенств. При взятии этих значений имеем такую оценку:
\begin{equation}
\label{[3]-47}
\begin{aligned}
&4\alpha_k\beta_k\langle u_k-v_k,v_{k+1}-z_k\rangle \leqslant\\
\leqslant &-\frac{4}{5}\alpha_k\beta_k\|v_k-z_k\|^2+42\alpha_k\beta_k\|u_k-v_k\|^2+\frac{142}{7}\alpha_k\beta_k\|v_{k+1}-u_k\|^2.
\end{aligned}
\end{equation}
Наконец, десятое слагаемое разбивается на три части, в которых каждая из них оцениваются аналогичным образом:
\begin{equation}
\label{[3]-47-2}
\begin{aligned}
&2\beta_k\langle \nabla_w^k\Phi(v_k,u_k)-\nabla_w^k\Phi(u_k,u_k),z_k-v_{k+1} \rangle=\\
&=2\beta_k\langle \nabla_w^k\Phi(v_k,u_k)-\nabla_w\Phi(v_k,u_k),z_k\textcolor{blue}{-v_k+v_k-u_k+u_k}-v_{k+1} \rangle+\\
&+2\beta_k\langle \nabla_w\Phi(v_k,u_k)-\nabla_w\Phi(u_k,u_k),z_k\textcolor{blue}{-v_k+v_k-u_k+u_k}-v_{k+1}  \rangle+\\
&+2\beta_k\langle \nabla_w\Phi(u_k,u_k)-\nabla_w^k\Phi(u_k,u_k),z_k\textcolor{blue}{-v_k+v_k-u_k+u_k}-v_{k+1}  \rangle\leqslant\\
&\leqslant \beta_k\delta_k\big((1+2R)^2+13\|v_k-z_k\|^2+10\|u_k-v_k\|^2+9\|u_k-v_{k+1}\|^2 \big)+\\
&+\beta_kL(3\|v_k-z_k\|^2+4\|u_k-v_k\|^2+3\|u_k-v_{k+1}\|^2)+\\
&+\beta_k\delta_k\big((1+2R)^2+13\|v_k-z_k\|^2+13\|u_k-v_k\|^2+9\|u_k-v_{k+1}\|^2 \big).
\end{aligned}
\end{equation}
Подставив оценки \eqref{[3]-41}-\eqref{[3]-47-2} в \eqref{[3]-40}, имеем
\begin{equation}
\label{[3]-48}
\begin{aligned}
&\|v_{k+1}-z_k\|^2\leqslant \|v_k-z_k\|^2\left[1-\frac{4}{5}\alpha_k\beta_k+18\beta_k(\delta_k+\delta_kA_k)+\beta_k(26\delta_k+3L)\right]+\\
&+\|v_k-u_k\|^2\left[-1+\beta_k(1+A_k)L+14\beta_k(\delta_k+\delta_kA_k)+\beta_k(42\alpha_k+23\delta_k+4L)\right]+\\
&+\|v_{k+1}-u_k\|^2\left[-1+\beta_k(1+A_k)L+5\beta_k(\delta_k+\delta_kA_k)+\beta_k(\frac{142}{7}\alpha_k+18\delta_k+3L)\right]+\\
&+(3\beta_k(\delta_k+\delta_kA_k)+2\beta_k\delta_k)(1+2R)^2.
\end{aligned}
\end{equation}
По условию теоремы, имеем
\begin{equation*}
\sup_{k\geqslant 0}\beta_k(1+A_k)<1/L,\quad \lim_{k\rightarrow\infty}(\delta_k+\delta_k A_k)=0,\quad \lim_{k\to \infty} \alpha_k=0.
\end{equation*}
Следовательно, коэффициенты при $\|v_k-u_k\|^2,\|v_{k+1}-u_k\|^2$ в \eqref{[3]-48} неположительны $\forall k\geqslant k_0$, где $k_0$ - достаточно большое натуральное число. Поэтому
\begin{equation}
\label{[3]-49}
\begin{aligned}
\|v_{k+1}-z_k\|^2&\leqslant \|v_k-z_k\|^2\left[1-\frac{4}{5}\alpha_k\beta_k+18\beta_k(\delta_k+\delta_kA_k)+\beta_k(26\delta_k+3L)\right]+\\
&+(3\beta_k(\delta_k+\delta_kA_k)+2\beta_k\delta_k)(1+2R)^2,\quad \forall k\geqslant k_0.
\end{aligned}
\end{equation}
Подставим в \eqref{[3]-36} оценки \eqref{[3]-49} и \eqref{[3]-15} при $m=k+1$. В силу монотонности $\{A_k\}$ имеем
\begin{equation*}
\begin{aligned}
0\leqslant b_{k+1}&\leqslant\big\{b_k^2\left[1-\frac{4}{5}\alpha_k\beta_k+18\beta_k(\delta_k+\delta_kA_k)+\beta_k(26\delta_k+3L)\right]+\\
&+(3\beta_k(\delta_k+\delta_kA_k)+2\beta_k\delta_k)(1+2R)^2\big\}^{1/2}+\frac{(A_{k+1}-A_k)R_1}{2\alpha_k},\quad \forall k\geqslant k_0.
\end{aligned}
\end{equation*}
Справедливо следующее:
\begin{equation*}
(a+b)^2\leqslant(1+\varepsilon)(a^2+b^2/\varepsilon),\quad\forall\varepsilon>0.
\end{equation*}
При взятии
\begin{equation*}
\begin{aligned}
a=\Big\{b_k^2\Big[1-\frac{4}{5}\alpha_k\beta_k&+18\beta_k(\delta_k+\delta_kA_k)+\beta_k(26\delta_k+3L)\Big]+(3\beta_k(\delta_k+\delta_kA_k)+2\beta_k\delta_k)(1+2R)^2\Big\}^{1/2},\\
&b=\frac{(A_{k+1}-A_k)R_1}{2\alpha_k},\quad \varepsilon=\frac{1}{2}\alpha_k\beta_k
\end{aligned}
\end{equation*}
имеем
\begin{equation*}
\begin{aligned}
&0\leqslant b_{k+1}^2\leqslant b_k^2\left[1-\frac{4}{5}\alpha_k\beta_k+18\beta_k(\delta_k+\delta_kA_k)+\beta_k(26\delta_k+3L)\right](1+\frac{1}{2}\alpha_k\beta_k)+\\
&+(3\beta_k(\delta_k+\delta_kA_k)+2\beta_k\delta_k)(1+2R)^2\times(1+\frac{1}{2}\alpha_k\beta_k)+\left(\frac{(A_{k+1}-A_k)R_1}{2\alpha_k}\right)^2\frac{2}{\alpha_k\beta_k}(1+\frac{1}{2}\alpha_k\beta_k)=\\
&=b_k^2\left[1-\frac{3}{10}\alpha_k\beta_k-\frac{2}{5}(\alpha_k\beta_k)^2+(18\beta_k(\delta_k+\delta_kA_k)+\beta_k(26\delta_k+3L))(1+\frac{1}{2}\alpha_k\beta_k)\right]+\\
&+(3\beta_k(\delta_k+\delta_kA_k)+2\beta_k\delta_k)(1+2R)^2\times(1+\frac{1}{2}\alpha_k\beta_k)+\left(\frac{(A_{k+1}-A_k)R_1}{2\alpha_k^2\beta_k}\right)^2 2\alpha_k\beta_k(1+\frac{1}{2}\alpha_k\beta_k).
\end{aligned}
\end{equation*}
Как мы видим, последовательность $\{b_k^2\}$ удовлетворяет неравенствам \eqref{[3]-34} при
\begin{equation}
\label{[3]-50}
\begin{aligned}
s_k=\frac{3}{10}\alpha_k\beta_k\Big[1+\frac{4}{3}\alpha_k\beta_k-(60\frac{\delta_k+\delta_kA_k}{\alpha_k}&+\frac{10}{3}\cdot\frac{26\delta_k+3L}{\alpha_k})(1+\frac{1}{2}\alpha_k\beta_k)\Big]\\
d_k=\alpha_k\beta_k\Big[(3\frac{(\delta_k+\delta_kA_k)}{\alpha_k}+2\frac{\delta_k}{\alpha_k})(1+2R)^2&\times(1+\frac{1}{2}\alpha_k\beta_k)+\\
&+\left(\frac{(A_{k+1}-A_k)R_1}{2\alpha_k^2\beta_k}\right)^2 (2+\alpha_k\beta_k)\Big].
\end{aligned}
\end{equation}
С учетом \eqref{[3]-30}, взяв при необходимости номер $k_0$ ещё большим, можем считать, что 
\begin{equation*}
0<s_k\leqslant 1,\quad \forall k\geqslant k_0.
\end{equation*} 
Кроме того, 
\begin{equation*}
0\leqslant A_{k+1}-A_k=\left(\frac{A_{k+1}-A_k}{\alpha_k^2\beta_k}\alpha_k \right)(\alpha_k\beta_k)\leqslant \alpha_k\beta_k,\quad \forall k\geqslant k_0.
\end{equation*}
Суммируя эти неравенства по $k$ от $k_0$ до некоторого $N$, получаем 
\begin{equation*}
\sum_{k=k_0}^{N}\alpha_k\beta_k\geqslant A_{N+1}-A_{k_0}\to \infty, \text{ при } N\to \infty,
\end{equation*}
т.е. ряд $\sum\limits_{k=0}^{\infty}\alpha_k\beta_k$ расходится. Как следует из \eqref{[3]-50}, $\lim\limits_{k\to \infty} \frac{s_k}{\alpha_k\beta_k}=\text{const}$, тогда из признака сравнения вытекает $\sum\limits_{k=0}^{\infty}s_k=+\infty$. Наконец, из условий \eqref{[3]-30} и выражений $\eqref{[3]-50}$ для $s_k,d_k$ вытекает, что 
\begin{equation*}
\lim\limits_{k\to\infty}\frac{d_k}{s_k}=\underbrace{\lim\limits_{k\to\infty}\frac{d_k}{\alpha_k\beta_k}}_{\to 0}\cdot\underbrace{\lim\limits_{k\to\infty}\frac{\alpha_k\beta_k}{s_k}}_{\to \text{const}}=0.
\end{equation*}
Видно, что все условия в \eqref{[3]-34} выполнены.\\
Всякая последовательность $\{b_k^2\}$, удовлетворяющая неравенствам и условиям \eqref{[3]-34}, сходится к нулю. (см. \cite{5} c.90, лемма 6). Тем самым доказано, что $\lim\limits_{k\to \infty}\|v_k-z_k\|=0$. Отсюда и из \eqref{[3]-14},\eqref{[3]-33} следует равенство \eqref{convergence}.\\
Так как коэффициенты $s_k,d_k$ в \eqref{[3]-34}, как можно увидеть из формул \eqref{[3]-50}, не зависят от выбора конкретной реализации приближений $\nabla_w^k\Phi(v,w),\nabla_w^kP(w)$, лишь бы выполнялось условие \eqref{[3]-6}, то сходимость в \eqref{convergence} равномерна относительно выбора $\nabla_w^k\Phi(v,w),\nabla_w^kP(w)$ из \eqref{[3]-6}. Теорема доказана. $\qedsymbol$

Итак, полностью описан регуляризованный экстраградиентный метод и доказана сходимость метода.
\clearpage
%\newpage
	\section{Численная проверка применимости методов}
Применимость метода оценивалась с помощью запусков программ на модельных примерах. 

Итак, в постановке задачи был рассмотрен простой
\begin{example}
	\label{exp-1}
	$\Phi(v,w)=vw,\mathbf{W}=\{w\in \mathbb{E}^1:|w|\leqslant 1 \},\mathbf{W_*}=\{0 \}.$
\end{example}
Как мы уже поняли, приближенная задача
\begin{equation}
v_*^{\delta}\in \mathbf{W},\quad\Phi^{\delta}(v_*^{\delta},v_*^{\delta})\leqslant \Phi^{\delta}(v_*^{\delta},w),\quad \forall w\in \mathbf{W}
\end{equation}
не имеет решения при любых сколь угодно малых $\delta >0$, то есть исходная задача имеет \textit{неустойчивый характер}.

Протестируем регуляризованный экстраградиентный метод, который предложен и рассмотрен в предыдущей главе. 

Сначала установим справедливость условий метода. Ограничения-неравенства можно записать в следующем виде: $g_1(w)=w-1,g_2(w)=-w-1$. Тогда штрафная функция $P(w)=g_1^+(w)+g_2^+(w)$ равна $-w-1$ при $w\leqslant -1$; $0$ при $-1\leqslant w\leqslant 1$ и $w-1$ при $w\geqslant 1$. Градиент $\nabla_w \Phi(v,w)=v.$ В качестве приближений градиентов возьмём $\nabla_w^k\Phi(v,w)=v+\delta_k w,\nabla_w^kP(w)=\nabla_k P(w)+\delta_k$. Понятно, что выполняются условия приближения \eqref{[3]-6}:
\begin{equation*}
\begin{aligned}
&\|\nabla_w^k\Phi(v,w)-\nabla_w\Phi(v,w)\|=\delta_k\|w\|\leqslant \delta_k(1+\|v\|+\|w\|),\\
&\|\nabla_w^kP(w)-\nabla_wP(w)\|=\delta_k\leqslant \delta_k(1+\|w\|).
\end{aligned}
\end{equation*}
А также ясно, что градиенты функции удовлетворяют условиям Липшица \eqref{[3]-29}\eqref{[3]-29-2}. Функция Тихонова выпукла по $w$, так как:
\begin{equation*}
\big\langle (T_k)_w^{''}(v,w)h,h\big\rangle = \langle 2\alpha_k h,h \rangle = 2\alpha_k\|h\|^2\geqslant 0,\forall h\in \mathbb{E}^1
\end{equation*}
Другие условия на функции $\Phi(v,w),g_i(w),i=1,...,m+s$ и множества $\mathbf{W},\mathbf{W_{0}}$ выполняются, что проверяется легко. Сразу заметим, что здесь множество $\mathbf{W_{0}}$ - вся числовая прямая $\mathbb{E}^1$, поэтому в предложенном методе операции проектирования являются излишними. 

Параметры $\alpha_k,A_k,\delta_k,\beta_k$, использованные в алгоритме, взяты из замечания \ref{remark 3-2} при $\alpha = 1, \delta =1, A = 1, L=1$.

Схема алгоритма выглядит так:
\newpage
\begin{table}[!htbp]
	\begin{tabular}{l}
		\toprule
		 \textbf{Алгоритм:} Регуляризованный экстраградиентный метод для примера \eqref{exp-1}. \\
		\midrule
		\textbf{Даны:} $\nabla_w^k\Phi(v,w;\delta_k) = v+\delta_kw$, и произвольное $v_0\in\mathbb{E}^1$.\\
		Для $k=1,2,...$ делаем цикл, пока $k$ меньше максимального числа итераций: \\
		%\textbf{1.} $\displaystyle \alpha_k = \frac{1}{\sqrt[5]{k+1}},A_k=\sqrt[5]{k+1},\delta_k = \frac{1}{k+1},\beta_k = \frac{1}{2(1+A_k)}$\\
		\textbf{1.} $\displaystyle \alpha_k = \frac{1}{k+1},A_k= k+1,\delta_k = \frac{1}{k+1},\beta_k = \frac{1}{2(1+A_k)}$\\
		\textbf{2.}  $ u_k = v_k-\beta_k\left[ \nabla_w^k\Phi(v_k,v_k;\delta_k)+A_k\nabla_wP(v_k;\delta_k)\right]$\\
		\textbf{3.}  $ v_{k+1} = v_k -\beta_k\left[\nabla_w^k\Phi(v_k,u_k;\delta_k)+A_k\nabla_wP(u_k;\delta_k)-2\alpha_k(u_k-v_k)\right]$\\
		\bottomrule
	\end{tabular}
	\caption{\label{tab:test-1} Схема алгоритма для примеров.}
\end{table}
Его реализация на Си++ доступен в \textbf{Приложениях}. Это алгоритм даёт положительный результат: При $k=7$ значение $v_k$ начинает уменьшаться; при $k=5000$ имеем $v=0.00812547$, что уже меньше $1\%$ отличается от решения $v_*=0$. Это означает, что данный вычислительный процесс сходится к реальному решению исходной задачи с нужной нам точностью.

На другом примере покажем, что метод хорошо работает в случае, когда число решений \textit{больше одного}.
\begin{example}
	\label{exp-2}
	$\Phi(v,w)=vw^2,\mathbf{W}=\{w\in \mathbb{E}^1:|w|\leqslant 1 \},\mathbf{W_*}=\{0,-1 \}.$
\end{example}
Видно, что у этой задачи есть два решения, из них нормальным решение является $v_*=0$. В принципе, эта схема алгоритма отличается от прошлого примера лишь заданием $\nabla_w^k\Phi(v,w)$. При тестировании на программе, получается что при $k=5000$ имеем $v_k=0.000750162$, что меньше чем на $0.1\%$ отличается от нормального решения $v_*=0$. Это в свою очередь показывает, что метод сходится к исходому решению.

Полученные результаты показывают, что методы, предложенные в работе, могут быть использованы для решения равновесных задач, неустойчивых к погрешностям задания исходных данных.
\clearpage
	\phantomsection 
\addcontentsline{toc}{section}{Заключение}
\section*{Заключение}
В данной работе были предложены и обоснованы как общий метод стабилизации для решения равновесных задач, так и конкретный метод решения. При исследовании были получены следующие результаты:
\begin{itemize}
	\item Предложен новый общий подход регуляризации к решению возмущенной равновесной задачи с использованием стабилизатора $-\alpha_k\|v-w\|^2$.
	\item Разработан новый регуляризованный экстраградиентный метод с использованием стабилизатора $-\alpha_k\|v-w\|^2$ для решения неустойчивых задач равновесного программирования.
	\item Доказан ряд теорем и предложений, подтвержающих, что разработанные методы сходятся при некоторых определенных условиях. 
	\item Проведены вычислительные эксперименты на различных тестовых примерах, показывающие способности предложенных методов при решении неустойчивых равновесных задач. 
	
\end{itemize}
	
\clearpage
%\newpage
	\phantomsection
\addcontentsline{toc}{section}{Список литературы}
\section*{Список литературы}
	\begingroup
		\renewcommand{\section}[2]{}%
		\begin{thebibliography}{}
		% textbooks
		\bibitem{numMethodopt} \emph{Васильев Ф.П.} Численные методы решения экстремальных задач. М.: Наука, 1988.
		\bibitem{5} \emph{Васильев Ф.П.} Методы оптимизации. М.: Факториал, 2002.
		\bibitem{num-methods} \emph{Самарский А.А., Гулин А.В.} Численные методы. М.: Наука, 1989.
		
		% main references
		\bibitem{centralbib} \textit{Антипин A.C., Васильев Ф.П.} Метод стабилизации для решения задач равновесного программирования // Ж. вычисл. матем. и матем. физ. 1999. Т. 39. № 11. С 1779-1786.
		\bibitem{centralbib-2}  \emph{Антипин А.С., Васильев Ф.П., Шпирко С.В.} Регуляризованный экстраградиентный метод решения задач равновесного программирования с неточно заданным множеством. // Ж. вычисл. матем. и матем. физ. 2005, том 45, №4, с. 650–660.
		
		% others main references
		\bibitem{4} \emph{Шпирко C.B.} Метод кососимметричной регуляризации для решения задач равновесного программирования: Дис. ... канд. физ.-матем. наук. М.: ВЦ РАН, 2001.
		\bibitem{6} \emph{Шпирко С.В.} О существовании и единственности решения задачи равновесного программирования, Изв. вузов. Матем., 2002, номер 12, 79–83
		
		% methods 
		\bibitem{7} \textit{Антипин A.C.} Метод внутренней линеаризации для задач равновесного программирования,Ж. вычисл. матем. и матем. физ., 2000, том 40, номер 8, 1142–1162.
		\bibitem{8} \textit{Антипин A.C.} Равновесное программирование: методы градиентного типа // Автоматика и телемехан., 1997. №8. С. 125-137.
		\bibitem{8-2} \textit{Антипин А.С.} Вычисление неподвижных точек экстремальных отображений при помощи методов градиентного типа, Ж. вычисл. матем. и матем. физ., 1997, том 37, номер 1, 42–53.
		\bibitem{9} \textit{Антипин A.C.} Равновесное программирование: проксимальные методы, Ж. вычисл. матем. и матем. физ., 1997, том 37, номер 11, 1327–1329.
		\bibitem{10} \emph{Будак Б.А.} Непрерывные методы решения задач равновесного программирования: Дис. ... канд. физ.-матем. наук. М.: ВЦ РАН, 2003.
		\bibitem{10-2} \emph{Будак Б.А.} Метод стрельбы для решения задач равновесного программирования, Ж. вычисл. матем. и матем. физ., 2013, том 53, номер 12, 2008–2013.

		
		% methods of regulazition
		\bibitem{11} \textit{Васильев Ф.П., Антипин A.C.} Методы регуляризации поиска неподвижной точки экстремальных отображений // Вестн. МГУ. Сер. 15. Вычисл. матем. и матем. физ. 1998. Т. 38. № 1. С. 11-14.
		\bibitem{12} \textit{Антипин А.С., Васильев Ф.П.} Метод квазирешений для решения равновесных задач с неточно заданным
		множеством // Вестн. ун-та Дружбы народов. 2001. № 8(2). С. 10-16.
		\bibitem{13} \textit{Антипин A.C., Васильев Ф.П.} Метод невязки для решения равновесных задач с неточно заданным множеством // Ж. вычисл. матем. и матем. физ. 2001. Т. 41. № 1. С. 3-8.
		\bibitem{14} \textit{Антипин A.C., Васильев Ф.П.} Методы регуляризации для решения задачи равновесного программирования с неточными входными данными, основанные на расширении множества // Ж. вычисл. матем.
		и матем. физ. 2002. Т. 42. № 8. С. 1158-1165.
		
		% concrete methods of regulazition
		\bibitem{15} \textit{Антипин A.C., Васильев Ф.П., Шпирко C.B.} Регуляризованный экстраградиентный метод решения задач
		равновесного программирования // Ж. вычисл. матем. и матем. физ. 2003. Т. 43. № 10. С. 1451-1458.
		\bibitem{16} \textit{Будак Б.А.} Регуляризованный непрерывный метод линеаризации первого порядка прогнозного типа с переменной метрикой для решения задач равновесного программирования с неточно заданным множеством // Ж. вычисл. матем. и матем. физ. 2005. Т. 45. № 4. С. 637-649.
		
		
		\end{thebibliography}
	\endgroup
	\clearpage
	\phantomsection
\addcontentsline{toc}{section}{Приложения}
\section*{Приложения}
\noindent\textbf{Реализация регуляризованного экстраградиентного метода на Си++ для модельных примеров.}
\begin{lstlisting}[caption={Реализация метода на Си++}]
#include <iostream>
#include <cmath>

using namespace std;

const int N = 5000;
const float v_start = 1.0;
const float eps = 0.01;

float gradPsiError(float v, float w, float delta) { return v + delta*w;}
\\ For the second example: { return 2*w*v + delta*w;}
float gradPError(float w, float delta){ return (w<=-1)? -1.0 + delta: ((w>=1)? 1.0+delta: delta);}

int main()
{
	int k = 1;
	float v = v_start;
	while(k <= N){
		float alpha = 1.0/(k+1);
		float A = k+1;
		float delta = 1.0/(k+1);
		float beta = 1.0/(2+2*A);
		float u = v - beta * (gradPsiError(v, v, delta) + A * gradPError(v, delta));
		float v = v - beta * ( gradPsiError(v, u, delta) + A * gradPError(u, delta) - 2*alpha*(u-v));
		k++;
		cout << "k=" << k << " v=" << v << endl;
	}
	return 0;
}

\end{lstlisting}


\end{document}