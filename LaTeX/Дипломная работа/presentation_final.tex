%%%%%%%%%%%%%%%% что - то про dvips
% latex yourfile.tex
% dvips yourfile.dvi
% ps2pdf yourfile.ps

%aspectratio=169
\documentclass[unicode, t, 11pt]{beamer}% [t], [c], или [b] --- вертикальное выравнивание на слайдах (верх, центр, низ)
%\documentclass[aspectratio=169]{beamer} % Соотношение сторон

%%% Работа с русским языком
\usepackage{cmap}					% поиск в PDF
\usepackage{mathtext} 				% русские буквы в формулах
% \usepackage[T2A]{fontenc}			% кодировка
\usepackage[T1, T2A]{fontenc}
\usepackage[utf8]{inputenc}			% кодировка исходного текста
\usepackage[english,russian]{babel}	% локализация и переносы
\usepackage{bm}

% %% Beamer по-русски
% \newtheorem{rtheorem}{Теорема}
% \newtheorem{rproof}{Доказательство}
% \newtheorem{rexample}{Пример}
\usepackage{amsmath, amsfonts, amssymb, amsthm, mathtools} % AMS
\usepackage{icomma} % "Умная" запятая: $0,2$ --- число, $0, 2$ --- перечисление

%% Номера формул
%\mathtoolsset{showonlyrefs=true} % Показывать номера только у тех формул, на которые есть \eqref{} в тексте.
%\usepackage{leqno} % Нумерация формул слева

%%% Работа с картинками
\usepackage{graphicx}  % Для вставки рисунков
%\graphicspath{{presentation/images/}}  % папки с картинками
\setlength\fboxsep{3pt} % Отступ рамки \fbox{} от рисунка
\setlength\fboxrule{1pt} % Толщина линий рамки \fbox{}
\usepackage{wrapfig} % Обтекание рисунков текстом

%%% Работа с таблицами
\usepackage{array,tabularx,tabulary,booktabs} % Дополнительная работа с таблицами
\usepackage{longtable}  % Длинные таблицы
\usepackage{multirow} % Слияние строк в таблице

%%% Программирование
\usepackage{etoolbox} % логические операторы

%%% Другие пакеты
\usepackage{lastpage} % Узнать, сколько всего страниц в документе.
\usepackage{soul} % Модификаторы начертания
\usepackage{csquotes} % Еще инструменты для ссылок
%\usepackage[style=authoryear,maxcitenames=2,backend=biber,sorting=nty]{biblatex}
\usepackage{multicol} % Несколько колонок

%%% Картинки
\usepackage{tikz} % Работа с графикой
% \usepackage{pgfplots}
% \usepackage{pgfplotstable}

\usepackage{enumerate}
\usepackage{enumitem}
\setlist[itemize]{noitemsep, topsep=0pt}

\usepackage{caption}
\DeclareCaptionFont{tiny}{\tiny}
\captionsetup{belowskip=0pt}
\setlength{\belowcaptionskip}{-10pt}

\newlength{\mylen}

\usepackage{tabularx}

%\usefonttheme{professionalfonts} % using non standard fonts for beamer
\usefonttheme{serif} % default family is serif

\setbeamertemplate{footline}[frame number]



%%%%%%%%%%%%%%%%%%%%%%%%%%%%%%%%%%%%%%%%%%%%%%%%%%%%%%%%%%%%%%%%%%%%%%%%%%%%%%%%%%%%
\title{
	{
	\footnotesize\color{black}Московский государственный университет имени М.В.\,Ломоносова\\
	Университет МГУ-ППИ в Шэньчжэне\\
    Факультет вычислительной математики и кибернетики\\}
    \vspace{\baselineskip}
    {\LARGE Регуляризующие алгоритмы в неустойчивых равновесных задачах}
}
\subtitle{Выпускная квалификационная работа}
\author{\footnotesize Студент: Сюй Минчуань\\
		Научный руководитель: к.ф-м.н., доцент ВМК МГУ\\
		Будак Борис Александрович}

\date{\includegraphics[height=0.8cm]{./images/MSU}\\
	  \scriptsize
	  Шэньчжэнь\\
	  7 июня 2021 г.}
%\logo{\includegraphics[height=1.5cm]{./images/MSU}}
%%%%%%%%%%%%%%%%%%%%%%%%%%%%%%%%%%%%%%%%%%%%%%%%%%%%%%%%%%%%%%%%%%%%%%%%%%%%%%%%%%%%


\begin{document}
	\frame[plain]{\titlepage}  % Титульный слайд
	\section{Кратое введение о равновесном программировании}
		\begin{frame}
			\frametitle{\insertsection}
			Рассматривается задача \textbf{равновесного программирования}: найти точку $v_*$ из условий 
			\begin{equation}\label{1}
			\begin{aligned}
			&v_*\in \mathbf{W}\subseteq \mathbb{R}^n,\quad \Phi(v_*,v_*)\leqslant\Phi(v_*,w),\quad w\in \mathbf{W},\\
			&\mathbf{W}=\{w\in \mathbf{W_0}:g_i(w)\leqslant 0,i=\overline{1,m},g_i(w)=0,i=\overline{m+1,s} \},
			\end{aligned}
			\end{equation}
			где $\mathbf{W_0}$ - множество из $\mathbb{R}^n$ (обычно имеет простой вид, возможно $\mathbf{W_{0}}=\mathbb{R}^n$).\\[5mm]
			\textbf{Типичные равновесные задачи:}
			\begin{itemize}
				\item Задача математического программирования.
				\item Задача поиска седловых точек.
				\item Задача поиска равновесия по Нэшу.
			\end{itemize}
		\end{frame}
	\section{Существующие методы к решению равновесной задачи}
		\begin{frame}
			\frametitle{\insertsection}
					\begin{itemize}[label=\(\bullet\)]
						\item \textbf{Градиентный метод прогнозного типа.}\\
						Применяется в случае $\mathbf{W}=\mathbb{R}^n$.\\
						Схема метода: начальное приближение $v_0\in \mathbf{W}$ - задано;\\
						\begin{equation*}
						\left\{\begin{array}{l}
						u_k=v_k-\alpha_k \nabla_w\Phi(v_k,v_k),\\
						v_{k+1}=v_k-\alpha_k\nabla_w\Phi(u_k,u_k),
						\end{array}\right.
						\end{equation*}
						где $k=0,1,2,\ldots,$ параметры $\alpha_k>0$.
						\item Экстраградиентный метод
						\item Метод линеаризации
						\item Проксимальный метод
						\item Непрерывные варианты методов
						\item Метод стрельбы
						\item \ldots
					\end{itemize}
				Перечислимые выше методы уже хорошо изучены. Отметим, что проблема \textbf{неустойчивости} задачи существенна, для которой требуются изменения в методах.
	\end{frame}
			
	\section{Неустойчивость к возмущению данных}
	\begin{frame}
	\frametitle{\insertsection}
	Допустим, что 
	\begin{equation}
	\label{2}
	\begin{aligned}
	&\lim\limits_{\delta\to 0} \left|\Phi^{\delta}(v,w)-\Phi(v,w)\right|=0;\\
	&\lim\limits_{\delta\to 0} \left\|\nabla_w^{\delta}\Phi(v,w)-\nabla_w\Phi(v,w)\right\|=0.
	\end{aligned}
	\end{equation}
	Однако подход к использованию метода с приближенными данными из \eqref{2} не всегда дает возможность получить приближенное решение задачи. \\[15mm]
	
	Исследования данной тематики могут найти в работах Антипин А.С., Васильев Ф.П., Шпирко С.В., Будак Б.А. и др.
	
	\end{frame}

\section{Основные методы регуляризации}
\begin{frame}
\frametitle{\insertsection}
\begin{itemize}[label=\(\bullet\)]
	\item\textbf{Метод стабилизации:} Введем функцию Тихонова: $t_{\delta} (v,w)=\Phi_{\delta}(v,w)+\alpha \Omega(w),\quad w\in \mathbf{W_0},\,\alpha>0$.\\
	Будем искать точку, удовлетворяющую условиям
	$v_{\delta}\in \mathbf{W_0},\quad t_{\delta}(v_\delta,v_\delta)\leqslant t_{\delta}(v_{\delta},w)+\varepsilon,\forall w\in\mathbf{W_0},\varepsilon>0.$\\
	Точки, выполняющие данные условия, сходятся к исходному решению при подобном выборе параметров.\\
	\emph{Сочетание с штрафной функцией:}\\
	Вместо сложного множества $\mathbf{W}$ работаем на более простом множестве $\mathbf{W_0}$ (возможно $\mathbf{W_0}=\mathbb{R}^n$), добавив штрафное слагаемое $AP_{\delta}(w)$ в тихоновскую функцию.
	\item Метод невязки.
	\item Метод квазирешений.
\end{itemize}
\end{frame}

\section{Основная цель: Применение нового стабилизатора}
\subsection{Часть 1. Метод стабилизации с новым стабилизатором $-\alpha_k\|v-w\|^2$}
\begin{frame}
\frametitle{\insertsection}
\framesubtitle{\insertsubsection}
Некоторые \textbf{предварительные} рассуждения:
\begin{itemize}[label=\(\bullet\)]
	\item Допустим, что для некоторой точки $v_*\in \mathbf{W_*}$ \textit{существуют} постоянные $v>0,c_i\geqslant 0$ такие, что
\begin{equation}
\label{7}
\Phi(v_*,v_*)\leqslant \Phi(v_*,w)+\sum_{i=1}^{s} c_i[g_i^+(w)]^v,\quad \forall w\in\mathbf{W_0}
\end{equation}
	\item \emph{Достаточное условие} для непустоты $\mathbf{W}_{*\delta}$.
	\begin{equation}
	\label{13}
	\begin{aligned}
	&BA^{-v/(p-v)}+\delta\|v_*\|^2(3+A)+2(\delta+A\delta)\leqslant 1/2\varepsilon(\delta),\\ &\alpha\|v_*-w\|^2+\|w\|^2(\delta+A\delta)\leqslant 1/2 \varepsilon(\delta),\forall w\in \mathbf{W_0},\delta>0
	\end{aligned}
	\end{equation}
	где $v_*\in \mathbf{W}_*$ взята из условия \eqref{7}, $B$ - константа определенного вида.
\end{itemize}
\end{frame}
\begin{frame}
\frametitle{\insertsection}
\framesubtitle{\insertsubsection}
Некоторые \textbf{предварительные} рассуждения:
\begin{center}
	\begin{itemize}[label=\(\bullet\)]
		\item \textit{Кососимметричность} функции: $\Phi(w,w)-\Phi(w,v)-\Phi(v,w)+\Phi(v,v)\geqslant 0,\quad \forall w,v\in \mathbf{W_0}$.
		\item Пусть приближения $\Phi_{\delta}(v,w),P_{\delta}(w)$ для функций $\Phi(v,w),P(w)$ таковы, что
		\begin{equation}
		\label{noise}
		\begin{aligned}
		&\big|\Phi_{\delta}(v,w)-\Phi(v,w)\big|\leqslant \delta\big(1+\|v\|^2+\|w\|^2\big),\quad v,w\in\mathbf{W_0},\delta >0,\\
		&\big|P_{\delta}(w)-P(w)\big|\leqslant \delta\big(1+\|w\|^2\big),\quad w\in\mathbf{W_0},\delta >0.
		\end{aligned}
		\end{equation}
	\end{itemize}
\end{center}
\end{frame}

\begin{frame}
\frametitle{\insertsection}
\framesubtitle{\insertsubsection}
\begin{block}{Результат 1: Теорема о сходимости метода стабилизации}\begin{footnotesize}
	Пусть выполнены следующие условия\\
	\textbf{1)} $W_0$ - замкнутое ограниченное множество, функции $g_i(w),|g_i(w)|,\Phi(w,w)$ полунепрерывны снизу на $\mathbf{W_0}$; $\Phi(v,w)$ полунепрерывна сверху по $v$ на $\mathbf{W_0}$ $\forall w\in W_0$; $\mathbf{W_*}\ne\emptyset$; для некоторой $v_*\in \mathbf{W_*}$ выполнено условие \eqref{7}; $\Phi(v,w)$ кососимметрична на $\mathbf{W_0}$;\\
	\textbf{2)} $\Omega(v,w)=-\alpha_k\|v-w\|^2$ - \textit{стабилизатор} задачи, $P(w)$ - штрафная функция при $p\geqslant v$;\\
	\textbf{3)} приближения $\Phi_{\delta}(v,w),P_{\delta}(w)$ функций $\Phi(v,w),P(w)$ удовлетворяют условиям \eqref{noise};\\
	\textbf{4)} параметры $\alpha=\alpha(\delta)>0,$ $A=A(\delta)>0,$ $\varepsilon=\varepsilon(\delta)>0,\delta>0$, удовлетворяют условиям \eqref{13} и, кроме того
	$\lim\limits_{\delta\rightarrow 0}\alpha(\delta)=0$,$\lim\limits_{\delta\rightarrow 0}A(\delta)=0$,$\lim\limits_{\delta\rightarrow 0}\varepsilon(\delta)=0,\lim\limits_{\delta\rightarrow 0} \delta A(\delta)=0,\sup\limits_{\delta>0}\frac{3\delta+\delta A(\delta)}{\alpha(\delta)}<+\infty, \sup\limits_{\delta>0}\frac{\varepsilon(\delta)}{\alpha(\delta)}<+\infty.$\\
	\textbf{Тогда} $\mathbf{W}_{*\delta}\ne\emptyset,\,\forall\delta>0$, и $\lim\limits_{\delta\rightarrow 0}\rho(v_{\delta},\mathbf{W_*})=0,$ $\lim\limits_{\delta\rightarrow 0} \rho(\Phi(v_{\delta},v_{\delta}),\Phi_*)=0$.
	где $\Phi_*=\{y|y=\Phi(v,v),v\in \mathbf{W_*}\}$, причем сходимость \textit{равномерная} относительно выбора $\Phi_{\delta}(v,w),P_{\delta}(w)$ и точки $v_{\delta}$ из $\mathbf{W}_{*\delta}$.
	\end{footnotesize}
\end{block}
\end{frame}

\subsection{Часть 2. Экстраградиентный метод с новым стабилизатором $-\alpha_k\|v-w\|^2$}
\begin{frame}
\frametitle{\insertsection}
\framesubtitle{\insertsubsection}
Было предложена
\begin{block}{Теорема о выпуклости и замкнутости $\mathbf{W_*}$}
	\begin{small}
		Пусть $\mathbf{W_0}$ - выпуклое замкнутое множество из $\mathbb{E}^n$, функция $\Phi(v,w), g_i(w)$ при $i=\overline{1,m+s}$ обладают определенными особенностями. Пусть $\mathbf{W_*}\ne\emptyset$. \textbf{Тогда} $\mathbf{W_*}$ - выпукло, замкнуто и задача имеет \textit{единственное} нормальное решение.
	\end{small}
\end{block}
\,\\
Введем функцию Тихонова\\
$T_k(v,w)=\Phi(v,w)+A_kP(w)-\alpha_k\|v-w\|^2,$ где $v,w\in \mathbf{W_0},A_k>0,\alpha_k>0,k=0,1,\ldots$.\\
Пусть вместо точных $\nabla_w\Phi(v,w),\nabla_wP(w)$ известны приближения $\{\nabla_w^k\Phi(v,w)\},\{\nabla_w^kP(w)\}$ такие, что
\begin{footnotesize}
\begin{equation}
\label{[3]-6}
\begin{aligned}
&\|\nabla_w^k\Phi(v,w)-\nabla_w\Phi(v,w)\|\leqslant \delta_k(1+\|v\|+\|w\|),\quad \forall v,w\in \mathbf{W_0},\\
&\|\nabla_w^kP(w)-\nabla_wP(w)\|\leqslant \delta_k(1+\|w\|),\quad\forall w\in \mathbf{W_0},\delta_k>0,k=0,1,\ldots 
\end{aligned}
\end{equation}
\end{footnotesize}
\end{frame}

\begin{frame}
\frametitle{\insertsection}
\framesubtitle{\insertsubsection}
\begin{block}{Результат 2: ``Промежуточная'' теорема}
	\begin{small}
		Пусть выпольнены условия предыдущей теоремы; $\nabla_w\Phi(v,w),\nabla_wg_i(w)$ на $\mathbf{W_0}$, функции $T_k(v,w)$ - выпуклы, выполнено условие \eqref{7}($v_*$ - нормальное решение); параметры удовлетворяют: $p\geqslant v$,$p>1$,$\alpha_k>0$,$A_k>0$,$\alpha_k\to 0$, и $A_k\to +\infty$,$\alpha_kA^{v/(p-v)}_k\to +\infty$. \textbf{Тогда} $z_k$, являющиеся \textit{точками равновесия} функции Тихонова, существуют однозначно $\forall k$, и
	\begin{equation*}
	\begin{aligned}
	&\bm{\lim\limits_{k\to \infty}\|z_k-v_*\|=0},\lim\limits_{k\to \infty}A_kP(z_k)=0,\|z_k\|\leqslant R_k\leqslant\sup\limits_{k\geqslant 0}R_k=R,\\
	& \|z_k-z_m\|\leqslant \frac{|A_m-A_k|R_1}{2\alpha_k},\forall k,m=0,1,..., R_1,R_2,... - \text{ постоянные.}
	\end{aligned}
	\end{equation*}  
	\end{small}
\end{block}
\end{frame}

\begin{frame}
\frametitle{\insertsection}
\framesubtitle{\insertsubsection}
Предлагаемый в данной работе \textbf{регуляризующий экстраградиентный метод}:\\[5mm]

	$u_k=\mathrm{Pr}_{\mathbf{W_0}}(v_k-\beta_k[\nabla_w^k\Phi(v_k,v_k)+A_k\nabla_w^kP(v_k)])$,\\
	$v_{k+1}=\mathrm{Pr}_{\mathbf{W_0}}(v_k-\beta_k[\nabla_w^k\Phi(v_k,u_k)+A_k\nabla_w^kP(u_k)-2\alpha_k(v_k-u_k)])$\\[10mm]

Пусть для параметров выполнены условия:
\begin{equation}
\label{[3]-30}
\begin{aligned}
&\alpha_k >0,A_{k+1}\geqslant A_k>0,\beta_k>0,\delta_k>0,\lim_{k\to \infty}\beta_k=0,\lim_{k\to\infty}\delta_k=0,\\
&\sup_{k\geqslant 0}\beta_k(1+A_k)<\frac{1}{L},\lim_{k\to\infty}\frac{\delta_k+\delta_kA_k}{\alpha_k}=0,\lim_{k\to \infty}\frac{A_{k+1}-A_k}{\alpha_k^2\beta_k}=0.
\end{aligned}
\end{equation}
\end{frame}

\begin{frame}
\frametitle{\insertsection}
\framesubtitle{\insertsubsection}
\begin{block}{Результат 3: Сходимость регуляризованного метода}
	\begin{small}
	
	Пусть выполнены все условия предыдущей теоремы, и пусть $\nabla_w\Phi(v,w),\nabla_wP(w)$ удовлетворяют условию Липлища. Пусть также выполнено модифицированное условие Липшица вида:
	\begin{equation}
	\label{[3]-29-2}
	\|\nabla_w\Phi(v,w)-\nabla_w\Phi(w,w)\|\leqslant L\|v-w\|, \forall v,w\in\mathbf{W_0},L=\mathrm{const}>0.
	\end{equation}
	Вместо точных $\nabla_w\Phi(v,w),\nabla_wP(w)$ известны их приближения $\{\nabla_w^k\Phi(v,w)\},\{\nabla_w^kP(w) \}$, удовлетворяющие условиям \eqref{[3]-6}.\\
	Параметры $\{\alpha_k\},\{\beta_k\},\{\delta_k\},\{A_k\}$ метода удовлетворяют условиям \eqref{[3]-30}.\\
	\textbf{Тогда} последовательность $\{v_k\}$, порожденная методом при любом выборе начального приближения $v_0\in \mathbf{W_0}$, сходится к нормальному решению $v_*$ задачи,	причем сходимость равномерна относительно выбора $\{\nabla_w^k\Phi(v,w)\},\{\nabla_w^kP(w) \}$ из \eqref{[3]-6}.
	\end{small}
\end{block}
\end{frame}
\subsection{Часть 3. Численный расчёт: проверка способности метода}
\begin{frame}
\frametitle{\insertsection}
\framesubtitle{\insertsubsection}
\begin{block}{Результат 4: Тестирование на примерах}
	В основном, надо попробовать сдедать программу, обеспечивающую нахождение точки равновесия.\\[5mm]
	
	Тестовые примеры:
	\begin{equation*}
	\Phi(v,w)=vw,\mathbf{W}=\{w\in \mathbb{E}^1:|w|\leqslant 1 \},\mathbf{W_*}=\{0 \}.
	\end{equation*}
	\begin{equation*}
	\Phi(v,w)=vw^2,\mathbf{W}=\{w\in \mathbb{E}^1:|w|\leqslant 1 \},\mathbf{W_*}=\{0,-1 \}.
	\end{equation*}
	Вычислительный процесс сходится к реальному решению исходной задачи с нужной точностью.
\end{block}

\end{frame}

\section{Заключение}
\begin{frame}
\frametitle{\insertsection}
\begin{center}
	\begin{itemize}[label=\(\bullet\)]
		\item Постановка равновесного программирования, примеры
		\item Существующие методы решения 
		\item Неустойчивость задачи и методы регуляризации
		\item Применение нового стабилизатора
		\begin{itemize}[label=\(\bullet\)]
			\item  Метод стабилизации для неустойчивых равновесных задач с использованием нового стабилизатора
			\item  Регуляризованный экстраградиентный метод с использованием нового стабилизатора
			\item Численная проверка
		\end{itemize} 
	\end{itemize}
\end{center}

\end{frame}

	\begin{frame}
	\vfill
	\centering
	\huge \textbf{Спасибо за внимания}!
	\vfill
\end{frame}

\end{document}
