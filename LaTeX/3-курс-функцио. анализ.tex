\documentclass{article}
\usepackage[T2A]{fontenc}
\usepackage{fontspec}
\setmainfont{CMU Serif}
\usepackage{amsmath}
\usepackage{amssymb}
\usepackage[russian]{babel}
\usepackage{xeCJK}% 调用 xeCJK 宏包
\setCJKmainfont{SimHei}

\begin{document}
\author{Сюй Минчуань}
\title{Обзор функционального анализа}
\maketitle
\tableofcontents
\newpage
\section{Открытые и замкнутые множества на прямой}
\subsection{Открытые и замкнутые множества на прямой. Канторово множество и его свойства. Лек.1 \textbf{Вопрос 1}}
	Множество \textbf{замкнуто}, если оно содержит все свои предельные точки.\\
	Множество \textbf{открыто}, есои все его точки являются внутренними.\\
	Объединение любого числа (в том числе бесконечного числа) открытых множеств - открыто.\\
	Пересечение любого числа (в том числе бесконечного числа) замкнутых множеств - замкнуто.\\
	Объединение бесконечного числа замкнутых множеств - может быть незамкнутым. Для конечного числа - замкнуто.\\
	Пересечение бесконечного числа открытых множеств - может быть неоткрытым. Для конечного числа - открыто.\\
	
	\textbf{Теорема 1.1} Любое открытое множество $E$ на прямой является объединением конечного
	или счётного числа попарно непересекающихся интервалов: $E=\bigcup_{a=1}^{\infty} I_{a}, I_{a}=(a, b)$, где, возможно, $a=-\infty, b=\infty$.\\
	
	\textbf{Канторово множество}: получается из удаления открытого множества $G=\left(\frac{1}{3}, \frac{2}{3}\right) \cup\left(\frac{1}{9}, \frac{2}{9}\right) \cup\left(\frac{7}{9}, \frac{8}{9}\right) \cup \ldots$ из интервала $(0,1))$, причем общая длина удаленных интервалов равна
	$\frac{1}{3}+2 \cdot \frac{1}{9}+4 \cdot \frac{1}{27}+\ldots=\frac{1}{3} \sum_{k=0}^{\infty}\left(\frac{2}{3}\right)^{n}=1$. Оставшееся множество  $K=[0,1] \setminus G$ называется канторовым множеством.
	в множество $K$ входят все точки, в разложении троичной дроби которых нет 1, а только 0 и 2. Следовательно множество $K$ содержит множество, эквивалентное множеству двоичных дробей и имеет мощность континуум.
\section{Измеримые множества на прямой. Мера Лебега}
\subsection{Свойства внешней меры. Измеримость открытого множества и счетного объединения измеримых множеств. Измеримость замкнутого множества, дополнения, разности и счетного пересечения измеримых множеств. Критерий измеримости. Лек. 1-2 \textbf{Вопрос 2}}
	\textbf{Внешней мерой} множества $E$ называется точная нижняя грань по всем покрытиям $s=s(E)$ множества $E$ и обозначается ${|E|}^{*}=\inf _{s(E)} \sigma(s)$.\\
	\textbf{Свойства внешней меры}
	\begin{enumerate}
		\item Если ${E}_{1} \subset {E}_{2}$, то ${|E1|}^{*} \leq {|E2|}^{*}$.
		\item Если $E=\bigcup_{k=1}^{\infty} E_{k}$, то ${|E|}^{*} \leq \sum_{k=0}^{\infty} {|{E}_{k}|}^{*}$.
		\item Если $\rho ({E}_{1},{E}_{2})>0$, то ${|{E}_{1} \bigcup {E}_{2}|}^{*}={|{E}_{1}|}^{*}+{|{E}_{2}|}^{*}$.
		\item Для любого множества $E$ и любого числа $\varepsilon$ существует открытое множество $G$
		и такое, что ${|G|}^{*}<{|E|}^{*}+ \varepsilon$.
	\end{enumerate}

	Множество $E$ называется \textbf{измеримым} (измеримым по Лебегу), если для любого $\varepsilon>0$ существует открытое множество $G$,  содержащее $E$ и такое, что  $|G \setminus E|^{*}<\varepsilon$ .  Внешняя мера измеримого множества $E$ называется \textbf{мерой} множества $E$ и обозначается $|E|$.
	Мера $E$ равна нулю тогда и только тогда, когда $|E|^{*}=0$.\\
	
	\textbf{Теорема 2.1} Любое открытое множество измеримо и его мера равна сумме длин составляющих его попарно непересакающихся интервалов.\\
	\textbf{Теорема 2.2} Объединение конечного или счетного числа измеримых множеств является измеримым множеством.\\
	\textbf{Теорема 2.3} Любое замкнутое множество измеримо.\\
	\textbf{Теорема 2.4} Если $E$ измеримо, то измеримо и его дополнение $CE$.\\
	\textbf{Теорема 2.5} Пересечение конечного или счетного числа измеримых множеств является измеримым множеством.\\
	\textbf{Теорема 2.6} Разность двух измеримых множеств является измеримым множеством.\\
	\textbf{Критерий измеримости} Для того чтобы множество $E$ было измеримо необходимо и достаточно, чтобы для любого $\varepsilon>0$ нашлось замкнутое множество $F$,  содержащееся в $E$ и такое, что  $|E \setminus F|^{*}<\varepsilon$.\\
\subsection{Свойство σ-аддитивности меры. Множества типа $F_{\sigma}$ и $G_{\delta}$. Мера Лебега-Стилтьеса. Общее понятие меры. Пример неизмеримого множества. Лек.2-3 \textbf{Вопрос 3}}
	\textbf{Теорема 3.1 $\sigma$ - аддитивность}\footnote{Эта теорема в частности доказывает, что множество рациональных чисел имеет меру нуль, так как это множество можно рассматриваться как объединение счетного числа попарно непересекающихся множеств(точек), у которых меры нуль, и в силу $\sigma$ - аддитивность Мера этого множества равна нулю.} Мера объединения конечного или счетного числа попрано непересекающихся множеств равна сумме мер этих множеств.\\
	Mножество $E$ называется множеством \textbf{типа $G_{\delta}$}\footnote{Множество иррациональных чисел - $G_{\delta}$ Пусть $X(q)$ - это $R \setminus {q}$, где $q$ - это одно рациональное число (т.е. вся прямая с одной выколотой точкой).  Рациональных чисел - счетное количество. Множество иррациональных чисел $I$ - это все числа, кроме рациональных. Это множество можно представить в виде $I = \cap_{q} X(q)$. Причем каждое $X(q)$ - открытое (т.е. это дополнение одной точки, т.е. замкнутого множества). Получили определение множества ${G}_{\delta}$.},  если оно представимо в виде пересечения счетного числа открытых множеств. \\
	Множество $E$ называется множеством \textbf{типа $F_{\sigma}$}\footnote{Множество рациональных чисел - $F_{\sigma}$ Т.к. каждое рациональное число само образует замкнутое множество, и объединение всех рациональных чисел естественно является $F_{\sigma}$},  если оно представимо в виде объединения счетного числа замкнутых множеств.\\
	\textbf{Теорема 3.2} Если множество Е измеримо, то существуют множество ${E}_{1}$ типа $G_{\delta}$ и множество ${E}_{2}$ типа $F_{\sigma}$ и такие, что ${E}_{2}\subset{E}\subset{E}_{1}$, причем $|{E}_{1}|=|{E}_{2}|=|E|$.\\
	\textbf{Пример неизмеримого множества}\footnote{微信-收藏-和qc的聊天 c1 c2 c3就是各个класс, 每个класс里面的点都相差pi*α(比如x11点转动piα得到x12,再转动piα得到x13)。Ф0是在每个класс中选一个点组成的,那么比如我们可以选择对角线的这样的元素作为Ф0, 那么对应的Ф1就是所有的点转动pi*α。所有Фn的并集是整个C而且互不相交。根据аддитивность |C|=1 =$\Sigma$|Фn|。但是Фn都是一样的 所以|Ф0|=|Ф1|=...=|Фn|。但这样的话|C|=1 =$\Sigma$|Фn|就不可能成立了.}\\
	Пусть $C$ - окружность, длина которой равна 1, $\alpha$-  некоторое иррациональное число. Отнесем к одному классу те точки окружности $C$,  которые могут быть переведены одна в другую поворотом окружности $C$ на угол  $\pi n\alpha $(n-  целое). Каждый из этих классов будет состоять из счетного числа точек. Выберем из каждого такого класса по одной точке. 
	
	Покажем, что полученное таким образом множество (обозначим его  $\Phi_{0}$)  неизмеримо. Обозначим через  $\Phi_{n}$  множество, получаемое из $\Phi_{0}$ поворотом на угол $\pi n\alpha $. Легко видеть, что все множества  $\Phi_{n}$ попарно не пересекаются и их объединенем является окружность $С$. Если бы множество $\Phi_{0}$ было измеримым, то были бы измеримы и все конгруентные ему множества $\Phi_{n}$. Так как$C=\cup_{n=-\infty}^{\infty} \Phi_{n}, \quad \Phi_{n} \cap \Phi_{m}=\emptyset, \quad n \neq m$, то в силу $\sigma$  -аддитивности меры отсюда следовало бы, что $1=\sum_{n=-\infty}^{\infty}\left|\Phi_{n}\right|$.	Но конгруентные множества должны иметь одну и ту же меру, так что если множество $\Phi_{0}$ измеримо, то $\left|\Phi_{n}\right|=\left|\Phi_{0}\right|$. Отсюда видно, что равенство для ряда невозможно, так как сумма ряда равна нулю, если $\left|\Phi_{n}\right|=0$,  и равна бесконечности, если $\left|\Phi_{n}\right|>0$. Итак, множество $\Phi_{0}$(а, следовательно, и каждое $\Phi_{n}$)  неизмеримо. Из этого примера следует также неаддитивность внешней меры.\\
	\\
	
	\textbf{Мера Лебега-Стилтьеса}\\
	Пусть $F(t)$ -  некоторая неубываюшдя, непрерывная слева функция на прямой. Положим
	\begin{eqnarray*}
		m(a, b)=F(b)-F(a+0), \quad m[a, b]=F(b+0)-F(a) \\
		m(a, b]=F(b+0)-F(a+0), \quad m[a, b)=F(b)-F(a)
	\end{eqnarray*}
  	Это мера называется \textbf{мерой Лебега-Стилтьеса}.
	
	\textbf{Общее понятие меры}\\
	Непустая система множеста $К$ иазывается \textbf{кольцом}, если она обладает тем свойством, что из $A\in K$ и $B \in K$  следует $A \triangle B \in K$ и $A \cap B \in K$. \\
	Кольцо множеств есть система множеств, замкнутое по отношению к взятию суммы и пересечения, вычитания и образования симметричной разности. \\
	Система состоящая только из пустого множества есть \textbf{наименьшее возможное кольцо} множеств.\\
	Множество Е называется \textbf{единицей} системы множеств $K$, если $E\subset K$ и если для любого $A\in K$ имеет $A\cap E = A$.\\
	Кольцо множеств с единицей называется \textbf{алгеброй множеств}.\\
	Для любого непустой системы множеств $\Sigma$ существует одно и только одно кольцо $R(\Sigma)$, содержащее $\Sigma$ и содержащееся в любом кольце $R$, содержащее $\Sigma$. Это кольцо называется \textbf{минимальным кольцом} и обозначается $R(\Sigma)$.\\
	Система множеств называется \textbf{полукольцом}, если она содержит пустое множество, замкнута по отношению к преобразованию пересечений и обладает тем же свойством, что из ${A}_{1}\subset A\in K$ вытекает возможности представления в виде $A=\bigcup_{k=1}^{n} {A}_{k}$, где ${A}_{k}$ - попарно непересекающихся множеств из $K$,первое из которых - заданное ${A}_{1}$.\\
	Функция множества $\mu (A)$ называется \textbf{мерой},
	если\\
	1) область определения $\Sigma_{\mu}$ функции $\mu (A)$  есть полукольцо множеств,\\
	2) значения функции $\mu (A)$ действительны и неотрицательны,\\
	3) $\mu (A)$ аддитивна, т.е. Для любого конечного разложения $A=A_{1} \cup A_{2} \cup \ldots \cup A_{n}$ множества $A\subset \Sigma_{\mu}$  на попарно непересекающихся множества $A_{n} \subset   \Sigma_{\mu}$  выполнено равенство $\mu(A)=\sum_{k=1}^{n} \mu\left(A_{k}\right)$.\\
	Мера $\mu$ называется \textbf{продолжением меры} $m$,  если $\Sigma_{m} \subset \Sigma_{\mu}$ и для каждого $A \subset \Sigma_{m}$ имеет место разложение $\mu (A)=m(A)$.\\
\section{Измеримые функции}
\subsection{Измеримые функции и их свойства. Измеримость верхнего и нижнего пределов последовательности измеримых функций. Лек.5 \textbf{Вопрос 4}}
	Функция $f(x)$, определённая на измеримом множестве $E$ называется \textbf{измеримой} на нём, если для любого числа $a$ множество $E[f\ge a]$\footnote{$E[f(x)\ge a]={x\in E:f(x)\ge a}$, причем это условие равносильно $E[f(x)\le a]$, $E[f(x)<a]$, $E[f(x)>a]$.} измеримо.\\
	Если функция $f(x)$ измерима на $E$, то она измерима и на любом измеримом подмножестве $E_{1}$ множества $E$.\\
	Если $E=\bigcup_{n=1}^{\infty} E_{n}$, $E_{n}$ - измеримы и $f(x)$ измерима на каждом множестве $E_{n}$, то она измерима и на всем $E$. \\
	Любая функция измерима на множестве $E$ меры нуль.\\
	\\
	Две заданные на измеримом множестве $E$ функции $f(x)$ и $g(x)$ называются \textbf{эквивалентными}, если множество $E[f(x)\ne g(x)]$ имеет меру нуль. Обозначают
	эквивалентность функции формулой $f(x)\sim g(x)$.\\
	Если $f(x)\sim g(x)$ на $E$ и функция $f(x)$ измерима на $E$, то и функция $g(x)$ измерима на $E$.\\
	\\
	Говорят, что некоторое свойство выполняется \textbf{почти всюду} на измеримом множестве $E$, если оно не выполняется на подмножестве множества $E$ меры нуль.\\
	Если функция $f(x)$ непрерывна почти всюду\footnote{Так, функция Дирихле, определяемая на отрезке $[0,1]$ и
	принимающая значение 1 в рациональных точках и значение 0 в иррациональных, не является непрерывной ни в одной точке, однако она эквивалентна функции, тождественно равной нулю на $[0,1]$.} на измеримом множестве $E$, то $f(x)$ измерима на $E$.\\
	\\
	\textbf{Теорема 4.1} Если функция f(x) измерима на $E$, то $|f(x)|$ также измерима на нём. Если $C=const$ , то $f(x)+C$ и $C\cdot f(x)$ измеримы на $E$. Если $f(x) $ и $g(x)$ измеримы на $E$, то	и множество $E[f>g]$ - измеримо.\\
	\textbf{Теорема 4.2} Если функции $f(x)$ и $g(x)$ измеримы на измеримом множестве $E$, то и функции $f\pm g$, $f\cdot g$ и $\frac{f}{g} (g\ne 0)$ измеримы на $E$.\\
	\textbf{Теорема 4.3} Если ${{f}_{n}(x)}$ – последовательность измеримых на $E$ функций, то
	нижний и верхний пределы этой последовательности – измеримы на $E$ (ограниченность функций не предполагается).\\
	
	
\subsection{Измеримость предела сходящейся почти всюду последовательности измеримых функций. Сходимость по мере. Теорема Лебега. Связь между сходимостью почти всюду и сходимостью по мере. Лек.5 \textbf{Вопрос 5}}
	\textbf{Теорема 5.1} Если ${{f}_{n}(x)}$ - последовательность измеримых на $E$ функций, сходится почти всюду на $E$ к функции $f(x)$, то $f(x)$ измерима на $E$.\\
	\\
	Пусть функции ${{f}_{n}(x)}, n=1,2,3,\ldots$, и $f(x)$ измеримы на $E$ и принимают почти всюду на $E$ \textbf{конечные значения}. Говорят, что последовательность ${{f}_{n}(x)}$ сходится к $f(x)$ \textbf{по мере} на $E$, если для любого $\varepsilon$ выполняется $\lim _{n \rightarrow \infty}\left(E\left[\left|f_{n}-f\right| \geq \varepsilon\right]\right)=0$, т.е. для любого $\varepsilon>0$ и любого $\delta>0$ существует номер $N=N(\varepsilon,\delta)$ такой, что для всех номеров $n\ge N$ справедливо равенство $|E\left[\left|f_{n}-f\right| \geq \varepsilon\right]|<\delta$.\\
	\textbf{Теорема 5.2 Теорема Лебега} Пусть $E$ - измеримое множество конечной меры и пусть функции ${{f}_{n}(x)}$ и $f(x)$ измеримы на $E$ и почти всюду конечны. Тогда из
	сходимости последовательности ${{f}_{n}(x)}$ к $f(x)$ почти всюду на $E$ следует сходимость ${{f}_{n}(x)}$ к $f(x)$ по мере.\\
	Из сходимости ${{f}_{n}(x)}\rightarrow f(x)$ по мере не следует сходимость последовательности ${{f}_{n}(x)}$ к функции $f(x)$ не только почти всюду, но даже сходимость хотя бы в одной точке.\footnote{см. замечание на лек.5 стр.4}\\
\subsection{Теорема Рисса. Эквивалентность функций. Лек.6 \textbf{Вопрос 6}}
	\textbf{Теорема 6.1 Теорема Рисса} Пусть $E$ - измеримое множество конечной меры и пусть функции ${{f}_{n}(x)}$ и $f(x)$ - измеримы и почти всюду конечны на $E$, то из последовательности ${{f}_{n}(x)}$ можно выделить подпоследовательность, сходящуюся к $f(x)$
	почти всюду на $E$.\\
	\textbf{Теорема 6.2} Пусть E - множество конечной меры\footnote{это условие может быть не существенным} и пусть функции ${{f}_{n}(x)}$, $f(x)$ и $g(x)$ почти всюду конечны. Тогда, если ${{f}_{n}(x)}\rightarrow f(x)$ и ${{f}_{n}(x)}\rightarrow g(x)$ по мере одновременно, то $f(x)\sim g(x)$.\\
\section{Интеграл Лебега}
\subsection{Интеграл Лебега от ограниченной функции. Интегрируемость ограниченной и измеримой функции. Лек.6 \textbf{Вопрос 7}}
	Пусть $E$ - измеримое множество и $|E|<+\infty$. Назовём \textbf{разбиением} $Е$ любое ceмействo  $T=\left\{E_{k}\right\}_{k=1}^{n}=\left\{E_{1}, E_{2}, \ldots, E_{n}\right\}$ конечного числа измеримых, попарно не
	пересекающихся множеств и таких, что $E=\bigcup_{k=1}^{n} E_{k}$.\\
	Пусть $f(x)$ - произвольная ограниченная на множестве $E$ функция. Для любого paзбиения $T$ обозначим $M_{k}=\sup _{x \in E_{k}} f(x), m_{k}=\inf _{x \in E_{k}} f(x)$ и рассмотрим суммы $S_{T}=\sum_{k=1}^{n} M_{k}\left|E_{k}\right|$, $s_{T}=\sum_{k=1}^{n} m_{k}\left|E_{k}\right|$,  которые называются \textbf{верхней и нижней суммами Лебега} разбиения $T$.\\ Ясно, что $s_{T}\leq S_{T}$ и эти суммы ограничены. Поэтому сушествуют  $\bar{I}=\inf _{T} S_{T}$ и $\underline{I}=\sup _{T} s_{T}$, называемые \textbf{верхним и нижним интегралами Лебега} функции f(x) соответственно.\\
	\\
	Ограниченная на множестве конечной меры $E$ функция $f(x)$
	называется \textbf{интегрируемой (по Лебегу)} на $E$, если $\bar{I}=\underline{I}$, т.е. её верхний и нижний интеграл Лебега совпадают.\\
	Общее значение $I=\bar{I}=\underline{I}$ называется \textbf{интегралом Лебега} от $f(x)$ по множеству $E$ и обозначается $I=\int_{E} f(x)dx$.\\
	
	\textbf{Свойства сумм и интегралов:}\\
	1. Если разбиение ${T}^{*}$ является измельчением $T$, то ${s}_{T}\le {s}_{{T}^{*}}, {S}_{{T}^{*}}\le {S}_{T}$.\\
	2. Для любых двух разбиений ${T}_{1}$ и ${T}_{2}$ справедливо неравенство ${s}_{{T}_{1}}\le {S}_{{T}_{2}}$.\\
	3. Имеет место также соотношение $\underline{I}\le \bar{I}$.\\
	\\
	\textbf{Теорема 7.1} Всякая интегрируемая по Риману функция $f(x)$ является
	интегрируемой по Лебегу, причём её интегралы Римана и Лебега совпадают.\\
	\textbf{Теорема 7.2} Всякая ограниченная и измеримая на измеримом множестве конечной меры функция интегрируема на нём по Лебегу.\\
\subsection{Свойства интеграла Лебега от ограниченной функции. Лек.7 \textbf{Вопрос 8}}
	\begin{enumerate}
	\item  $\int_{E} 1 d x=|E|$. Ecли $f(x) \equiv 1$ на $E$, то $s_{T}=S_{T}=|E|$. 
	\item Eсли $f(x)$ ограничена, интегрируема на $E$, $|E|<+\infty$, и $\alpha$ - любое действительное число, то функция $\alpha f(x)$ также интегрируема на $E$ и $\int_{E} \alpha f(x)dx=\alpha \int_{E} f(x)dx$. 
	\item Если функции $f_{1}(x)$ и $f_{2}(x)$ ограничены и интегрируемы на $E$, $|E|<+\infty$, то их сумма интегрируема на множестве $Е$ и $\int_{E}\left(f_{1}(x)+f_{2}(x)\right)dx=\int_{E} f_{1}(x) d x+\int_{E} f_{2}(x)dx$.
	\item Если $f(x)$ ограничена и интегрируема на каждом из непересекаюшихся множеств конечной меры $E_{1}$ и $E_{2}$, то $f(x)$ интегрируема и на  $E=E_{1} \cup E_{2}$  и  $\int_{E} f(x)dx=\int_{E_{1}} f(x)dx+\int_{E_{2}} f(x)dx$.
	\item Если функции $f_{1}(x)$ и $f_{2}(x)$ ограничены и интегрируемы на $E$, $|E|<+\infty$, и $f_{1}(x) \geq f_{2}(x)$ всюду (почти всюду) на $E$, то $\int_{E} f_{1}(x)dx \geq \int_{E} f_{2}(x)dx$.
	\end{enumerate}
\subsection{Интеграл Лебега от неограниченной и неотрицательной функции. Полная аддитивность и абсолютная непрерывность интеграла Лебега. Мажорантный признак суммируемости. Лек.7 \textbf{Вопрос 9}}
	Пусть $f(x)\geq 0$ всюду на $E$, $|E|<+\infty$, $f(x)$  - измерима и, возможно, не ограничена. Для любого числа $N>0$ положим $f_{N}(x)=\min \{N, f(x)\}$ - \textbf{срезка} функции $f(x)$. Функция $f_{N}(x)$ также измерима на $E$:
	\begin{equation}
	E\left[f_{N}>a\right]=\left\{\begin{array}{c}
	E[f>a], \text{пpи} \text{ } a<N \\
	\varnothing, \qquad \quad \text{пpи} \text{ } a \geq N
	\end{array}\right.
	\end{equation}
	Причём срезка $f_{N}(x)$ - ограничена. Поэтому существует интеграл $I_{N}(f)=\int_{E} f_{N}(x)dx$.\\
	\\
	Если при $N \rightarrow \infty$ существует $\lim _{N \rightarrow \infty} I_{N}(f)<+\infty$, то $f(x)$ называется \textbf{суммируемой} на множестве $Е$, а этот предел называется \textbf{интегралом} от $f(x)$ на $Е$ и обозначается $\int_{E} f(x) dx=I(f)$.\\
	Если $f(x)\ge 0$ и суммируема на $E$, то $f(x)=+\infty$ лишь на подмножестве множества $E$ меры нуль.\\
	\\
	\textbf{Теорема 9.1 Полная аддитивность интервала Лебега} Пусть $E=\bigcup_{k=1}^{\infty} {E}_{k}, |E|<+\infty, {E}_{i}\cap {E}_{j}=\varnothing, i\ne j$, все множества ${E}_{k}$ - измеримы. Тогда \\
	1) если $f(x)\ge 0$ и $f(x)$ суммируема на $E$, то $f(x)$ суммируема на ${E}_{k}$, причем
	\begin{equation}
	\label{8.1}
	\int_{E} f(x) d x=\sum_{k=1}^{\infty} \int_{E_{k}} f(x)dx
	\end{equation}
	2) если $f(x)\ge 0$, $f(x)$ суммируема на всех множествах ${E}_{k}$ и ряд в формуле \ref{8.1} сходится, то $f(x)$ суммируема на $E$ и справедливо равенство \ref{8.1}.\\
	
	\textbf{Теорема 9.2 Абсолютная непрерывность интергала Лебега}
	Если $f(x)\ge 0$ и суммируема на $E$, $|E|<+\infty$, то для любого $\varepsilon >0$ существует $\delta >0$ такое, что каково бы ни
	было измеримое подмножество $e\subset E$ с мерой $|e|<\delta$ справедливо неравенство $\int_{e} f(x)dx<\varepsilon$.\\
	
	\textbf{Теорема 9.3} Если $f(x)\ge 0$, суммируема на $E$, где $|E|<+\infty$ и $\int_{E} f(x)dx=0$, то $f(x)\sim 0$.\\
	
	\textbf{Теорема 9.4 Мажорантный признак суммируемости} Если ${f}_{1}(x)\ge 0$ - измерима на $E$, $|E|<+\infty$,
	${f}_{2}(x)$ - суммируема на $E$ и, если всюду на $E$ выполняется неравенство ${f}_{1}(x)\le {f}_{2}(x)$, то ${f}_{1}(x)$ - суммируема на $E$ и справедлива оценка $\int_{E} {f}_{1}(x)dx\le \int_{E} {f}_{2}(x)dx$.\\
\subsection{Интеграл Лебега от неограниченной функции любого знака. Теорема Лебега о предельном переходе под знаком интеграла. Лек.8 \textbf{Вопрос 10}}
	Пусть $|E|<+\infty$, $f(x)$ - измеримая функция на $Е$. Введём в рассмотрение две функции $f^{+}(x)=\frac{1}{2}(|f(x)|+f(x))$ и  $f^{-}(x)=\frac{1}{2}(|f(x)|-f(x))$, которые также измеримы на $E$.  Справедливы равенства $f^{+}(x)+f^{-}(x)=|f(x)|$,  $f^{+}(x)-f^{-}(x)=f(x)$.\\
	\\
	Измеримая функция $f(x)$ называется \textbf{суммируемой} на $E$, $|E|<+\infty$, если на $Е$ суммируемы обе неотрицательные функции  $f^{+}(x)$ и $f^{-}(x)$. При этом \textbf{интегралом Лебега} от $f(x)$ называется разность $\int_{E} f(x)dx=\int_{E} f^{+}(x)dx-\int_{E} f^{-}(x)dx$.\\
	Для интеграла Лебега суммируемость $f(x)$ эквивалентна суммируемости функции $f(x)$.\\
	Во втором утверждении теоремы о полной аддитивности надо требовать сходимость ряда $\sum_{k=1}^{\infty} \int_{E_{k}} |f(x)|dx$. В теореме об абсолютной непрерывности выполняется неравенство $\int_{E} |f(x)|dx<\varepsilon$.\\
	\\
	Совокупность всех суммируемых на измеримом множестве $E$ функций обозначается $L(E)\equiv L_{1}(E)$. Говорят, что последовательность  $f_{n}(x)\in L(E)$ сходится в $L(E)$ (\textbf{сходится в среднем}) к $f(x)\in L(E)$, если $\lim _{n \rightarrow \infty}\int_{E}\left|f_{n}(x)-f(x)\right|dx=0$.\\ 
	Имеет место равенства $\lim _{n \rightarrow \infty}\int_{E}f_{n}(x)dx=\int_{E}f(x)dx$.\\
	Если ${f_{n}(x)}$ сходится в $L(E)$ к $f(x)\in L(E)$, то эта
	последовательность сходится к $f(x)$ и по мере на $E$. Обратное утверждение вообще говоря неверно.\footnote{см. лек.8 стр.2}\\
	\\
	\textbf{Теорема 10.1 Теорема Лебега о предельном переходе под знаком интеграла}
	Пусть функции $f_{n}(x)$ почти всюду конечны и измеримы на измеримом множестве $E$ конечной меры и эта последовательность сходится
	по мере к измеримой и почти всюду конечной функции $f(x)$. Если существует суммируемая на множестве $E$ функция $F(x)$, такая что для всех номеров $n$ и почти всех точек $x\in E$, справедливо неравенство $|f_{n}(x)|\le F(x)$, то последовательность ${f_{n}(x)}$ cходится к $f(x)$ в среднем т.е. в $L(E)$.\\
\subsection{Теорема Леви и следствие из нее для рядов. Теорема Фату. Теорема Лебега — критерий интегрируемости. Лек.8-9 \textbf{Вопрос 11}}
	\textbf{Теорема 11.1 Теорема Леви} Пусть $f_{n}(x)$ - суммируемые на множестве $E$ функции, $|E|<+\infty$, и пусть для любого натурального числа $n$ выполняется неравенство ${f}_{n}(x)\le {f}_{n+1}(x)$ для почти всех $x\in E$. Если существует постоянная $M$ такая, что для
	любого натурального числа $n$ выполняется неравенство $|\int_{E} {f}_{n}(x)dx|\le M$, то последовательность ${f_{n}(x)}$ сходится почти всюду на $E$ к некоторой функции $f(x)$, причем $f(x)\in L(E)$ и $\lim _{n \rightarrow \infty}\int_{E}f_{n}(x)dx=\int_{E}f(x)dx$.\\
	\textbf{Следствие(для функциональных рядов)} Если все функции $u_{n}(x)\geq 0$ почти всюду на $E$, суммируемы на $E$ и если сходится ряд $\sum_{n=1}^{\infty} \int_{E} u_{n}(x)dx$, то почти всюду на $E$ cходится ряд $\sum_{n=1}^{\infty} u_{n}(x)$, причем сумма $S(x)$ ряда суммируема на $E$ и $\int_{E} S(x)dx=\sum_{n=1}^{\infty} \int_{E} u_{n}(x)dx$. Т.е. ряд можно интегрировать почленно. \\
	Здесь в качестве $f_{n}(x)$ берем частичную сумму $S_{n}(x)=\sum_{k=1}^{n} u_{k}(x)$.\\ 
	
	\textbf{Теорема 11.2 Теорема Фату} Если последовательность ${f_{n}(x)}$ измеримых и суммируемых на измеримом множестве $E$ функций, сходится п.в. на $E$ к предельной функции $f(x)$ и если сушествует постоянная $A$ такая, что для всех номеров $n$ выполняется неравенство $\int_{E}\left|f_{n}(x)\right| d x \leqslant A$, то функция $f(x)$ суммируема на $E$ и справедливо нepaвeнство $\int_{E}|f(x)|dx\leqslant A$.\\ 
	
	\textbf{Теорема 11.3 Теорема Лебега о критерии интегрируемости} Для того чтобы ограниченная на измеримом множестве конечной меры $E$ функция $f(x)$ являлась интегрируемой по Лебегу на $E$ необходимо и достаточно, чтобы $f(x)$ была измерима на $E$.\\
\subsection{Теорема Фубини. Интеграл Лебега для множества бесконечной меры. Лек.9 \textbf{Вопрос 12}}
	\textbf{Теорема 12.1 Теорема Фубини} Пусть функция $f(x,y)$  интегрируема по Лебегу на прямоугольнике $P=\{(x,y): a<x<b, c<y<d\}$, тогда для п.в. $y\in (c,d)$ сушествует интеграл $J(y)=\int_{a}^{b} f(x,y)dx$ и для п.в. $x\in (a,b)$ сушествует интеграл $I(x)=   \int_{c}^{d} f(x,y)dy$, причем функция $J(y)$ интегрируема на $(c,d)$, а функция $I(x)$ интегрируема на $(a,b)$ и справедливо равенство
	\begin{equation}
	\iint_{P} f(x,y)dxdy=\int_{c}^{d} J(y)dy=\int_{a}^{b}I(x)dx
	\end{equation}
	Обратное утверждение вообще говоря неверно.\footnote{см. лек.9 стр.4}\\
	\\
	Пусть измеримое множество $E$ имеет бесконечную меру $|E|=+\infty$. Рассмотрим $E=\cup_{n=1}^{\infty} E_{n}$, где $E_{n}$ измеримы и $|{E}_{n}|<+\infty$. Последовательность $\{{E}_{n}\}$ называется \textbf{исчерпываюшей} для $E$. Измеримая функция $f(x)$ на измеримом множестве $E$ бесконечной меры называется \textbf{суммируемой} на этом множестве, если она суммируема на любом измеримом подмножестве конечной меры $E_{n}\subset E$ и сушествует конечный предел $\lim _{n \rightarrow \infty} \int_{\cup_{k=1}^{n} E_{k}} f(x) d x=\int_{E} f(x)dx$, независимо от выбора исчерпываюшей последовательности. Все основные свойства интеграла Лебега сохраняются.
\section{Пространство ${L}_{p}$}	
\subsection{Классы ${L}_{p}, p>1$. Неравенства Гельдера и Минковского. Лек.10 \textbf{Вопрос 15}}
	Пусть $E$ - измеримое множество, число $p\ge 1$. Множество всех измеримых на $E$ функций $f(x)$, для которых
	функции ${|f(x)|}^{p}$ суммируемы на $E$, называется \textbf{пространством ${L}_{p}(E)$}.
	Норма в пространстве ${L}_{p}(E)$ вводится по формуле ${||f(x)||}_{{L}_{p}(E)}={||f(x)||}_{p}={(\int_{E} {|f(x)|}^{p}dx)}^{1/p}$.\\
	
	\textbf{Неравенство Гельдера}\footnote{Доказательство см. лек.10 стр.1}\\
	Если $p>1$, число $q$ связано с числом $p$ по формуле $\frac{1}{p}+\frac{1}{q}=1$, $f(x)\in {L}_{p}(E)$, $g(x)\in {L}_{q}(E)$, то функция $f(x)g(x)$ суммируема на $E$ и справедливо
	\begin{equation}
	\int_{E} |f(x)g(x)|dx\le {||f(x)||}_{{L}_{p}(E)} {||g(x)||}_{{L}_{q}(E)}
	\end{equation}
	\textbf{Неравенство Минковского}\footnote{Это и непосредственно проверяет справедливость третьего аксиома нормы}\\
	Если $p\ge 1$, $f(x),g(x)\in {L}_{p}(E)$, то функция ${|f(x)+g(x)|}^{p}$ суммируема на $E$ и справедливо
	\begin{equation}
	{||f(x)+g(x)||}_{{L}_{p}(E)}\le {||f(x)||}_{{L}_{p}(E)}+{||g(x)||}_{{L}_{p}(E)}
	\end{equation}
\subsection{Полнота пространства ${L}_{p}$. Лек.10 \textbf{Вопрос 16}}
	Последовательность $\{{f}_{n}\}$ элементов нормированного пространства называется \textbf{фундаментальной}, если числовая последовательность $||{f}_{m}-{f}_{n}||$ стремится к нулю при $m,n\rightarrow 0$.\\ 
	Последовательность $\{{f}_{n}\}$ элементов нормированного пространства называется \textbf{сходящейся}, если в этом пространстве существует элемент $f$ такой, что $\lim_{n \rightarrow \infty} ||{f}_{n}-f||=0$.\\
	Нормированное пространство называется \textbf{полным(банаховым)}\footnote{Легко доказать что всякая сходящаяся последовательность всегда является фунтаментальной: $||{f}_{n}-f||<=||{f}_{m}-f||+||f-{f}_{n}||$\\
	Существует пример фун. после-сти, которая не является сходящейся:
	Пусть ${||f||}_{C[a,b]}=max|f(x)|$, где $f(x)$ - непрерывная функция на $[a,b]$.	Рассмотрим множество многочленов ${P}_{n}(x)$, которое является подмножество множества непрерывных функций и обладает тем же нормой. Теорема Вейрштрасса: Для любого $\varepsilon$ и $f(x)$ из $C[a,b]$ существует ${P}_{n}(x)$: $||f(x)-{P}_{n}(x)||<\varepsilon$. Возьмем последовательность $\{{P}_{n}(x)\}$ такое что $\{{P}_{n}(x)\}\rightarrow f(x)$, и $f(x)$ не является многочленом. Поэтому $\{{P}_{n}(x)\}$ фунтаментальна но не сходится.}, если
	любая фундаментальная последовательность в этом пространстве является сходящийся.\\
	\textbf{Теорема 16.1 Полнота пространства ${L}_{p}$}\footnote{Интеграл Римана здесь не обеспечивает полнату пространства, так как сущестсует несходящаяся фунтаментальная последовательность.} Пространство ${L}_{p}(E)$, $|E|<+\infty$, $p\ge 1$, является полным(банаховым)
	пространством.\\
	
\subsection{Плотность множества непрерывных функций в ${L}_{p}$. Непрерывность в метрике ${L}_{p}$. Лек.11 \textbf{Вопрос 17}}
	Измеримая функция $F(x)$ на измеримом множестве $Е$ называется \textbf{простой}\footnote{Пример: $f(x)=sgn(x)$, $f(x)=[x]$\\
	Ещё пример: $E=(0,+\infty)$, ${E}_{k}={x:k-1<x\le k}, k=1,2,3\ldots$, $f(x)= 1/\sqrt{k},x\in {E}_{k}$. Принадлежности $f(x)\in {L}_{p}$:
	$\int_{E} {|f(x)|}^{p}dx=\sum_{k=1}^{+\infty} {|{C}_{k}|}^{p}|{E}_{k}|=\sum_{k=1}^{+\infty} 1/{k}^{p/2}
	\Rightarrow  f(x)\in {L}_{p}, \text{при}\,p>2.$}, если она принимает не более чем счетное число различных значений $f(x)=C_{k}$, при $x \in {E}_{k}$, причем ${C}_{k}$ может быть равным $\pm \infty$.\\
	\textbf{Характеристической функцией} множества $E$ называется функция
	\begin{equation}
	\chi_{E}(x)=\left\{\begin{array}{l}1, x\in E\\ 0, x\notin E\end{array}\right. 
	\end{equation}
	Очевидно, что всякая простая функция $f(x)$ имеет вид  $f(x)=\sum_{k=1}^{\infty} {C}_{k}\chi_{{E}_{n}}(x)$,причем в этой
	сумме при каждом $х$ отлично от нуля лишь одно слагаемое.\\
	Ясно, что функция $\chi_{E}(x)$ измерима тогда и только тогда, когда множество $E$ - измеримо.\\
	\textbf{Лемма 17.1} Для любой на измеримом множестве $E$ неотрицательной измеримой функции $f(x)$ существует неубывающая последовательность ${{f}_{n}(x)}$ простых
	неотрицательных функций ${f}_{n}(x)$ таких, что $\lim_{n \rightarrow \infty} {f}_{n}(x)=f(x)$ в каждой точке $x$ множества $E$, причем сходимость равномерная на множестве конечных значений функции $f(x)$.\\
	\textbf{Следствие к Лемме 17.1} Последовательность $\{{g}_{n}(x)\}$, в которой функции ${g}_{n}(x)$ определяются по фopмyлe  ${g}_{n}(x)=\left\{\begin{array}{c}f_{n}(x), f_{n}(x) \leq n, \\ n, f_{n}(x)>n,\end{array}\right.$ обладает свойством: $\lim_{n \rightarrow \infty} {g}_{n}(x)=f(x)$ для любой точки $x\in E$, ${g}_{n}(x)=\sum_{k=1}^{m(n)} C_{k} \chi_{E_{k}}(x)$, но равномерной сходимости на $E\setminus {E}_{0}$ может и не быть.\\
	\\
	\textbf{Теорема 17.1}\footnote{$X$ - линейное нормированное пространство , $M\subset X$. $M$ - \textbf{всюду плотное} множество $X$, если для любого $\varepsilon>0$, для любого $f$ из $X$, существует $g$ из $M$: $||f-g||<\varepsilon$.
	
	Пример: множество многочленов, принадлежащее множеству непрерывных функций.} Пусть $E$ - ограниченное измеримое множество, $p\ge 1$. Тогда пространство непрерывных на $E$ функций $C(E)$ плотно в ${L}_{p}(E)$.\\
	\textbf{Теорема 17.2 Непрерывность в метрике ${L}_{p}$} Пусть $E$ - ограниченное измеримоe множество, $p\ge 1$. Тогда любая функция $f(x)\in {L}_{p}(E)$ непрерывна в метрике ${L}_{p}$, то есть для
	любого $\varepsilon>0$ найдется число $\delta>0$ такое, что справедливо неравенство ${||f(x+h)-f(x)||}_{p}\le \varepsilon$, если $|h|<\delta$, а функция $f(x)$ считается продолженной нулем на все
	пространство ${R}_{n}$.\\
\section{Метрические и нормированные пространства}
\subsection{Метрические пространства. Теорема о вложенных шарах. 	Лек.13-14 \textbf{Вопрос 13}}
	Множество $M$ называется \textbf{метрическим пространством}\footnote{пример см. лек.13 стр.1}, если каждой паре его элементов $x$ и $y$ поставлено в соответствии неотрицательное число $\rho(x,y)$, удовлетворяющее условиям:\\
	1) $\rho(x,y)=0$ тогда и только тогда, когда $x=y$ (\textbf{аксиома тождества})\\
	2) $\rho(x,y)=\rho(y,x)$ (\textbf{аксиома симметрии})\\
	3) $\rho(x,z)\le \rho(x,y)+\rho(y,z)$ (\textbf{аксиома треугольника})\\
	\\
	Элемент $x$ метрического пространства $V$ называется \textbf{пределом} последовательности элементов ${x}_{1},{x}_{2},\ldots,{x}_{n},\ldots$, если $\rho({x}_{n},x)\rightarrow0$ при $n\rightarrow +\infty$. Последовательность элементов ${x}_{1},{x}_{2},\ldots,{x}_{n},\ldots$ из $M$ называется \textbf{фунтаментальной}, если $\rho({x}_{n},{x}_{m})\rightarrow0$ при $n,m\rightarrow +\infty$.\\
	\textbf{Утверждение 13.1} Если последовательность точек $\{{x}_{n}\}$ метрического пространства $M$ сходится к точке $x\in M$, то и любая подпоследовательность $\{{x}_{{n}_{k}}\}$ последовательности $\{{x}_{n}\}$ сходится к этой же точке.\\
	\textbf{Утверждение 13.2} Последовательность точек $\{{x}_{n}\}$ метрического пространства $M$ может сходится не более чеи к одному  пределу.\\
	\textbf{Утверждение 13.3} Если последовательность точек $\{{x}_{n}\}$ метрического пространства $M$ сходится к точке $x\in M$, то эта последовательность ограничена в том смысле, что числа $\rho({x}_{n},z)$ ограничены для любой фиксированной точки $z\in M$.\\
	\\
	Haзoвeм \textbf{шapoм(замкнутым шаром)} с центром в точке $a\in M$ и радиусом $r$ совокупность точек пространства $M$, удовлетворяюших неравенству $\rho(x,a)<r(\rho(x,a)\le r)$. Будем обозначать такой шар $B(a,r)(\overline{B(a,r)})$. \\
	Назовем \textbf{окрестностью} точки любой шар с центром в этой точке. Множество, лежашцее целиком внутри некоторого шара, называется \textbf{ограниченным}.\\
	Пусть дано множество $X$ метрического пространства $M$. Точка $a\in M$ называется \textbf{предельной точкой} этого множества, если любая окрестность точки $a$ содержит хотя бы одну точку множества $X\setminus \{a\}$, т.е. множество $B(a,r)\cap[X\setminus \{a\}]$ не пусто для любого $r$.\\
	Множество, полученное присоединением к $X$ всех его предельных точек, называется \textbf{замыканием} множества $X$ и обозначается $\overline{X}$.\\
	Множество называется \textbf{замкнутым}, если $X=\bar{X}$. Множество $X$ называется \textbf{откpытым}, если его дополнение $M\setminus X$ замкнуто. \\
	Множество $X$ называется \textbf{всюду плотным в пространстве $M$}\footnote{пример см. лек.13 стр.2},  если $\bar{X}=M$. Мнохество $X$ называется \textbf{нигде не плотным в пространстве $M$}, если каждый шар этого пространства содержит в себе шар, свободный от точек множествa $X$.\\
	\\
	Если в метрическом пространстве $M$ каждая фунтаментальная последовательность сходится к некоторому пределу, являющемуся элементом того же пространства, то пространство $M$ называется \textbf{полным}.\\
	Пусть $p\ge 1$ и дано множество числовых последовательностей $x=\{{\xi}_{i}\}$ таких, что $\sum_{i=1}^{\infty} {|{\xi}_{i}|}^{p} < \infty$. Это множество обозначается ${l}_{p}$. Метрика для элементов $x=\{{\xi}_{i}\}$ и $y=\{{\eta}_{i}\}$ вводится по формуле $\rho(x,y)={(\sum_{i=1}^{\infty} {|{\xi}_{i}-{\eta}_{i}|}^{p})}^{1/p}$.\\
	\textbf{Утверждение 13.4} Пространство ${l}_{p}, p\ge 1$ является полным.\\
	\\
	\textbf{Теорема 13.1 Теорема о вложенных шарах} Пусть дана в полном метрическом пространстве $M$ последовательность замкнутых шаров, вложенных друг в друга(т.е. таких, что каждый последующий шар содержится в предыдущем), радиусы которых стремятся к нулю\footnote{Это условие существенно, см. лек.14 стр.2}. Тогда существует и причем единственная точка, принадлежащая всем этим шарам.\\

\subsection{Теорема Бэра о категориях. Принцип сжимающих отображений. Лек.14-15 \textbf{Вопрос 14}}
	Множество $M$ называется \textbf{множеством 1-ой категории}, если оно может быть представлено в виде суммы не более чем счетного числа нигде неплотных множеств.\\
	Множество, неявляющееся множеством 1-ой категории, называется \textbf{множеством 2-ой категории}\footnote{Пример множества 1-ой категории: множество всех рациональных чисел - одна точка - нигде не плотное множество. Пример множества 2-ой категории: множество всех иррациональных чисел: Если было бы 1-ая категории то и вся прямая будет 1-ой ка-ри(сдедует из "объединение счетных множество - счетное множество"), а это не так, так как Теорема о категориях.}.\\
	\\
	\textbf{Теорема 14.1 Теорема Бэра о категориях} Полное метрическое пространство есть множество 2-ой категории.\\
	\\
	\textbf{Теорема 14.2 Принцип сжимающих отображений}\footnote{Пример существенности условия полноты: $M=(0,1)$,$\rho(x,y)=|x-y|$, $Ax=x/2$. Не существует неподвижная точка, так как $Ax=x, x=0\notin M$.\\
	Пример существенности условия сжатия: $M=(-\infty,+\infty)$, $\rho(x,y)=|x-y|$, $Ax=x+pi/2-arctgx$, $|Ax-Ay|=|(x-y)-(arctgx-arctgy)|<|x-y|$. (так как при $x>y,(x-y)>(arctgx-arctgy)>0$, при $y>x$, $(x-y)<(arctgx-arctgy)<0), \rho(Ax,Ay)\le \rho(x,y)$ т.е. $\alpha=1$. но $x=x+\pi/2-arctgx$, $arctgx =\pi/2$ не имеет решение.} Пусть в полном мeтрическое пространстве $M$ задан оператор $A$, переводящий элементы пространства $M$ в элементы этого же пространства. Пусть, кроме того, $\rho(A(x),A(y))\le \alpha\rho(x,y)$, где $\alpha \in [0,1)$\footnote{Условие сжимаемости зависит от выбора метрик см. лек.15 стр.3}. Тогда существует и притом единственная точка ${x}_{0}$ такая, что $A{x}_{0}={x}_{0}$. Эта точка называется \textbf{неподвижной точкой} оператора $A$, а сам оператор называется \textbf{оператором сжатия}.\\ 
\subsection{Линейные нормированные пространства. Теорема Рисса. Лек.15 \textbf{Вопрос 18}}
	Пусть $X$ - линейное пространство над полем действительных или комплексных чисел. Это пространство $X$ называется \textbf{линейным нормированным пространством}, если каждому элементу $x\in X$ поставлено в соответствии действительное число $||x||$, называемое \textbf{нормой} этого элемента, причем выполнены следующие аксиомы:\\
	1) $||x||\ge 0, ||x||=0 \Leftrightarrow x=0$,\\
	2) $||\lambda x||=\lambda||x||$,\\
	3) $||x+y||\le ||x||+||y||$\\
	для любых элементов $x,y\in X$и любого числа $\lambda$ из поля.\\
	\\
	Последовательность элементов $\{{x}_{n}\}$ линейного нормированного пространства $X$ называется \textbf{фунтаментальной}, если $||{x}_{n}-{x}_{m}||\rightarrow0$ при $n,m\rightarrow0$.\\
	Последовательность элементов $\{{x}_{n}\}$ линейного нормированного пространства $X$ называется \textbf{сходящейся к элементу} $x\in X$, если $||{x}_{n}-x||\rightarrow0$ при $n\rightarrow0$.\\
	Линейное нормированное пространство $X$ называется \textbf{полным(банаховым)}, если любая его фунтаментальная последовательность является сходящейся.\\
	Линейное многообразие\footnote{В этом курсе под линейным многообразием понимаем как линейное пространство внутри другого более широкого линейного пространства.} $L$ линейного нормированного пространства $X$ называется \textbf{подпространством}, если множество $L$ замкнуто относительно сходимости по норме.\\
	Из третьего аксиома нормы вытекает что, если${x}_{n}\rightarrow x$, то $||{x}_{n}||\rightarrow ||x||$.\\
	\\
	\textbf{Теорема 15.1 Теорема Рисса} Пусть $L$ - подпространство линейного нормированного пространства $X$, $L\ne X$. Тогда для любого $\varepsilon \in (0,1)$ существует элемент $y\in X\setminus L, ||y||=1$ и такой, что $||x-y||>1-\varepsilon$ для всех $x\in L$.\\
\section{Линейные операторы}
\subsection{Линейные операторы и их свойства. Теорема о полноте пространства линейных ограниченных операторов. Лек.16 \textbf{Вопрос 19}}
	Пусть $X$ и $Y$ – линейные нормированные пространства над полем действительных или полем комплексных чисел.\\
	Отображение $A:X\rightarrow Y (y = Ax)$, то есть оператор $A$,
	определяемый на $X$ с областью значений в $Y$, называется \textbf{линейным оператором}, если для любых элементов ${x}_{1},{x}_{2}\in X$ и любого числа $\lambda$ справедливы равенства\\
	а) $A({x}_{1}+{x}_{2}) = A({x}_{1})+A({x}_{2})$,\\
	б) $A(\lambda {x}_{1}) = \lambda A({x}_{1})$\\
	\\
	Оператор $A:X\rightarrow Y$ \textbf{непрерывен} в точке ${x}_{0}\in X$, если для любой последовательности $\{{x}_{n}\}$ сходящейся к ${x}_{0}$, соответствующая последовательность образов $\{{Ax}_{n}\}$ сходится к элементу ${Ax}_{0}$, то есть для любого $\varepsilon>0$ существует $\delta>0$ и такое, что как только выполняется неравенство ${||{x}_{n}-{x}_{0}||}_{X}<\delta$ будет выполняться неравенство ${||{Ax}_{n}-{Ax}_{0}||}_{Y}<\varepsilon$.\\
	\textbf{Теорема 19.1} Линейный оператор $A$ непрерывен на всем пространстве $X$ тогда и только тогда, когда $A$ - непрерывен в одной точке ${x}_{0}\in X$.\\
	\\
	Оператор $A$ называется \textbf{ограниченным}, если существует постоянная $M$ такая, что оценка $||Ax||\le M||x||$ выполняется для всех ${x}_{0}\in X$.\\
	\textbf{Теорема 19.2} Для того чтобы линейный оператор $A$ был непрерывен необходимо и достаточно, чтобы $A$ был ограничен.\\
	\\
	Наименьшая из постоянных $M$, удовлетворяющих
	условию $||Ax||\le M||x||$ для линейного ограниченного оператора $A$ называется \textbf{нормой оператора} $A$ и обозначается $||A||$. Другими словами $||A||=\sup_{x\ne 0} \frac{||Ax||}{||x||}$.\\
	Норму линейного ограниченного оператора $A$ можно вычислить по формуле $||A||=\sup_{||x||\le 1} ||Ax||$.\\
	\\
	Совокупность\footnote{\textbf{Замечание 1.} Множество $E\rightarrow E$ всевозможных линейных операторов образует некоммутативное кольцо. см. лек.19 стр.3\\
	\textbf{Замечание 2.} Пусть $X$ и $Y$ банаховы пространства с нормами ${||x||}_{X}$ и ${||y||}_{Y}$. Определим в пространстве $X\times Y$ функцию $||(x,y)||={||x||}_{X}+{||y||}_{Y}$. Это функция является нормой в пространстве $X\times Y$, а пространство $X\times Y$ является банаховым. см. лек.19 стр.4} всех линейных ограниченных операторов, отображающих
	линейное нормированное пространство $X$ в линейное нормированное
	пространство $Y$, образует линейное пространство $L(X\rightarrow Y)$. Если $A$ и $B$ - линейные ограниченные операторы, то равенство $(A+B)x=Ax+Bx$ определяет \textbf{сумму операторов}, а $(\lambda A)x=\lambda Ax$ - \textbf{умножение оператора} на число. \textbf{Нулем} этого пространства является оператор $0x=0$ для любого $x\in X$. В $L(X\rightarrow Y)$ можно ввести норму\footnote{Справедливость аксиомов см. лек.16 стр.3} $||A||=sup_{||x||\le 1} ||Ax||$. Таким образом $L(X\rightarrow Y)$ - линейное нормированное пространство.
	Если линейный ограниченный оператор действует из линейного
	нормированного пространства $X$ на числовую прямую ${\mathbb{R}}_{1}$, то такой оператор называется \textbf{линейным функционалом} $f(x)$. Совокупность всех линейных функционалов, действующих из Х называется \textbf{сопряженным пространством} к $X$ и обозначается ${X}^{*}=L(X\rightarrow {\mathbb{R}}_{1})$. Норма линейного функционала вычисляется по формуле $||f||=\sup_{||x||\le 1} |f(x)|$.\\
	
	\textbf{Теорема 19.3} Если $X$ - линейное нормированное пространство, а $Y$ - банахово пространство (полное линейное нормированное пространство), то пространство $L(X\rightarrow Y)$ также будет полным, то есть банаховым.\\
	\textbf{Следствие к Теореме 19.3} Пространство ${X}^{*}$, сопряженное к линейному нормированному пространству $X$ - банахово.\\
\subsection{Теорема Банаха-Штейнгауза (принцип равномерной ограниченности) и следствие из нее. Пример применения теоремы в теории рядов Фурье. Лек.17 \textbf{Вопрос 20}}
	\textbf{Теорема 20.1 Теорема Банаха-Штейнгауза (принцип равномерной ограниченности)} Пусть $X$ и $Y$ - банаховы пространства. Если ${A}_{n}\in L(X\rightarrow Y)$ и последовательность $\{||{A}_{n}(x)||\}$ ограничена для любого элемента $x$ из пространства $X$, то найдется постоянная $C$ такая, что $||{A}_{n}||<C$, то есть числовая последовательность $\{||{A}_{n}||\}$ ограничена.\\
	\textbf{Следствие к Теореме 20.1} Пусть $X$ и $Y$ - банаховы пространства, ${A}_{n}\in L(X\rightarrow Y)$, существует последовательность $\{{x}_{n}\}$ такая, что $||{x}_{n}||<1$ и $\lim_{n \rightarrow \infty} ||{A}_{n}{x}_{n}||=\infty$. Тогда существует ${x}_{0}\in X$, $||{x}_{0}||<1$ и $\overline{\lim}_{n \rightarrow \infty} ||{A}_{n}{x}_{0}||=\infty$.\\
	\\
	Используя теорему Банаха-Штейнгауза в теории рядов Фурье, мы можем доказать существование непрерывной периодической функции, для которой ряд Фурье расходиться.\footnote{см. лек.17 стр.2}\\
\section{Обратный оператор}
\subsection{Обратный оператор. Достаточные условия существования обратного оператора. Лек.18 \textbf{Вопрос 21}}
	Линейный оператор $C$ называется \textbf{левым обратным} для линейного
	оператора $A$, если $CA=I$, где $I$ - тождественный оператор.\\
	Линейный оператор $B$ называется \textbf{правым обратным} для линейного оператора $A$, если $AB=I$, где $I$ - тождественный оператор.\\
	Если $B=C$, то говорят, что оператор $A$ имеет
	\textbf{обратный оператор}, который обозначают ${A}^{-1}$. Если ${A}^{-1}$ существует, то по определению ${A}^{-1}A=A{A}^{-1}=I$.\\
	\\
	С понятием обратного оператора связаны вопросы о
	существовании и единственности решения операторных уравнений
	вида $Ax=y$, где $y$ - известный элемент пространства $E$, а $х$ - искомый элемент того же пространства. Если существует левый оператор $C$, то если существует решение оно обязательно единственно, но решение может быть не существовать. Если существует правый оператор $B$, то обязательно существует решение, но оно может быть неединственно.\\
	\\
	\textbf{Другое определение обратного оператора} Пусть оператор $A$ действует из множества $X$ на множество $Y$,
	$R(A)\subset Y$- область значений оператора $A$. Если для любого элемента $y\in R(A)$ уравнение $Ax=y$ имеет единственное решение, то говорят, что оператор $A$ имеет обратный оператор ${A}^{-1}$.\\
	Если $A$ – линейный оператор, то и ${A}^{-1}$ - линейный оператор.\\
	\\
	\textbf{Теорема 21.1 Достаточные условия существования обратного оператора} Пусть $A$ – линейный оператор, действующий из
	линейного нормированного пространства $X$ в линейное
	нормированное пространство $Y$, причем существует постоянная $m>0$
	такая, что $||Ax||\ge m||x||$ для всех $x\in X$. Тогда существует ${A}^{-1}$ - линейный ограниченный оператор.\\
	\textbf{Теорема 21.2 Теорема Неймана} Пусть $A$ – линейный
	ограниченный оператор, отображающий банахово пространство $X$ на
	себя и $||A||\le q<1$ Тогда оператор $I-A$ имеет обратный линейный
	ограниченный оператор ${(I-A)}^{-1}$.\\
	\textbf{Замечание} Пусть $A,B\in L(X\rightarrow X)$. Тогда определен оператор $AB\in L(X\rightarrow X)$ по формуле $ABx=A(Bx)$, причем $||AB||\le ||A||||B||$.\\
	\textbf{Теорема 21.3} Пусть оператор $A\in L(X\rightarrow X)$, где $X$ – банахово пространство, имеет обратный оператор ${A}^{-1}$ и существует линейный ограниченный оператор $\Delta A$ такой, что $||\Delta A||\le \frac{1}{||{A}^{-1}||}$. Тогда оператор $B=A+\Delta A$, то есть возмущение оператора $A$, имеет обратный оператор ${B}^{-1}$, причем 
	\begin{equation}
	||{B}^{-1}-{A}^{-1}||\le \frac{||\Delta A||{||{A}^{-1}||}^{2}}{1-||\Delta A||||{A}^{-1}||}
	\end{equation}
\subsection{Теорема Банаха об обратном операторе. Лек.18 \textbf{Вопрос 22}}
	\textbf{Теорема 22.1 Теорема Банаха об обратном операторе}\footnote{Доказательство см. лек.18} Если линейный ограниченный оператор $A$ отображает банахово
	пространство $X$ на банахово пространство $Y$ взаимно однозначно, то
	существует линейный ограниченный оператор ${A}^{-1}$, обратный к
	оператору $A$, отображающий $Y$ на $X$.\\
\section{Линейные функционалы}
\subsection{Теорема Хана-Банаха о продолжении линейного непрерывного функционала в линейном нормированном пространстве. Лек.19-20,24 \textbf{Вопрос 23}}
	\textbf{Теорема 23.1} Пусть $X$ и $Y$ - банаховы пространства, $A: X\rightarrow Y$ - линейный оператор. Тогда следуюшцие утверждения равносильны:\\
	1) $A$ непрерывен,\\
	2) Eсли $\left\{x_{n}\right\}$ - последовательность элементов из $X$, такая что $x_{n}\rightarrow x$ и $Ax_{n}\rightarrow y$, то $Ax=y$.\\
	\\
	\textbf{Теорема 23.2 Теорема Хана-Банаха} Любой линейный функционал $f(x)$, определённый на линейном многообразии $L\subset X$ линейного
	нормированного пространства $X$, можно продолжить на всё пространство
	$X$ с сохранением нормы, то есть существует линейный функционал $F(x)$, определённый на всём $X$ и такой, что $F(x)=f(x)$ для любой точки $x\in L$, ${||F||}_{X}={||f||}_{L}$.\\
	\\
	Пространство $X$ называется \textbf{сепарабельным}\footnote{Пример: $C[a,b]$. Так как существует множество многочленов с рациональными коэффициентами, которое является счетным множеством. \\
	Но $C(a,b)$ - это не сепарабельное: пусть $\alpha=({\alpha}_{1},{\alpha}_{2},\ldots,{\alpha}_{n},\ldots), {\alpha}_{i}\in \{0,1\}$. ${f}_{\alpha}(x)=1/{2}^{m} \Rightarrow ||{f}_{\alpha}(x)-{f}_{\beta}(x)||\ge 1$.}, если в этом пространстве существует счётное всюду плотное множество элементов ${x}_{1},{x}_{2},\ldots \in X$.\\
	\\
	\textbf{Следствие 1.}\footnote{Это следствие доказывает существование в любом линейном нормированном пространстве $X$ нетривиальных линейных функционалов, то есть $f(x)\equiv 0$. С другой стороны из следствия 1 вытекает, что если для некоторого элемента ${x}_{0}\in X$ выполнено равенство $f({x}_{0})=0$ для всех $f(x)\in {X}^{*}$, то ${x}_{0}=0$.} Пусть $X$ – линейное нормированное пространство, ${x}_{0}\in X, {x}_{0}\ne 0$. Тогда в $X$ существует линейный функционал такой, что $||f||=1, f({x}_{0})=||{x}_{0}||$.\\
	\textbf{Следствие 2.} Пусть $X$ – линейное нормированное пространство, ${x}_{1},{x}_{2}\in X, {x}_{1}\ne {x}_{2}$. Тогда в $X$ существует линейный функционал такой, что $f({x}_{1})\ne f({x}_{2})$.\\
	\textbf{Следствие 3. Теорема 23.3} Пусть $\{{x}_{n}\}$ - последовательность элементов из банахова пространства $X$ такая, что последовательность $\{f({x}_{n})\}$ ограничена для любого функционала $f(x)\in {X}^{*}$. Тогда существует постоянная $M>0$ и такая, что $||{x}_{n}||\le M$, то есть последовательность $\{{x}_{n}\}$ ограничена в $X$.\\
	\textbf{Следствие 4. Теорема 23.4} Пусть $X$ – банахово пространство, ${f}_{n}\in {X}^{*}$, числовая последовательность $\{{f}_{n}(x)\}$ ограничена для любого элемента $x$, тогда $||{f}_{n}||\le M$.\\
	\textbf{Следствие 5. Теорема 23.5 из лек.24} Пусть в линейном нормированном пространстве $E$ задано линейное многообразие $G$ и элемент ${y}_{0}\in E\setminus G$, находящейся на расстоянии $d>0$ от $G(d=\inf_{x\in G} ||{y}_{0}-x||)$. Тогда существует линейный функционал $f(x)$, определенный всюду на $E$ и такой, что \\
	1) $f(x)=0$ для $x\in G$\\
	2) $f({y}_{0})=1$\\
	3) $||f||=1/d$\\
\subsection{Общий вид линейного функционала в конкретных пространствах. Лек.20 \textbf{Вопрос 24}}
	\begin{enumerate}
		\item Если $X={\mathbb{R}}^{n}$ - конечномерное и ${e}_{1},{e}_{2},\ldots,{e}_{n}$ - ортонормированный базис, то $x=\sum_{i=1}^{n} \xi_{i}{e}_{i}$. Тогда любой линейный функционал $f(x)=\sum_{i=1}^{n} \xi_{i}f({e}_{i})=\sum_{i=1}^{n} \xi_{i}{f}_{i}$ однозначно определяется числами ${f}_{i}=f({e}_{i}), i=\overline{1,n}$.
		\item Если $X={l}_{p},p>1$ - бесконечномерное пространство элементов $x=({\xi}_{1},{\xi}_{2},\ldots)$ таких, что ${||x||}_{p}={(\sum_{i=1}^{\infty} {|{\xi}_{i}|}^{p})}^{1/p}<+\infty$. Пусть ${e}_{1},{e}_{2},\ldots$- ортонормированный базис,
		тогда $x=\sum_{i=1}^{\infty} \xi_{i}{e}_{i}$, $f(x)=\sum_{i=1}^{\infty} \xi_{i}f({e}_{i})=\sum_{i=1}^{\infty} \xi_{i}{c}_{i}$, причём $||f||={||c||}_{q},{l}_{p}^{**}={l}_{q}^{*}={l}_{p}$.
		\item Если $X={L}_{p}(E),p>1,|E|<+\infty$, Можно показать, что $f(x)=\int_{E} x(t)\alpha(t)dt,x(t)\in {L}_{p}(E),\alpha(t)\in {L}_{p}(E)$ - однозначно определяемая функция по
		функционалу $f(x)$, причём $||f||={||\alpha(t)||}_{{L}_{p}(E)},{L}_{p}^{**}={L}_{q}^{*}={L}_{p}$.
		\item Если $X=C[0,1]$, то любой линейный функционал $f(x)$ на $C[0,1]$ имеет вид $f(x)=\int_{0}^{1} x(t)dh(t)$, где $x(t)\in C[0,1]$, $h(t)$ - фиксированная функция с ограниченным изменением: $||f||={\sup}_{T} \sum_{i=1}^{n} |h({t}_{i}-h({t}_{i-1}))|$, где точная верхняя грань берётся по
		всевозможным разбиениям $T=\{{t}_{i}\},0={t}_{0}<{t}_{1}<\ldots<{t}_{n}=1$.
	\end{enumerate}
\subsection{Слабая сходимость. Связь между сильной и слабой сходимостью. Критерий сильной сходимости. Лек.20-21,24 \textbf{Вопрос 25}}
	Последовательность $\{{x}_{n}\}$ элементов линейного
	нормированного пространства $X$ называется \textbf{слабо сходящейся} к
	элементу ${x}_{0}\in X$, если для любого линейного функционала $f(x)\in {X}^{*}$ числовая последовательность $\{f({x}_{n})\}$ сходится к $f({x}_{0})$.\\
	В силу замечания к следствию 1 из теоремы Хана-Банаха слабый предел
	единственен. \\
	Из теоремы 23.3 вытекает ограниченность слабо сходящейся
	последовательности.\\
	\\
	Сильная сходимость влечёт за собой слабую сходимость, так как $|f({x}_{n})-f({x}_{0})|\le ||f||||{x}_{n}-{x}_{0}||$. Обратное неверно. См. лек.20 стр.5\\
	\textbf{Теорема 25.1 Критерий сильной сходимости} Последовательность $\{{x}_{n}\}$ линейного нормированного
	пространства $X$ сходится сильно тогда и только тогда, когда
	последовательность $\{f({x}_{n})\}$ сходится равномерно в единичном шаре $||f||\le 1,f\in {X}^{*}$.\\
	\\
	\textbf{Некоторые выводы о слабой сходимости и непрерывности функционала}\\
	\textbf{Теорема 25.2} Если последовательность элементов  $\{{x}_{n}\}$ банахова пространства $X$ сходится слабо, то последовательность норм этих элементов $\{||{x}_{n}||\}$ ограничена.\\
	\textbf{Теорема 25.3} Линейный функционал непрерывен в линейном нормированном пространстве тогда и только тогда, когда его ядро замкнуто.\\
	\textbf{Теорема 25.4} Линейный функционал, непринимающий в некотором шаре линейного нормированного пространства хотя бы одно значение, непрерывен.\\
	\textbf{Теорема 25.5 из лек.24} В конечномерном пространстве сильная сходимость совпадает со слабой.\\
	\textbf{Теорема 25.6 из лек.24} Если последовательность $\{{x}_{n}\}$ слабо сходится к ${x}_{0}$, то существует последовательность линейных комбинаций $\{\sum_{i=1}^{m} {\lambda}_{i}{x}_{i}\}$, сильно сходящаяся к ${x}_{0}$.
\section{Гильбертовы пространства}
\subsection{Определение гильбертова пространства и его основные свойства. Теорема об элементе с наименьшей нормой. Лек.22 \textbf{Вопрос 26}}
	\textbf{Гильбертовым пространством} $H$ называется множество
	элементов $x,y,z,\ldots$ со свойствами:\\
	1) $H$ – линейное пространство над полем действительных (комплексных)
	чисел;\\
	2) каждой паре $x,y\in H$ поставлено в соответствие действительное
	(комплексное) число $(x,y)$, называемое \textbf{скалярным произведением} и удовлетворяющее условиям:\\
	а) $(x,y)=(y,x)$,\\
	б) $(x+z,y)=(x,y)+(z,y)$,\\
	в) $(\lambda x,y)=\lambda(x,y)$ для любого $\lambda \in \mathbb{R}$($\lambda \in \mathbb{C}$),\\
	г) $(x,x)\ge 0$, причем $(x,x)=0$ тогда и только тогда, когда $x=0$,\\
	$||x||=\sqrt{(x,x)}$ - норма элемента $x$ в $H$;\\
	3) $H$ – полное в метрике $\rho(x,y)=||x-y||$, то есть является банаховым пространством;\\
	4) $H$ – бесконечномерное, то есть для любого натурального числа
	$n$ существует $n$ линейно независимых элементов.\\
	\textbf{Примеры}\\
	1. Комплексное пространство ${l}_{2}$ - гильбертово пространство со скалярным произведением $(x,y)=\sum_{i=1}^{\infty} {\xi}_{i}\overline{{\eta}_{i}}$, где $x=({\xi}_{1},{\xi}_{2},\ldots),y=({\eta}_{1},{\eta}_{2},\ldots)$.\\
	2. Пространство ${L}_{2}(E)$ - гильбертово пространство со
	скалярным произведением $(x,y)=\int_{E}x(t)\overline{y(t)}dt$.\\
	\textbf{Свойства гильбертова пространства(скалярного произведения)}\\
	1. $(x,y+z)=(x,y)+(x,z)$.\\
	2. $(x,\lambda y)=\overline{\lambda}(x,y)$.\\
	3. $|(x,y)|\le ||x||||y||$ - неравенство Коши-Буняковского.\\
	4. $||x+y||\le ||x||+||y||$- неравенство треугольника.\\
	5. ${||x+y||}^{2}+{||x-y||}^{2}=2{||x||}^{2}+2{||y||}^{2}$ - равенство параллелограмма\footnote{Если равенство параллелограмма не выполняется, то пространство не гильбертово. Это необходимое условие. Если $X$ - банахово пространство, и выполняется равенство параллелограмма, то вводя скалярное произведeние $(x,y)=1/4({||x+y||}^{2}+{||x-y||}^{2})$ получим Н.}.\\
	\\
	\textbf{Теорема 26.1 Теорема об элементе с наименьшей нормой} Замкнутое выпуклое\footnote{Множество $W$ выпуклое, если $\frac{1}{2}({x}_{1}+{x}_{2})\in W$, для любого ${x}_{1},{x}_{2}\in W$.} множество $W$ в гильбертовом пространстве $H$ содержит элемент с наименьшей нормой, и причем только один.\\
\subsection{Теорема Леви об ортогональной проекции. Разложение гильбертова пространства на прямую сумму подпространства и его ортогонального дополнения. Лек.22 \textbf{Вопрос 27}}
	Два элемента $x,y\in H$ называются \textbf{ортогональными}($x\perp y$), если $(x,y)=0$; говорят, что элемент $x\in H$ \textbf{ортогонален множеству} $L\subset H$, если $x\perp y$ для любого $y\in L$.\\
	\textbf{Теорема 27.1 Теорема Леви об ортогональной проекции} Пусть $L$ – подпространство $H$. Каждый вектор $x\in H$ допускает единственное представление $x=y+z,y\in L,z\perp L$ причем элемент
	$y$ осуществляет наилучшее приближение вектора $x$ в подпространстве $L$, то есть $||x-y||=\min_{u\in L}||x-u||$.\\
	\\
	Элементы, ортогональные к $L$, образуют подпространство, которое называется \textbf{ортогональным дополнением}\footnote{Предел последовательности элементов, ортогональных подпространству $L$, ортогонален $L$.} к $L$ и обозначается ${L}^{\perp}$.\\
	Так как любой элемент $x\in H$ равен $x=y+z,y\in L,z\perp L$, то говорят, что пространство $H$ \textbf{разлагается в прямую сумму} подпространств $L$ и ${L}^{\perp}$. Записывают этот факт в виде $H=L\oplus {L}^{\perp}$. \\
	Элемент $y$ называется \textbf{ортогональной проекцией} элемента $x$ на подпространство $L$.\\
	Оператор $P$, действующий по закону $y = Px$, то есть каждому элементу $x\in H$ ставящий в соответствие его проекцию $y$, называется \textbf{оператором ортогонального проецирования} или \textbf{ортопроектором}. Нетрудно проверить, что ${{L}^{\perp}}^{\perp}=L$.\\
\subsection{Теорема Рисса-Фреше об общем представлении линейного функционала в гильбертовом пространстве. Лек.23 \textbf{Вопрос 28}}
	Рассмотрим линейный ограниченный оператор, действующий из
	гильбертова пространства $H$ на комплексную плоскость $\mathbb{C}$. Этот оператор мы также будем называть \textbf{линейным функционалом}. Обозначим через $\ker f={x\in H: f(x)=0}$ - множество, называемое \textbf{ядром} функционала $f(x)$. Очевидно, что $\ker f$ – подпространство $H$.\\
	\textbf{Лемма 28.1} Для любого нетривиального $f(x)$ в $H$, $\dim{(\ker f)}^{\perp}=1$.\\
	\textbf{Теорема 28.1 Теорема Рисса-Фреше} Любой линейный функционал $f(x)$ в гильбертовом пространстве $H$ представим в виде скалярного произведения $f(x)=(x,y)$, где элемент $y$ однозначно определяется по функционалу $f(x)$, причем $||f||=||y||$.\\
	\textbf{Лемма 28.2} Для того, чтобы линейное многообразие $M$ было всюду плотно в $H$, необходимо и достаточно, чтобы в $H$ не существовало элемента, отличного от нуля и ортогонального $M$.\\
\subsection{Ортонормированные системы. Ортогонализация по Шмидту. Неравенство Бесселя. Полнота и замкнутость ортонормированной системы. Слабая сходимость ортонормированной системы к нулю. Лек.23 \textbf{Вопрос 29}}
	Система $\{{e}_{n}\}$ элементов гильбертова пространства $H$ называется \textbf{ортонормированной}, если $({e}_{i},{e}_{j})=\sigma_{ij}$, где $\sigma_{ij}$ - символ Кронекера.\\
	Бесконечная система элементов линейного пространства
	называется \textbf{линейно независимой}, если любая конечная подсистема этой системы - линейно независима.\\
	\textbf{Лемма 29.1 Ортогонализация по Шмидту} Любую систему $\{{h}_{n}\}$ линейно независимых элементов можно
	сделать ортонормированной с помощью процесса ортогонализации Шмидта.\\
	\\
	Если совокупность степеней $1,t,{t}^{2},\ldots$ ортогонализировать в пространстве $L_{2,\rho}(a,b)$ с весом $\rho(t)$, то есть в пространстве со скалярным произведением
	\begin{equation}
	\begin{aligned}
	(x(t),y(t))=\int_{a}^{b} \rho(t)x(t)y(t)dt
	\end{aligned}
	\end{equation}
	мы придем к системе полиномов.\\ 
	При $\rho(t)\equiv 1,a=-1,b=1$ получим \textbf{полиномы Лежандра},\\ 
	При $\rho(t)= {e}^{-{t}^{2}},a=-\infty,b=\infty$ получим \textbf{полиномы Чебышева–Эрмита},\\
	При $\rho(t)= {e}^{-t},a=0,b=\infty$ получим \textbf{полиномы Чебышева–Лагерра}.\\
	\\
	Пусть $L\subset H$ – подпространство, порожденное ортонормированной системой $\{{e}_{n}\}$ и $x\in L$. \textbf{Равенство Пасерваля} $\sum_{i=1}^{\infty} {|{c}_{i}|}^{2}={||x||}^{2}, {c}_{i}=(x,{e}_{i})$.\\
	Пусть теперь $x$ – любой элемент из $H$. \textbf{Неравенство Бесселя} $\sum_{i=1}^{\infty} {|{c}_{i}|}^{2}\le {||x||}^{2}, {c}_{i}=(x,{e}_{i})$.\\
	\\
	Ортонормированная в $H$ система $\{{e}_{n}\}$ называется \textbf{полной}, если в $H$ не существует никакого элемента кроме нуля, ортогонального каждому члену ${e}_{n}$ системы $\{{e}_{n}\}$.\\
	Система $\{{e}_{n}\}$ называется \textbf{замкнутой}, если подпространство $L$, порожденное этой системой, совпадает с $H$.\\
	Ряд Фурье по замкнутой системе, построенный для любого элемента $x\in H$ сходится к нему сильно, то есть по норме, и выполняется
	равенство Парсеваля $\sum_{i=1}^{\infty} {|{c}_{i}|}^{2}={||x||}^{2}, {c}_{i}=(x,{e}_{i})$.\\
	Замкнутая ортонормированная система называется
	\textbf{ортонормированным базисом} в гильбертовом пространстве $H$.\\
	\\
	\textbf{Два полезные утверждения}\\
	1. В гильбертовом пространстве $H$ полнота и замкнутость ортонормированной системы совпадают.\\
	2. Любая ортонормированная система $\{{e}_{n}\}$ в гильбертовом пространстве слабо сходится к нулю.\\
	\\
	\textbf{Ещё полезные понятия}\\
	\textbf{Носителем} функции $h(x)$ называется замыкание множества точек, в которых она отлична от нуля, т.е. $\operatorname{supp}\,h =\overline{\{x:f(x)\ne 0\}}$.\\
	Назовем функцию $g(x)$ из ${L}_{1}(a,b)$ \textbf{обобщенной производной функции} $f(x)$ из ${L}_{1}(a,b)$, если она удовлетворяет
	равенству
	\begin{equation}
	\begin{aligned}
	\int_{a}^{b} f(x)h'(x)dx=-\int_{a}^{b} g(x)h(x)dx
	\end{aligned}
	\end{equation}
	для любой бесконечно дифференцируемой на сегменте $[a,b]$ функции $h(x)$ с носителем $\operatorname{supp}\,h$, принадлежащим интервалу $(a,b)$.\\
	
	Определим пространство $W_{2}^{1}(a,b)$ как множество всех функций из ${L}_{2}(a,b)$, у которых обобщенные производные так же принадлежат классу ${L}_{2}(a,b)$. Норма в этом пространстве вводится по формуле 
	\begin{equation}
	\begin{aligned}
	{||u(x)||}_{W_{2}^{1}}=\sqrt(\int_{a}^{b} {u}^{2}(x)dx+\int_{a}^{b} {[u'(x)]}^{2}dx)
	\end{aligned}
	\end{equation}
	Это пространство гильбертово и называется \textbf{пространством Соболева}. Его можно получить как пополнение по этой норме множества бесконечно дифференцируемых на сегменте $[a,b]$ функций.\\
	Если рассматривать пополнение по этой норме множества бесконечно дифференцируемых на сегменте $[a,b]$ функций с носителем из интервала $(а,b)$, то получается пространство, которое
	обозначается как $\overset{0}{W_{2}^{1}}$.
\subsection{Теорема о существовании ортонормированного базиса в сепарабельном гильбертовом пространстве. Теорема об изоморфизме и изометрии всех сепарабельных гильбертовых пространств над одним полем. Лек.25 \textbf{Вопрос 30}}
	\textbf{Теорема 30.1 Теорема о существовании ортонормированного базиса в сепарабельном гильбертовом пространстве}\footnote{Следующая теорема существенно для доказательство этой теоремы: Всякое конечномерное подпространство в линейном нормированном пространстве замкнуто. см. лек.27 стр.1} В любом сепарабельном гильбертовом пространстве
	существует ортонормированный базис, то есть полная
	ортонормированная система.\\
	\textbf{Теорема 30.2 Теорема об изоморфизме и изометрии всех сепарабельных гильбертовых пространств над одним полем.} Любое комплексное (вещественное) сепарабельное гильбертово пространство изоморфно и изометрично комплексному (вещественному) пространству ${l}_{2}$, то есть все комплексные (вещественные) сепарабельные гильбертовы пространства изоморфны и изометричны между собой.\\
\subsection{Теорема Рисса-Фишера. Теорема о слабой компактности сепарабельного гильбертова пространства. Лек.25 \textbf{Вопрос 31}}
	\textbf{Теорема 31.1 Теорема Рисса-Фишера} Пространства ${L}_{2}(E)$ и ${l}_{2}$ изоморфны и изометричны, причём $\int_{E} {|f(x)|}^{2}dx=\sum_{i=1}^{\infty} {|{c}_{i}|}^{2}$ , где ${c}_{i}=\int_{E} f(x)\overline{{e}_{i}(x)}dx=(f,{e}_{i})$.\\
	\textbf{Теорема 31.2 Теорема о слабой компактности сепарабельного гильбертова пространства} В сепарабельном гильбертовом пространстве H
	ограниченная последовательность $\{{x}_{n}\}$ содержит слабо сходящуюся подпоследовательность.\\
\section{Сопряженный оператор}
\subsection{Сопряженный оператор. Теорема о сопряженном операторе. Теорема о прямой сумме замыкания образа линейного ограниченного оператора и ядра сопряженного. Лек.26-27 \textbf{Вопрос 32}}
	Пусть $A$ – линейный оператор, действующий из линейного нормированного пространства $X$ в линейное нормированное пространство $Y$, или $y=Ax$. Если $\phi(y)$ - любой линейный функционал, определённый на $Y$, то $f=\phi(Ax)$ - линейный функционал, определенный на $X$: $|f(x)|=|\phi(Ax)|\le ||\phi||||A||||x||$. Таким образом любому линейному функционалу $\phi \in {Y}^{*}$ ставится в
	соответствие линейный функционал $f \in {X}^{*}$, то есть построен оператор, определённый на ${Y}^{*}$ со значениями в ${X}^{*}$. Этот оператор обозначим ${A}^{*}$ и назовём \textbf{сопряжённым}\footnote{Он обладает теми же свойствами сопряженного оператора в лин. алгебре. см. лек.26 стр.3}: $f={A}^{*}\phi$. Если $f(x)=(f,x)$, то $({A}^{*}\phi,x)=(\phi,Ax)$.\\ 
	\textbf{Другое определение (обобщенное)} Оператор ${A}^{*}$ \textbf{сопряжён} к линейному непрерывному оператору $A$, действующему в гильбертовом пространстве $H$, если для любых элементов $x,y\in H$ выполняется равенство $(Ax,y)=(x,{A}^{*}y)$.\\
	Линейный ограниченный оператор называется \textbf{самосопряженным(или эрмитовым)} оператором, если ${A}^{*}=A$.\\
	Свойства самосопряженного оператора: 1) $||A||=\max(|\inf_{||x||=1} (Ax,x)|,|\sup_{||x||=1} (Ax,x)|)=\sup_{||x||=1} |(Ax,x)|$. 2) если для самосопряженного операторов $A$ и $B$ для всех $x\in H$ выполняется $(Ax,x)=(Bx,x)$ то $A=B$.\\
	\\
	\textbf{Теорема 32.1 Теорема о сопряженном операторе}
	Пусть $A\in L(X\rightarrow Y)$. Тогда существует ${A}^{*}\in L({Y}^{*}\rightarrow {X}^{*})$, то есть ${A}^{*}$ - линейный ограниченный оператор, причём $||A||=||{A}^{*}||$.\\
	\\
	Пусть $A\in L(H\rightarrow H)$. Обозначим $ImA=\{y\in H:y=Ax\}$ - \textbf{образ} оператора $A$, $KerA=\{x\in H:Ax=0\}$ - \textbf{ядро} оператора $А$. Если $A$ – ограничен, то $KerA$ - подпространство.\\
	\\
	\textbf{Теорема 32.2} Если $A\in L(H\rightarrow H)$, $H$ - гильбертово пространство, то $H=\overline{ImA}\oplus Ker{A}^{*}$.\\
	\\
	Пусть $A$ - оператор, действующий из конечномерного
	пространства ${\mathbb{R}}_{n}$ в конечномерное пространство ${\mathbb{R}}_{m}$ и определяемый матрицей $\{{a}_{ij}\}$. Cопряженный
	оператор задается матрицей, транспонированной по отношению к матрице
	$A$\footnote{см. лек.26 стр.2}.\\
	Последовательности $\{{x}_{n}\}$ и $\{{z}_{n}\}$ называются \textbf{биортогональными}, если 
	\begin{equation}
	({z}_{n},{x}_{m})=\left\{\begin{array}{l}1, n=m\\ 0, n\ne m\end{array}\right. 
	\end{equation}
	В самосопряженном случае обе биортогональные последовательности совпадают. $y=\sum_{n=1}^{\infty} ({z}_{n},y){x}_{n}$ - этот ряд называется \textbf{рядом Фурье по соответствующим биортогональным
	последовательностям}.\\
\section{Вполне непрерывные и компактные операторы}
\subsection{Вполне непрерывные и компактные операторы. Свойства вполне непрерывного оператора. Лек.28 \textbf{Вопрос 33}}
	Множество $M$ метрического пространства $X$	называется \textbf{предкомпактным (относительно компактным)}, если любая
	последовательность его элементов содержит фундаментальную
	подпоследовательность.\\
	Оператор, действующий из одного метрического пространства в другое называется \textbf{компактным}, если он всякое ограниченное множество переводит в предкомпактное.\\
	Линейный оператор $A$, действующий из линейного
	нормированного пространства $X$ в линейное нормированное пространство $Y$	называется \textbf{вполне непрерывным}, если он слабо сходящуюся
	последовательность переводит в сильно сходящуюся.\\
	\textbf{Ряд утверждений. Разбор см. лек.28 стр.2}\\
	1) Предкомпактное множество ограничено\\
	2) Любое ограниченное множество в конечномерном пространстве
	является предкомпактным\\
	3) Единичная сфера в бесконечномерном пространстве – множество ограниченное, но не предкомпактное\\
	4) Любой ограниченный оператор переводит ограниченное множество в
	ограниченное и предкомпактное в предкомпактное. \\
	5) Вполне непрерывный оператор является ограниченным\\
	6) Любой ограниченный оператор в конечномерном пространстве - компактным.\\
	7) В бесконечномерном пространстве единичный (тождественный) оператор – ограничен, но не компактен\\
	\\
	\textbf{Теорема 33.1 Критерий предкомпактности в $C(E)$} Для того, чтобы
	множество $K$ было предкомпактным в $C(E)$ необходимо и достаточно
	выполнение условий:\\
	1) множество $K$ - равномерно ограничено в $C(E)$ , то есть
	существует постоянная $M$ такая, что ${||x(t)||}_{C(E)}\le M$ для любой функции $x(t)\in K$;\\
	2) множество $K$ - равностепенно непрерывно в $C(E)$ , то есть для
	любого $\varepsilon>0$ найдётся $\delta=\delta(\varepsilon)>0$ и такое, что как только $|h|<\delta$, $x,x+h\in E$, будет выполняться неравенство ${||x(t+h)-x(t)||}_{C(E)}\le \varepsilon$ для любой функции $x(t)\in K$;\\
	\textbf{Теорема 33.2 Критерий предкомпактности в ${L}_{p}(E)$} Для того чтобы множество $K\subset {L}_{p}(E)$, $p\ge 1$, $|E|<+\infty$, было предкомпактно ${L}_{p}(E)$ необходимо и достаточно, чтобы это множество было равномерно ограничено в ${L}_{p}(E)$ и равностепенно непрерывно в ${L}_{p}(E)$.\\
	\textbf{Лемма 33.1} Пусть $X$ – банахово пространство. Если последовательность элементов ${x}_{n}\in X$ сходится слабо к элементу ${x}_{0}\in X$ и предкомпактна, то ${x}_{n}\rightarrow {x}_{0}$ сильно.\\
	\textbf{Теорема 33.3} Пусть $X$ и $Y$ – банаховы пространства. Любой компактный оператор $A$, действующий из $X$ в $Y$, переводит всякую слабо сходящуюся последовательность в $X$ в сильно сходящуюся в $Y$.\\
	\textbf{Теорема 33.4 Свойство вполне непрерывного оператора} Если $A$ – вполне непрерывный оператор, действующий из гильбертова пространства $H$ в $H$, то оператор ${A}^{*}$ также вполне непрерывен.
\section{Теоремы Фредгольма}
\subsection{Первая теорема Фредгольма о разрешимости уравнения $Lx=f$, где $L=I-A$, $I$ - тождественный оператор, $A$ - вполне непрерывный. Лек.29 \textbf{Вопрос 34}}
	Пусть имеется интегральное уравнение $x(t)=\int_{E}K(t,s)x(s)ds+f(t),|E|<+\infty,f(x)\in {L}_{2}(E),K(t,s)\in {L}_{2}(E\times E)$. Рассмотрим теорию уравнений $(I-A)x=f$, где $I$ – тождественный оператор, $A$ – вполне непрерывный оператор, действующий из $H$ в $H,f\in H$. Обозначим $L=I-A$, тогда имеем $Lx=f$.\\
	\textbf{Лемма 34.1} Многообразие $ImL$ – замкнуто, где $ImL=\{y\in H:y=Lx\}$.\\
	\textbf{Лемма 34.2} $H=\ker L\oplus Im{L}^{*},H=\ker {L}^{*}\oplus ImL$.\\
	\textbf{Теорема 34.1 Первая теорема Фредгольма} Уравнение $Lx=f$ разрешимо тогда и только тогда, когда $f\perp \ker {L}^{*}$, то есть элемент $f$ ортогонален любому решению уравнения ${L}^{*}y=0$.\\
\subsection{Вторая теорема (альтернатива Фредгольма). Лек.29 \textbf{Вопрос 35}}
	Для любого натурального числа $k$ положим ${H}^{k}=Im{L}^{k}$, в частности, ${H}^{0}=H=Im{L}^{0}$, где ${L}_{0}=I,{H}^{1}=ImL$ и так далее. По лемме 34.1 все ${H}^{k}$ – замкнуты и очевидно $H\supset {H}^{1}\supset {H}^{2}\supset \ldots$, причем $L({H}^{k})={H}^{k+1}$.\\
	\textbf{Лемма 35.1} Существует номер $j$ такой, что ${H}_{k+1}={H}_{k}$ при всех $k\ge j$.\\
	\textbf{Лемма 35.2} Если $\ker L=0, то ImL=H$. Если $\ker {L}^{*}=0$, то $Im{L}^{*}=H$.\\
	\textbf{Лемма 35.3} Если $ImL=H$, то $\ker L=0$.\\
	\textbf{Теорема 35.1 Вторая теорема Фредгольма (альтернатива Фредгольма)} Либо уравнение $Lx=f$ имеет единственное решение при любой правой части $f\in H$, либо уравнение $Lx=0$ имеет ненулевое решение.\\
\subsection{Третья теорема Фредгольма Лек.29 \textbf{Вопрос 36}}
	\textbf{Теорема 36.1 Третья теорема Фредгольма} Однородные уравнения $Lx=0$ и ${L}^{*}y=0$ имеют одно и то же и при том конечное число линейно независимых решений.\\
\section{Элементы спектральной теории}
\subsection{Спектральная теория в бесконечномерных пространствах. Понятие о точечном, непрерывном и остаточном спектре. Теоремы о непустоте спектра, его замкнутости и ограниченности. Теорема Гильберта-Шмидта. Лек.30-32 \textbf{Вопрос 37}}
	Пусть $X$ – линейное нормированное пространство над полем комплексных чисел. Оператор $A$ есть линейный оператор, действующий из $X$ в $X$.\\
	\textbf{Резольвентное множество} $\rho(A)$ оператора $A$ есть множество комплексных чисел $\lambda$, для которых существует ${(\lambda I-A)}^{-1}$ - ограниченный оператор, определенный на всем $X$. \\
	\textbf{Спектром} $\sigma(A)$ оператора $A$ называется дополнение к множеству $\rho(A)$ на комплексной плоскости, то есть $\sigma(A)=\mathbb{C}\setminus \rho(A)$.\\
	Операторнозначная функция $R(\lambda,A)={(\lambda I-A)}^{-1}$ определенная на множестве $\rho(A)$, называется \textbf{резольвентой} оператора $A$, а $\lambda \in \rho(A)$ называется \textbf{регулярным значением} оператора $A$.\\
	Таким образом\footnote{Эти 3 условия эквивалентны вышесказанному определению:\\
	1) ``Существует ${(\lambda I-A)}^{-1}$" $\Leftrightarrow$ $\ker (\lambda I-A)=\{0\}$. Так как 1. необходимость: допустим $\ker (\lambda I-A)\ne \{0\}$, тогда $(\lambda I-A)x=0$ имеет два разных решения и означает что не существует ${(\lambda I-A)}^{-1}$. 2. достаточность: пусть $\ker (\lambda I-A)=\{0\}$, и предположим $\forall y, \exists {x}_{1}, {x}_{2}: (\lambda I-A){x}_{1}=y, (\lambda I-A){x}_{2}=y$, тогда $(\lambda I-A)({x}_{1}-{x}_{2})=0$ и ${x}_{1}={x}_{2}$, то есть $\forall y$ уравнение $(\lambda I-A)x=y$ однозначно разрешимо.\\
	2) ``Определенный на всем $X$" $\Leftrightarrow$ $Im(\lambda I-A)=X$,\\
	3) ``Ограниченный оператор" $\Leftrightarrow$ ${||\lambda I-A||}^{-1}<+\infty$.}, $\lambda$ - регулярно, если:\\
	1) $\ker (\lambda I-A)=\{0\}$\\
	2) $Im(\lambda I-A)=X$\\
	3) ${||\lambda I-A||}^{-1}<+\infty$\\
	В конечномерном пространстве либо $\ker (\lambda I-A)\ne \{0\}$, то есть $\lambda$ - собственное значение, либо $\lambda$ - регулярное значение.\\
	В бесконечномерном пространстве возможно, что $\ker (\lambda I-A)=\{0\}$, но обратный оператор действует не на всем пространстве\footnote{пример см. лек.30. стр.1}. Кроме того оператор ${(\lambda I-A)}^{-1}$ может быть и неограничен.\\
	В случае $Im(\lambda I-A)=X$ и $X$ - банахово по
	теореме Банаха мы имеем существование ограниченного оператора ${(\lambda I-A)}^{-1}$, то есть условие 3 выполняется автоматически.\\
	\\
	Комплексное число $\lambda$ называется \textbf{собственным значением оператора} $A$,	если $\ker (\lambda I-A)\ne \{0\}$, а любой не равный нулю элемент $x\in \ker (\lambda I-A)$ называется
	\textbf{собственным элементом}, отвечающим собственному значению $\lambda$.\\
	Совокупность чисел $\lambda$, не являющихся собственными значениями, для которых $\overline{Im(\lambda I-A)}$ является собственным подпространством с $X$, называется \textbf{остаточным спектром}.\\
	Совокупность чисел λ из спектра, не являющихся собственными значениями и не принадлежащих остаточному спектру($\overline{Im(\lambda I-A)}=X$), называется \textbf{непрерывным спектром}.\\
	\\
	\textbf{Теорема 37.1} Пусть $X$ - банахово. Резольвентное множество $\rho(A)$ - открыто.\\
	\textbf{Следствие 1. Замкнутость спектра.} Спектр $\sigma(A)$ замкнуто.\\
	\textbf{Следствие 2.} Если $d(\lambda)$ - расстояниe от $\lambda \in \rho(A)$ до спектра $\sigma(A)$, то $||R(\lambda,A)||\ge 1/d(\lambda)$, то есть $||R(\lambda,A)||\rightarrow +\infty$ при стремлении $d$ к нулю. \\
	\textbf{Теорема 37.2 Непустота и ограниченность спектра} Пусть $X$ - банахово. Спектр $\sigma(A)$ - ограниченное и непустое множество, $\sup_{\lambda} |\sigma(A)|=\lim_{n\rightarrow \infty} \sqrt[n]{||{A}^{n}||}\le ||A||$.\\
	\\
	\textbf{Другие выводы}\\
	\textbf{Лемма 37.1 Тождество Гильберта} Для любых $\lambda, \mu \in \rho(A)$ справедливо равенства $R(\lambda,A)-R(\mu,A)=(\mu-\lambda)R(\lambda,A)R(\mu,A)$.\\
	\textbf{Теорема 37.3} Пусть $X$ - бесконечномерное банахово пространство и $A$ - вполне непрерывный оператор, действующий в нем. Тогда $0\in \sigma(A)$, то есть 0 не является регулярным значением.\\
	\textbf{Теорема 37.4} Пусть $H$ - гильбертово пространство и $A$ - вполне непрерывный оператор, действующий в нем. Тогда если $\lambda \in \sigma(A), \lambda \ne 0$, то $\lambda$ является собственным значением конечной кратности.\\
	\textbf{Теорема 37.5} Пусть $H$ - гильбертово пространство и $A$ - вполне непрерывный оператор, действующий в нем. Тогда точка 0 - единственная возможная предельная точка $\sigma(A)$.\\
	\\
	\textbf{Теорема 37.6} Пусть $H$ - гильбертово пространство и $A={A}^{*}$ - самосопряженный непрерывный оператор, действующий в нем. Тогда $||A||=\sup_{||x||=1} |(Ax,x)|$.\\
	${A}^{**}=A$. Если $A$ - произвольный ограниченный оператор в гильбертом пространстве, то ${A}_{1}=\frac{A+{A}^{*}}{2},{A}_{1}=\frac{A-{A}^{*}}{2i}$ - cамосопряженные операторы. То есть, $A={A}_{1}+i{A}_{2}$ есть линейная комбинация самосопряженных операторов.\\
	\textbf{Теорема 37.7} Если $A\in L(H\rightarrow H)$, то $A={A}^{*}$ тогда и только тогда, когда $Im(Ax,x)\equiv 0$.\\
	Пусть $A={A}^{*}:H\rightarrow H$, $H$ - гильбертово пространство, $Ax=\lambda x,x\ne 0$. Тогда $\lambda$ - действительное число и собственные элементы, отвечающие различным собственным значениям, ортогональны. Обозначим $m(A)=\inf_{||x||=1} (Ax,x), M(A)=\sup_{||x||=1} (Ax,x)$.\\
	\textbf{Теорема 37.8} Если $A={A}^{*}$ и вполне непрерывный, то $\sigma(A)\subset [m(A),M(A)]$.\\
	\textbf{Теорема 37.9} Если $A={A}^{*}$ действует в сепарабельном гильбертом пространстве $H$ и вполне непрерывный, то существует собственное значение $\lambda$ оператора $A$ такой, что $|\lambda|=||A||$.\\
	\\
	\textbf{Теорема 37.10 Теорема Гильберта-Шмидта} Если $A={A}^{*}$ и вполне непрерывный в сепарабельном гильбертовом пространстве $H$, то тогда любой элемент $x\in ImA$ представим в виде
	\begin{equation}
	x=\sum_{\lambda_{k} \neq 0}(x,e_{k})e_{k}.
	\end{equation}
	и этот ряд Фурье по ортонормированной системе $\left\{e_{k}\right\}$  собственных элементов оператора $A$ сильно сходится к $x$.\\ 
	
\end{document}

